\section{Lecture 13: Introduction to Quantum Approximate Optimization
Algorithms}\label{sec:lecture13}

\subsubsection*{Review}

\qs{VQA}{
  What varies when running a variational quantum algorithm?
}

\sol{
  Rotation gate angles.
}

\qs{Ansatz in QAOA}{
  What does "ansatz" refer to in Quantum Alternating Operator Ansatz (aka QAOA)?
}

\sol{
  Making a guess about the quantum circuit to use for optimization. In QAOA,
  the ansatz is a parameterized circuit alternating between cost and mixing
  Hamiltonians, designed to approximate the optimal state for a problem, with
  parameters tuned iteratively.
}


%%%%%%%%%%%%%%%%%

\index{Quantum Approximate Optimization Algorithms}
\subsubsection*{Quantum Approximate Optimization Algorithms}

The Quantum Approximate Optimization Algorithm (QAOA) is a hybrid
quantum-classical algorithm within the VQA family, tailored for combinatorial
optimization problems (e.g., \textsc{Max-Cut}, graph coloring). Introduced by
Farhi et al.\footnote{"A Quantum Approximate Optimization Algorithm":
https://arxiv.org/abs/1411.4028}, it approximates the ground state of a cost
Hamiltonian by alternating two types of unitary operators, controlled by
parameters optimized classically.

\vspace{0.3cm}

\noindent
\textbf{Core Idea:}
\begin{itemize}
  \index{Quantum Approximate Optimization Algorithms!Cost Hamiltonian}
  \item \textbf{Cost Hamiltonian ($H_C$)}: Encodes the problem’s objective,
    e.g., $H_C = \sum_{\langle i,j \rangle} Z_i Z_j$ for Max-Cut, where lower
    energy states correspond to better solutions.

  \item \textbf{Mixing Hamiltonian ($H_M$)}: Typically $H_M = \sum_i X_i$,
    drives exploration across the solution space.

  \index{Quantum Approximate Optimization Algorithms!ansatz}
  \item \textbf{Ansatz State}: $\ket{\psi(\gamma, \beta)} = \prod_{p=1}^P
    e^{-i\beta_p H_M} e^{-i\gamma_p H_C} \ket{s}$, where $\ket{s}$ is a
    uniform superposition (e.g., $H^{\otimes n} \ket{0}^{\otimes n}$), and
    $P$ is the number of layers.

  \item \textbf{Goal}: Find $\gamma = (\gamma_1, \ldots, \gamma_P)$ and
    $\beta = (\beta_1, \ldots, \beta_P)$ minimizing $\langle \psi(\gamma,
    \beta) | H_C | \psi(\gamma, \beta) \rangle$.

\end{itemize}

\vspace{0.3cm}

\noindent
\textbf{Algorithm Steps:}
\begin{enumerate}
  \item \textbf{Initialize}: For $n$ qubits, apply Hadamard gates: $\ket{s} =
    H^{\otimes n} \ket{0}^{\otimes n} = \frac{1}{\sqrt{2^n}} \sum_{x \in
    \{0,1\}^n} \ket{x}$. (Your note `$10 \mathrm{~N} \rightarrow \mathrm{H}$`
    suggests $n=10$.)

  \item \textbf{Apply Ansatz}: For $p=1$ to $P$, apply $e^{-i\gamma_p H_C}$
    (cost evolution) and $e^{-i\beta_p H_M}$ (mixing evolution).

  \item \textbf{Measure}: Compute $\langle H_C \rangle$ by measuring in the
    computational basis.

  \item \textbf{Optimize}: Use a classical algorithm (e.g., gradient descent)
    to adjust $\gamma_p$ and $\beta_p$.

  \item \textbf{Repeat}: Iterate until $\langle H_C \rangle$ converges to an
    approximate minimum.

\end{enumerate}

\subsubsection*{\textsc{Max-Cut} Example}

Consider a 4-node cycle graph with edges $\{(0,1), (1,2), (2,3), (3,0)\}$,
weights $w_{ij} = 1$. The Max-Cut problem seeks a partition maximizing cut
edges.

\vspace{0.3cm}

\noindent
\textbf{Cost Hamiltonian:}
\[
  H_C = Z_0 Z_1 + Z_1 Z_2 + Z_2 Z_3 + Z_3 Z_0
\]

Eigenstate $\ket{0101}$ (alternating partition) has eigenvalue $-4$ (4 cuts,
optimal).

\textbf{Mixing Hamiltonian:}
\[
  H_M = X_0 + X_1 + X_2 + X_3
\]

\textbf{Ansatz Circuit ($P=1$):}

\[
  \begin{quantikz}
    \lstick{$q_0: \ket{0}$} & \gate{H} & \gate{e^{-i\gamma Z_0 Z_1}} & \gate{e^{-i\gamma Z_3 Z_0}} & \gate{R_X(2\beta)} & \meter{} \\
    \lstick{$q_1: \ket{0}$} & \gate{H} & \gate[2]{Z_1 Z_2} & \qw & \gate{R_X(2\beta)} & \meter{} \\
    \lstick{$q_2: \ket{0}$} & \gate{H} & \qw & \gate[2]{Z_2 Z_3} & \gate{R_X(2\beta)} & \meter{} \\
    \lstick{$q_3: \ket{0}$} & \gate{H} & \qw & \qw & \gate{R_X(2\beta)} & \meter{}
  \end{quantikz}
\]

\begin{itemize}

  \item $e^{-i\gamma Z_i Z_j}$: Use CNOTs and $R_Z(2\gamma)$, since
    $e^{-i\theta Z_i Z_j} = \text{CNOT} \cdot (I \otimes R_Z(\theta)) \cdot
    \text{CNOT}$.

  \item $e^{-i\beta X_i}$: Direct $R_X(2\beta)$ gate.

\end{itemize}

\subsubsection*{Cirq Implementation}

Here’s a QAOA circuit for the 4-qubit Max-Cut with $P=1$:

\begin{minted}[linenos]{python}
import cirq
import numpy as np

# Qubits
qubits = cirq.LineQubit.range(4)

# Parameters (example)
gamma, beta = 0.5, 0.3

# Circuit
circuit = cirq.Circuit()
circuit.append(cirq.H.on_each(qubits))  # Superposition
# Cost terms
for i in range(4):
    circuit.append(cirq.ZZPowGate(exponent=2 * gamma/ np.pi).on(qubits[i], qubits[(i + 1) % 4]))
# Mixing terms
circuit.append(cirq.rx(2*beta).on_each(qubits))
# Measure
circuit.append(cirq.measure(*qubits, key='result'))

# Simulate
simulator = cirq.Simulator()
result = simulator.run(circuit, repetitions=8192)
counts = result.histogram(key='result')
print("Counts:", counts)
\end{minted}

\subsubsection*{Properties}

\textbf{Advantages:}
\begin{itemize}
  \item Works on NISQ devices with shallow circuits ($P$ small).
  \item $P \to \infty$ theoretically yields the exact solution (adiabatic limit).
\end{itemize}

\textbf{Limitations:}
\begin{itemize}
  \item Barren plateaus may trap optimization.
  \item Speedup not guaranteed; depends on problem and $P$.
\end{itemize}

\ex{Toy Example: 3-Qubit Triangle}{
  Graph: edges $(0,1), (1,2), (2,0)$, $H_C = Z_0 Z_1 + Z_1 Z_2 + Z_2 Z_0$.
  Max cut is 2 (e.g., $\ket{010}$). For $P=1$, QAOA with tuned $\gamma,
  \beta$ boosts $\ket{010}$ and $\ket{101}$ probabilities.
}

%%%%%%%%%%%%%%%%%%%%%%%%%%%%%%%%%%%%%%%%%%%%%%
% End of Lecture 13
%%%%%%%%%%%%%%%%%%%%%%%%%%%%%%%%%%%%%%%%%%%%%%
