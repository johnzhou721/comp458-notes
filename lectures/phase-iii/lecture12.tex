\section{Lecture 12: Introduction to Variational Quantum
Algorithms}\label{sec:lecture12}

Before we can begin discussions on Quantum Approximate Optimization
Algorithms (QAOA)\footnote{This class of algorithms are also interchangeably
  referred to as "Quantum Alternating Operator Ansatz," with "ansatz" meaning
"guess" in German for reasons that will become clear in the next lectures.},
we must first refine our conception of the classes of algorithms that are
affected by quantum computing. Additionally, we will get an introduction to a
broader class of algorithms called \emph{variational quantum algorithms
(VQA)}, of which QAOA is a subset.

\subsection*{Classes of Algorithms}

We can group algorithms into two classes: (1) those with proven quantum
speedup and (2) those with not fully proven quantum speedup.

\index{proven quantum speedup algorithms}
\subsubsection*{Proven Quantum Speedup Algorithms}

We have already seen one of these algorithms with Grover's Algorithm, which
has proven quadratic speedup. This means that for a search problem over $N$
items, Grover’s algorithm reduces the complexity from $O(N)$ in classical
computing to $O(\sqrt{N})$ in the quantum domain. Other examples include:

\begin{itemize}
  \index{proven quantum speedup algorithms!Shor's Algorithm}
  \item \textbf{Shor’s Algorithm}: Provides exponential speedup for integer
    factorization, reducing the complexity from $O(\exp((\log N)^{1/3}))$ to
    $O((\log N)^3)$.

    \index{proven quantum speedup algorithms!QFT}
  \item \textbf{Quantum Fourier Transform (QFT)}: Used in phase estimation,
    offering exponential speedup over classical FFT for certain applications.
\end{itemize}

These algorithms leverage quantum superposition, entanglement, and
interference to achieve rigorously proven advantages, often validated through
mathematical analysis of their unitary operators and measurement probabilities.

\index{heuristic quantum algorithms}
\subsubsection*{Heuristic Quantum Algorithms}

Heuristic quantum algorithms lack a fully proven speedup but show promise
based on empirical evidence or specific problem instances. These include:

\begin{itemize}
  \index{heuristic quantum algorithms!QAOA}
  \item \textbf{Quantum Approximate Optimization Algorithm (QAOA)}: A hybrid
    quantum-classical approach for combinatorial optimization, conjectured to
    offer speedup for certain graphs, though the exact advantage remains
    under investigation.

    \index{heuristic quantum algorithms!quantum machine learning}
  \item \textbf{Quantum Machine Learning (QML)}: Algorithms like quantum
    support vector machines or clustering may provide polynomial or
    context-specific speedups, often relying on variational techniques (see
    below).

    \index{heuristic quantum algorithms!quantum stimulation}
  \item \textbf{Quantum Simulation}: Simulating quantum systems (e.g.,
    molecular dynamics) is believed to be exponentially faster than classical
    methods, but the speedup depends on the Hamiltonian and is not
    universally proven.
\end{itemize}

These algorithms often use an \textbf{"ansatz"}—a parameterized guess for the
quantum state—adjusted iteratively, making their performance
problem-dependent and harder to analyze theoretically.

\index{ansatz}
\dfn{Ansatz}{
  A trial wavefunction or trial state used as a starting point for
  approximations or optimizations. It is a parameterized quantum state that
  serves as an educated guess for the solution to a particular problem, such
  as finding the ground state of a quantum system.
}

%%%%%%%%%%%

\index{Variational Quantum Algorithms}
\subsection*{Variational Quantum Algorithms (VQA)}

Variational Quantum Algorithms (VQA) are a class of hybrid quantum-classical
algorithms that optimize a cost function by varying circuit parameters to
achieve an objective. As noted in the lecture notes, they involve "varying
circuit parameters," typically through a parameterized quantum circuit (the
ansatz), followed by classical optimization of the parameters based on
measurement outcomes. Key features include:

\begin{itemize}
  \item \textbf{Hybrid Nature}: A quantum circuit prepares a state
    $\ket{\psi(\theta)}$ with parameters $\theta$, measured to compute a cost
    $C(\theta)$, which a classical optimizer (e.g., gradient descent) minimizes.

  \item \textbf{Applications}: VQAs are used for optimization (e.g., QAOA),
    quantum chemistry (e.g., Variational Quantum Eigensolver, VQE), and
    machine learning.

  \item \textbf{Flexibility}: The ansatz can be tailored to the problem,
    balancing expressiveness with hardware constraints like qubit count and
    gate depth.
\end{itemize}

The general process is depicted as:
\[
  \ket{\psi(\theta)} \xrightarrow{\text{varying circuit}} \text{Measure }
  C(\theta) \xrightarrow{\text{classical optimization}} \theta_{\text{opt}}
\]

\vspace{0.3cm}

\noindent
\textbf{Advantages:}
\begin{itemize}
  \item Robust to noise in near-term quantum devices (NISQ).

  \item Scales with available quantum resources, unlike exact algorithms
    requiring deep circuits.
\end{itemize}

\vspace{0.3cm}

\noindent
\textbf{Challenges:}
\begin{itemize}
  \item Barren plateaus in the optimization landscape can hinder convergence.

  \item Ansatz design impacts solution quality and computational cost.

\end{itemize}

\vspace{0.3cm}

\ex{Toy Example: Bell State Preparation}{
  A simple VQA example is preparing a Bell state, such as $\ket{\Phi^+} =
  \frac{\ket{00} + \ket{11}}{\sqrt{2}}$, using a parameterized circuit.
  Consider a circuit with a rotation gate $R_y(\theta)$ on one qubit followed
  by a CNOT:

  \textbf{Circuit Diagram:}
  \[
    \begin{quantikz}
      \lstick{$\ket{0}$} & \gate{R_y(\theta)} & \ctrl{1} & \qw \\
      \lstick{$\ket{0}$} & \qw & \targ{} & \qw
    \end{quantikz}
  \]

  \textbf{Objective}: Minimize the cost function $C(\theta) = 1 - |\langle
  \Phi^+ | \psi(\theta) \rangle|^2$, where $\ket{\psi(\theta)} = \text{CNOT}
  \cdot (R_y(\theta) \ket{0}) \otimes \ket{0}$. For $\theta = \pi/2$,
  $R_y(\pi/2)\ket{0} = \frac{\ket{0} + \ket{1}}{\sqrt{2}}$, and the circuit
  yields the Bell state exactly.

  \textbf{Truth Table (post-measurement probabilities):}
  \[
    \begin{array}{cc|c}
      q_0 & q_1 & P(q_0, q_1) \\
      \hline
      0 & 0 & 0.5 \\
      0 & 1 & 0 \\
      1 & 0 & 0 \\
      1 & 1 & 0.5 \\
    \end{array}
  \]

  This toy example illustrates parameter tuning to achieve a target state, a
  core concept in VQAs.
}

\vspace{0.3cm}

%%%%%%%%% Cirq code

\noindent
\textbf{Cirq Implementation}

Below is a Cirq implementation of the Bell state preparation:

\begin{minted}[linenos]{python}
import cirq
import numpy as np

# Define qubits
q0, q1 = cirq.LineQubit.range(2)

# Parameterized circuit
theta = np.pi / 2  # Optimal parameter
circuit = cirq.Circuit(
    cirq.ry(theta).on(q0),  # Rotation
    cirq.CNOT(q0, q1)       # Entanglement
)

# Simulate
simulator = cirq.Simulator()
result = simulator.simulate(circuit)
state = result.final_state_vector
print("Final state:", np.round(state, 3))

# Measure (8192 shots)
circuit.append(cirq.measure(q0, q1, key='result'))
samples = simulator.run(circuit, repetitions=8192)
counts = samples.histogram(key='result')
print("Counts:", counts)
\end{minted}

\textbf{Output Explanation:}
\begin{itemize}
  \item \textit{State Vector}: Approximately $[0.707, 0, 0, 0.707]$, matching
    $\ket{\Phi^+}$.

  \item \textit{Counts}: Roughly $\{0: 4096, 3: 4096\}$ (binary 00 and 11),
    confirming 50\% probability each.
\end{itemize}

This demonstrates a VQA’s ability to prepare a desired quantum state via
parameter optimization, foundational to more complex algorithms like QAOA.

%%%%%%%%%%%%%%%%%%%%%%%%%%%%%%%%%%%%%%%%%%%%%%
% End of Lecture 12
%%%%%%%%%%%%%%%%%%%%%%%%%%%%%%%%%%%%%%%%%%%%%%
