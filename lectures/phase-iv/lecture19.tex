\section{Lecture 19: More on Logical to Physical Mapping}\label{sec:lecture19}

We will continue to delve deeper into the process of mapping logical quantum
circuits to physical quantum hardware, focusing on IBM's heavy-hex lattice
topology. This mapping is a critical step in quantum computing due to the
constrained connectivity of physical qubits, often requiring the insertion of
SWAP gates to enable interactions between non-adjacent qubits. We will
explore how to decide on an effective mapping, and the consequences of
suboptimal choices on circuit efficiency. The IBM hardware operates with a
specific basis gate set—CX, X, RZ, and SX—into which all circuits must be
decomposed, adding another layer of complexity to the mapping process.

\subsection*{IBM's Heavy-Hex Lattice Topology}

IBM's quantum processors, such as those in the Falcon and Hummingbird
families, employ the heavy-hex lattice, a hexagonal arrangement of qubits
with additional edge qubits designed to minimize error rates. This topology
reduces frequency collisions and spectator errors compared to denser layouts
like square lattices, enhancing scalability and error correction
capabilities.\footnote{See \href{https://www.ibm.com/quantum/blog/heavy-hex-lattice}
{The IBM Quantum heavy hex lattice} for more details on the design motivation.}
For example, in a simplified seven-qubit heavy-hex layout, qubits might be
arranged as a chain (p0-p1-p2-p3-p4) with dangling qubits (p5 connected to
p1, p6 to p3), where each qubit has at most three neighbors.

\vspace{0.3cm}

\noindent
This connectivity can be represented conceptually as:

\begin{itemize}
  \item p0 -- p1 -- p2 -- p3 -- p4
  \item p5 -- p1
  \item p6 -- p3
\end{itemize}

\vspace{0.3cm}

\noindent
Such a structure balances connectivity with error mitigation, as explored in
research on topological codes.\footnote{Refer to \href{https://arxiv.org/abs/2404.15989}
{Creating entangled logical qubits in the heavy-hex lattice with topological
codes} for insights into error correction on this topology.}

\subsubsection*{Basis Gates and Hardware Constraints}

IBM's superconducting qubit platforms utilize a basis gate set consisting of
CX (CNOT), X, RZ, and SX gates. All logical circuits must be transpiled into
these gates for execution on IBM hardware.\footnote{Details on native gates
are available at \href{https://docs.quantum.ibm.com/guides/native-gates}{Native
gates and operations}.} This constraint, enforced by tools like Qiskit’s
transpiler, ensures compatibility but impacts circuit depth and fidelity. For
instance, a SWAP gate, necessary for non-adjacent qubit interactions,
decomposes into three CX gates, amplifying the importance of efficient mapping.

\ex{Mapping Logical Circuits to Physical Hardware}{

  Consider a simple logical circuit with a single CNOT gate between two qubits,
  q0 and q1:
  \[
    \begin{quantikz}
      \qw & \ctrl{1} & \qw \\
      \qw & \targ{} & \qw \\
    \end{quantikz}
  \]

  On a physical device with a linear topology (p0-p1-p2, where p0 is connected
  to p1, and p1 to p2), the mapping decision determines the circuit’s
  efficiency:

  \subsubsection*{Optimal Mapping}

  Map q0 to p0 and q1 to p1, where p0 and p1 are adjacent. The physical circuit
  remains:
  \[
    \begin{quantikz}
      \lstick{p0 (q0)} & \ctrl{1} & \qw \\
      \lstick{p1 (q1)} & \targ{} & \qw \\
      \lstick{p2} & \qw & \qw \\
    \end{quantikz}
  \]


  This requires no additional gates, preserving the circuit’s depth and
  minimizing error accumulation.

  \subsubsection*{Suboptimal Mapping}

  Map q0 to p0 and q1 to p2, where p0 and p2 are not directly connected. To
  execute the CNOT, we must swap q1 to an adjacent position:

  \begin{enumerate}
    \item SWAP p1 and p2 to move q1 to p1.
    \item Apply CNOT between p0 and p1.
    \item SWAP p1 and p2 back to restore the original mapping (if needed).
  \end{enumerate}

  The physical circuit becomes:
  \[
    \begin{quantikz}
      \lstick{p0 (q0)} & \qw & \qw & \ctrl{1} & \qw & \qw \\
      \lstick{p1} & \swap{1} & \qw & \targ{} & \swap{1} & \qw \\
      \lstick{p2 (q1)} & \swap{-1} & \qw & \qw & \swap{-1} & \qw \\
    \end{quantikz}
  \]

  Each SWAP decomposes into three CX gates, adding six CX gates total,
  significantly increasing depth and error potential.\footnote{For SWAP gate
    decomposition, see \href{https://learning.quantum.ibm.com/tutorial/explore-gates-and-circuits-with-the-quantum-composer}
  {Explore gates and circuits with IBM Quantum Composer}.}
}

\subsection*{Efficiency Impacts of Mapping Choices}

The choice of mapping directly affects circuit efficiency:

\begin{itemize}
  \item \textbf{Optimal Mapping:} No SWAPs, minimal gate count, and low error
    rate.
  \item \textbf{Suboptimal Mapping:} Two SWAPs, six additional CX gates,
    increased depth, and higher error probability due to more operations.
\end{itemize}

\vspace{0.3cm}

\noindent
For a larger circuit, such as one with three qubits (q0, q1, q2) and gates CX
q0 q1, CX q1 q2, CX q0 q2, mapping q0 to p0, q1 to p1, and q2 to p2 leverages
adjacency for the first two gates but requires a SWAP for the third (p0 to
p2). A better strategy might cluster frequently interacting qubits, reducing
SWAP overhead, as seen in optimization studies.\footnote{See
  \href{https://arxiv.org/html/2402.09705v1}{Linear Depth QFT over IBM
Heavy-hex Architecture} for mapping optimization techniques.}

\vspace{0.3cm}

\begin{center}
  \begin{table}[h]
    \centering
    \begin{tabular}{lcccc}
      \toprule
      Mapping Type & Number of SWAPs & Additional CNOTs & Circuit Depth Increase & Error Potential \\
      \midrule
      Optimal (Adjacent) & 0 & 0 & 0 & Low \\
      Suboptimal (Non-Adjacent) & 2 & 6 & Significant & High \\
      \bottomrule
    \end{tabular}
    \caption{Comparison of mapping efficiency for a two-qubit CNOT circuit.}
  \end{table}
\end{center}

\ex{Larger Circuit Mapping Example}{

  Consider a three-qubit logical circuit:
  \[
    \begin{quantikz}
      \lstick{q0} & \ctrl{1} & \qw & \ctrl{2} & \qw \\
      \lstick{q1} & \targ{} & \ctrl{1} & \qw & \qw \\
      \lstick{q2} & \qw & \targ{} & \targ{} & \qw \\
    \end{quantikz}
  \]


  On a heavy-hex segment (p0-p1-p2), an optimal mapping (q0 to p0, q1 to p1, q2
  to p2) executes the first two gates directly but requires a SWAP for CX q0 q2:
  \[
    \begin{quantikz}
      \lstick{p0 (q0)} & \ctrl{1} & \qw & \qw & \ctrl{1} & \qw \\
      \lstick{p1 (q1)} & \targ{} & \ctrl{1} & \swap{1} & \targ{} & \qw \\
      \lstick{p2 (q2)} & \qw & \targ{} & \swap{-1} & \targ{} & \qw \\
    \end{quantikz}
  \]

  Here, the SWAP p1 p2 moves q2 to p1, allowing the CNOT, adding three CX
  gates. A suboptimal mapping (e.g., q0 to p0, q1 to p2, q2 to p1) would
  require multiple SWAPs, worsening efficiency.
}

\subsection*{Practical Implications}

Efficient mapping is crucial for quantum algorithm performance, especially as
hardware scales to larger qubit counts (e.g., IBM’s Crossbill with 408
qubits).\footnote{See \href{https://www.ibm.com/quantum/blog/ibm-quantum-roadmap-2025}
{IBM roadmap to quantum-centric supercomputers} for future hardware plans.}
Suboptimal mappings can render circuits impractical due to excessive depth,
while optimal mappings, guided by tools like Qiskit, enhance fidelity.\footnote{
  For hardware-aware strategies, see \href{https://www.ibm.com/blogs/research/2020/09/hardware-aware-quantum/}
{Hardware-aware approach for fault-tolerant quantum computation}.}

%%%%%%%%%%%%%%%%%%%%%%%%%%%%%%%%%%%%%%%%%%%%%%
% End of Lecture 19
%%%%%%%%%%%%%%%%%%%%%%%%%%%%%%%%%%%%%%%%%%%%%%

