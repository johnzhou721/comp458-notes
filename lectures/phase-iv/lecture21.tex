\section{Lecture 21: Quantum Error Correction-- Phase Flip Errors} \label{sec:lecture21}

In the previous lecture, we explored bit flip error detection and correction
in quantum error correction, addressing errors that flip a qubit’s state from
\( |0\rangle \) to \( |1\rangle \) or vice versa. However, quantum systems
are also susceptible to another class of errors: phase flip errors, which
apply a \( Z \) gate, transforming \( |\psi\rangle = \alpha |0\rangle + \beta
|1\rangle \) into \( \alpha |0\rangle - \beta |1\rangle \). These errors
alter the relative phase between basis states, a critical issue in quantum
computing due to the importance of phase in superpositions and entanglement.
In this section, we will examine how to detect and correct phase flip errors,
highlighting their similarity to bit flip correction but with key differences
involving Hadamard and Z gates.

\index{error correction!phase flip errors}
\subsection*{Phase Flip Error Correction}

Phase flip errors are represented by the Pauli \( Z \) operator, which
applies a -1 phase to the \( |1\rangle \) component of a qubit’s state. For a
single qubit in \( |\psi\rangle = \alpha |0\rangle + \beta |1\rangle \), a
phase flip results in \( Z |\psi\rangle = \alpha |0\rangle - \beta |1\rangle
\). Unlike bit flip errors, which are detectable in the computational basis
(\( |0\rangle, |1\rangle \)), phase flips are more apparent in the Hadamard
basis (\( |+\rangle = \frac{|0\rangle + |1\rangle}{\sqrt{2}}, |-\rangle =
\frac{|0\rangle - |1\rangle}{\sqrt{2}} \)), where a \( Z \) gate flips \(
|+\rangle \) to \( |-\rangle \).

To correct phase flip errors, we can adapt the repetition code used for bit
flips by transforming the basis. The phase flip code encodes a logical qubit
into three physical qubits, similar to the bit flip code, but uses Hadamard
gates to switch to the \( |+\rangle, |-\rangle \) basis for error detection.
The logical states are:
\begin{itemize}
  \item Logical \( |0_L\rangle = |+++\rangle \)
  \item Logical \( |1_L\rangle = |---\rangle \)
\end{itemize}

\vspace{0.3cm}

For an arbitrary state \( |\psi\rangle = \alpha |0\rangle + \beta |1\rangle
\), the encoded state is \( \alpha |+++\rangle + \beta |---\rangle \). The
encoding circuit is:
\[
  \begin{quantikz}
    \lstick{$q_2 : |\psi\rangle$} & \gate{H} & \ctrl{1} & \ctrl{2} & \gate{H} & \qw \\
    \lstick{$q_1 : |0\rangle$} & \gate{H} & \targ{} & \qw & \gate{H} & \qw \\
    \lstick{$q_0 : |0\rangle$} & \gate{H} & \qw & \targ{} & \gate{H} & \qw \\
  \end{quantikz}
\]

This circuit:
\begin{enumerate}
  \item Applies Hadamard gates to all qubits, transforming \( q_2 :
    |\psi\rangle \to \alpha |+\rangle + \beta |-\rangle \), \( q_1, q_0 :
    |0\rangle \to |+\rangle \).
  \item Uses CNOTs to entangle the qubits, producing \( \alpha |+++\rangle +
    \beta |---\rangle \).
  \item Applies Hadamard gates again to return to the computational basis for
    further operations, though for error detection, we work in the Hadamard
    basis.
\end{enumerate}

To detect a phase flip (e.g., \( Z \) on \( q_1 \)), we measure syndromes
using ancilla qubits in the Hadamard basis. A \( Z \) error on \( q_1 \)
flips \( |+++\rangle \to |+-+\rangle \) or \( |---\rangle \to |-+-\rangle \).
The syndrome measurement circuit is:
\[
  \begin{quantikz}
    \lstick{$q_2$} & \gate{H} & \ctrl{3} & \qw & \qw & \gate{H} & \qw \\
    \lstick{$q_1$} & \gate{H} & \qw & \ctrl{3} & \ctrl{3} & \gate{H} & \qw \\
    \lstick{$q_0$} & \gate{H} & \qw & \qw & \qw & \gate{H} & \qw \\
    \lstick{$a_1 : |0\rangle$} & \qw & \targ{} & \qw & \qw & \meter{} & \qw \\
    \lstick{$a_0 : |0\rangle$} & \qw & \qw & \targ{} & \targ{} & \meter{} & \qw \\
  \end{quantikz}
\]

Here, ancillas measure parities \( X_0 X_1 \) and \( X_1 X_2 \):
\begin{itemize}
  \item \( X_0 X_1 \): CNOTs from \( q_0 \) and \( q_1 \) to \( a_0 \),
    measured as 1 if \( q_0 \neq q_1 \).
  \item \( X_1 X_2 \): CNOTs from \( q_1 \) and \( q_2 \) to \( a_1 \),
    measured as 1 if \( q_1 \neq q_2 \).
\end{itemize}

The syndrome outcomes are:

\begin{table}[H]
  \centering
  \begin{tabular}{cccl}
    \toprule
    \( a_1  (X_0 X_1) \) & \( a_0  (X_1 X_2) \) & Syndrome & Action \\
    \midrule
    0 & 0 & 00 & No error \\
    1 & 0 & 10 & Apply \( Z \) to \( q_0 \) \\
    1 & 1 & 11 & Apply \( Z \) to \( q_1 \) \\
    0 & 1 & 01 & Apply \( Z \) to \( q_2 \) \\
    \bottomrule
  \end{tabular}
  \caption{Syndrome measurements for phase flip error correction.}
\end{table}

For example, a \( Z \) error on \( q_1 \) yields syndrome 11, indicating \(
q_1 \) differs from both \( q_0 \) and \( q_2 \). Applying \( Z \) to \( q_1
\) corrects the error, restoring \( \alpha |+++\rangle + \beta |---\rangle
\).\footnote{See \href{https://www.ibm.com/quantum/blog/error-correction-codes}
{Error correcting codes for near-term quantum computers} for syndrome
measurement techniques.}

This approach mirrors bit flip correction but operates in the Hadamard basis,
leveraging the fact that a \( Z \) gate in the computational basis
corresponds to an \( X \) gate in the Hadamard basis (\( H Z H = X \)),
allowing us to repurpose bit flip correction circuits with basis changes.

\subsubsection*{Generalizing Phase Correction to an n-Qubit System}

Extending phase flip correction to an n-qubit system scales linearly in terms
of physical qubits required. For a code distance \( d = n \), we encode one
logical qubit into \( n \) physical qubits, producing logical states:

\begin{itemize}
  \item \( |0_L\rangle = |+\rangle^{\otimes n} \)
  \item \( |1_L\rangle = |-\rangle^{\otimes n} \)
\end{itemize}

\vspace{0.3cm}

\noindent
The encoding circuit generalizes to:

\begin{enumerate}
  \item Apply Hadamard gates to \( q_0 : |\psi\rangle \) and \( q_1, \dots,
    q_{n-1} : |0\rangle \).
  \item Apply CNOTs from \( q_0 \) to each \( q_i \) (for \( i = 1, \dots,
    n-1 \)).
  \item Optionally apply Hadamard gates to return to the computational basis.
\end{enumerate}

\vspace{0.3cm}

\noindent
For \( n = 5 \), the circuit is:
\[
  \begin{quantikz}
    \lstick{$q_4 : |\psi\rangle$} & \gate{H} & \ctrl{1} & \ctrl{2} & \ctrl{3} & \ctrl{4} & \gate{H} & \qw \\
    \lstick{$q_3 : |0\rangle$} & \gate{H} & \targ{} & \qw & \qw & \qw & \gate{H} & \qw \\
    \lstick{$q_2 : |0\rangle$} & \gate{H} & \qw & \targ{} & \qw & \qw & \gate{H} & \qw \\
    \lstick{$q_1 : |0\rangle$} & \gate{H} & \qw & \qw & \targ{} & \qw & \gate{H} & \qw \\
    \lstick{$q_0 : |0\rangle$} & \gate{H} & \qw & \qw & \qw & \targ{} & \gate{H} & \qw \\
  \end{quantikz}
\]

Syndrome measurements require \( n-1 \) ancillas to measure parities \( X_i
X_{i+1} \) for \( i = 0, \dots, n-2 \). Each ancilla detects differences
between adjacent qubits, identifying up to \( \lfloor (n-1)/2 \rfloor \)
errors. The number of physical qubits scales as \( O(n) \), and the circuit
depth for encoding is linear in \( n \), as each CNOT is sequential from \(
q_0 \). Error correction scales similarly, with \( n-1 \) syndrome
measurements.\footnote{For scaling details, see
  \href{https://arxiv.org/abs/1907.11157}{Quantum Error Correction: an
Introductory Guide}.}

This linear scaling makes phase flip correction feasible for larger systems,
though practical implementations must balance qubit count against error
rates, as more qubits increase the likelihood of multiple errors, requiring
higher-distance codes.

\ex{Phase Flip Correction for Bell State Preparation}{

  Consider a Bell state preparation circuit, creating \( |\Phi^+\rangle =
  \frac{|00\rangle + |11\rangle}{\sqrt{2}} \):
  \[
    \begin{quantikz}
      \lstick{$q_1 : |0\rangle$} & \gate{H} & \ctrl{1} & \qw \\
      \lstick{$q_0 : |0\rangle$} & \qw & \targ{} & \qw \\
    \end{quantikz}
  \]

  To protect against phase flips, we encode each qubit into a 3-qubit phase
  flip code. For \( q_1 \), we encode \( |\psi_0\rangle = \frac{|0\rangle +
  |1\rangle}{\sqrt{2}} \) (after Hadamard) into \( \frac{|+++\rangle +
  |---\rangle}{\sqrt{2}} \), and similarly for \( q_0 : |0\rangle \). The
  encoded circuit, adapted for phase flip protection, requires 6 qubits (3 per
  logical qubit):
  \[
    \begin{quantikz}
      \lstick{$q_{1,0} : |0\rangle$} & \gate{H} & \ctrl{1} & \ctrl{2} & \gate{H} & \ctrl{3} & \qw \\
      \lstick{$q_{1,1} : |0\rangle$} & \gate{H} & \targ{} & \qw & \gate{H} & \qw & \qw \\
      \lstick{$q_{1,2} : |0\rangle$} & \gate{H} & \qw & \targ{} & \gate{H} & \qw & \qw \\
      \lstick{$q_{0,0} : |0\rangle$} & \gate{H} & \ctrl{1} & \ctrl{2} & \gate{H} & \targ{} & \qw \\
      \lstick{$q_{0,1} : |0\rangle$} & \gate{H} & \targ{} & \qw & \gate{H} & \qw & \qw \\
      \lstick{$q_{0,2} : |0\rangle$} & \gate{H} & \qw & \targ{} & \gate{H} & \qw & \qw \\
    \end{quantikz}
  \]

  Here, \( q_{1,0}, q_{1,1}, q_{1,2} \) encode the first logical qubit, and \(
  q_{0,0}, q_{0,1}, q_{0,2} \) encode the second. The CNOT from \( q_{0,0} \)
  to \( q_{1,0} \) entangles the logical qubits, approximating the Bell state
  in the encoded space. If a phase flip occurs on \( q_{0,1} \), syndrome
  measurements (as above) on \( q_{0,0}, q_{0,1}, q_{0,2} \) detect it with
  syndrome 11, and a \( Z \) gate corrects it. This protects the entangled
  state, ensuring the Bell pair’s integrity.\footnote{For encoded circuit
    examples, see \href{https://www.nature.com/articles/s41567-023-02282-2}{Protecting
  expressive circuits with a quantum error detection code}.}
}

\aside{
  \textbf{FPGA: Field-Programmable Gate Array} \\
  Field-Programmable Gate Arrays (FPGAs) are integrated circuits that can be
  configured post-manufacturing to perform specific tasks, such as real-time
  error correction in quantum computers. In quantum systems, the decoding and
  encoding units for error correction are often implemented on FPGAs due to
  their proximity to the hardware. This allows immediate application of
  corrective operations, minimizing latency and reducing the risk of
  decoherence during error correction cycles. FPGAs enable dynamic adjustments
  to syndrome measurements and corrections, critical for maintaining quantum
  state integrity.\footnote{See
    \href{https://www.quantum-machines.co/blog-posts/quantum-error-correction}{Quantum
  Error Correction in Real-Time with FPGAs} for implementation details.}
}

\aside{
  \textbf{CAT Qubits} \\
  As of time of writing (2025-04-11), a company named Alice \& Bob is developing
  a novel type of superconducting qubit called a CAT qubit, named after
  Schr\"{o}dinger’s cat to reflect its coherent superposition of macroscopically
  distinct states. CAT qubits are engineered to exhibit biased noise,
  significantly increasing bit flip error rates while suppressing phase flip
  errors to near-negligible levels. This design reduces the overhead of error
  correction by focusing solely on bit flip correction, which is less
  resource-intensive. For example, their approach could lower the number of
  physical qubits needed for fault-tolerant computation, potentially
  accelerating practical quantum computing. However, the trade-off of higher
  bit flip errors requires robust bit flip correction protocols.\footnote{See
  \href{https://alice-bob.com/}{Alice \& Bob’s CAT Qubit Technology}
  for details.}
}


%%%%%%%%%%%%%%%%%%%%%%%%%%%%%%%%%%%%%%%%%%%%%%
% End of Lecture 21
%%%%%%%%%%%%%%%%%%%%%%%%%%%%%%%%%%%%%%%%%%%%%%

