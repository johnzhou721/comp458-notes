\section{Lecture 18: Quantum Circuits as DAGs, Qubit Topologies, Logical to
Physical Mapping}\label{sec:lecture18}

\subsubsection*{Representing Quantum Circuits as Directed Acyclic Graphs (DAGs)}

Quantum circuits are complex, but they can be visualized as Directed Acyclic
Graphs (DAGs), where each gate is a node, and directed edges show which gates
must happen before others. This helps in understanding the circuit's flow and
dependencies.

\index{directed acyclic graph}
\dfn{Directed Acyclic Graph (DAG)}{
  A graph with directed edges and no cycles, used to represent a quantum
  circuit where:
  \begin{itemize}
    \item Nodes represent quantum gates or operations
    \item Directed edges indicate dependencies between gates (which
      operations must be performed before others)
    \item Time flows in the direction of the edges
    \item The acyclic property ensures no operation can depend on future
      operations
    \item Qubit lines from circuit diagrams become vertices at different time
      steps
    \item Two-qubit gates create edges between the affected qubits,
      establishing execution dependencies
  \end{itemize}

  This representation provides a mathematical foundation for analyzing
  circuit complexity, parallelism, and execution constraints.
}

\vspace{0.3cm}

\noindent
The conversion process from a standard quantum circuit to a DAG follows these
principles:

\begin{enumerate}
  \item Each quantum gate operation (single-qubit gates like \(H, X, R_x,
    R_y, R_z\) or two-qubit gates like CNOT, CZ) becomes a node in the graph.

  \item Directed edges connect a gate to all subsequent gates that depend on it:

    \begin{itemize}
      \item Gates acting on the same qubit(s) later in the circuit

      \item Two-qubit gates create dependencies between their control and
        target qubits
    \end{itemize}

  \item The resulting graph is inherently acyclic due to the temporal
    sequential nature of quantum circuits, with operations flowing forward in
    time without feedback loops.

  \item Gates that can be executed in parallel (acting on independent qubits)
    will not have dependency paths between them.
\end{enumerate}

This representation makes it clear which operations can be performed
simultaneously and which ones must wait for prior operations to complete.

\index{critical path}
\dfn{Critical Path}{
  The longest path through the circuit's DAG representation from input to
  output, which determines:

  \begin{itemize}
    \item The minimum execution time of the quantum circuit

    \item The sequence of dependent operations that cannot be parallelized

    \item The circuit depth when optimal parallelization is applied
  \end{itemize}

  The critical path is particularly important for quantum computing because:

  \begin{itemize}
    \item Quantum states are highly susceptible to decoherence over time

    \item Minimizing the critical path reduces the total time qubits must
      maintain coherence

    \item The number of two-qubit gates in the critical path often dominates
      error rates

    \item It provides a key metric for comparing different implementations of
      the same algorithm
  \end{itemize}
}

\vspace{0.3cm}

From the DAG, finding the critical path is a well-studied graph problem that
can be solved efficiently using topological sorting and dynamic programming
to compute the longest path length.

\subsubsection*{Different Current Qubit Topology Designs}

Where early qubit topology designs would potentially have some qubits with
up to 4 qubit connections, this proved to produce noise effects that would
cause incorrect output. Modern designs as of writing make use of altered
qubit topology layouts that have at most 3 connections to other qubits.

An interesting aspect is the fundamental trade-off between connectivity and
noise: more connections between qubits can enable faster computations with
fewer SWAP operations, but each additional coupling point introduces
potential sources of crosstalk and errors. This trade-off has led modern
designs to favor more conservative connectivity patterns that prioritize gate
fidelity over maximum connectivity.

This process of converting the circuit design onto actual, physical quantum
hardware is appropriately called \textbf{logical to physical mapping}.
Furthermore, the process seen in the above example where each qubit's path is
manifested in the physical circuit (potentially with the use of SWAP gates
due to the physical connections of the hardware) is called \textbf{routing}.

\vspace{0.3cm}

\dfn{Logical $\rightarrow$ Physical Mapping}{
  The process of assigning logical qubits in a quantum algorithm to physical
  qubits on a quantum processor, considering:

  \begin{itemize}
    \item The hardware connectivity constraints of the physical device
    \item The specific interactions required by the quantum algorithm
    \item The quality and error rates of individual physical qubits
    \item The fidelity of two-qubit gate operations between specific physical
      qubits
  \end{itemize}

  Effective logical-to-physical mapping minimizes the number of additional
  operations (like SWAP gates) needed to execute the circuit on hardware with
  limited connectivity, which is crucial for reducing noise exposure and
  improving overall circuit performance.
}

\dfn{Routing}{
  The process of determining how to move quantum information between
  non-adjacent qubits on a quantum processor with limited connectivity,
  including:

  \begin{itemize}
    \item Inserting SWAP operations to move quantum states between physical
      qubits
    \item Determining the optimal sequence of SWAP operations to enable
      two-qubit gates between logical qubits mapped to non-adjacent physical
      qubits
    \item Balancing the trade-off between circuit depth increase and total
      gate count
    \item Considering the relative error rates of different physical
      connections when deciding routing paths
  \end{itemize}

  Efficient routing is critical for quantum circuit execution on NISQ (Noisy
  Intermediate-Scale Quantum) devices, as poor routing decisions can
  significantly degrade algorithm performance through accumulated errors.
}

\vspace{0.3cm}

As quantum devices continue to scale up, the complexity of the mapping and
routing problems grows significantly, making efficient compilation techniques
an essential area of research for practical quantum computing.

%%%%%%%%%%%%%%%%%%%%%%%%%%%%%%%%%%%%%%%%%%%%%%
% End of Lecture 18
%%%%%%%%%%%%%%%%%%%%%%%%%%%%%%%%%%%%%%%%%%%%%%

