\section{Lecture 2: Review of Linear Algebra Concepts}\label{sec:lecture2}

Linear algebra provides the foundation for manipulating quantum states, which
are represented using vectors and matrices in a complex vector space.

\dfn{Vectors: Row and Column Vectors}{A \textbf{vector} \index{vector} is an
  ordered list of numbers, which can be represented as either a row or column
  vector. The components of vectors in quantum computing belong to the field of
complex numbers ($\mathbb{C}$).}

\subsection*{Column Vectors} \index{vector!column| see{ ket }}

A column vector is a vertical arrangement of numbers:

\[
  \mathbf{v} =
  \begin{bmatrix}
    v_1 \\
    v_2 \\
    \vdots \\
    v_n
  \end{bmatrix}, \quad v_i \in \mathbb{C}.
\]

\subsection*{Row Vectors}

A row vector is the complex conjugate transpose \textbf{vector}
\index{vector!adjoint} of a column vector:

\[
  \mathbf{v}^\dagger =
  \begin{bmatrix}
    \overline{v_1} & \overline{v_2} & \dots & \overline{v_n}
  \end{bmatrix}.
\]

The adjoint of a column vector is a row vector, and vice versa. We represent
the adjoint of a vector using the dagger symbol ($\dagger$).
\index{vector!dagger@\textsl{dagger}| see{ adjoint }}

\subsection*{Dirac Notation \index{vector!Dirac notation}}

In quantum computing, vectors are represented using \textbf{Dirac notation}
(bra-ket notation):

\begin{itemize}
  \item \textbf{Ket} \index{vector!Dirac notation!ket@\textsl{ket}} \(
    |v\rangle \): Represents a column vector.

  \item \textbf{Bra} \index{vector!Dirac notation!bra@\textsl{bra}} \(
    \langle v | \): Represents the adjoint (conjugate transpose) of the ket.

  \item Example: \( |v\rangle = \begin{bmatrix} 1 + i \\ 2 \end{bmatrix},
    \quad \langle v | = \begin{bmatrix} 1 - i & 2 \end{bmatrix} \).
\end{itemize}

\dfn{Euler's Formula}{Euler's formula \index{Euler's formula} relates complex
  exponentials to trigonometric functions:

  \[
    e^{i\omega} = \cos(\omega) + i\sin(\omega)
  \]

This is fundamental in representing quantum states and transformations.}

\dfn{Inner Product}{The \textbf{inner product} \index{vector!inner product}
  of two vectors $\mathbf{v}, \mathbf{w} \in \mathbb{C}^n$ is defined as:

  \[
    \langle \mathbf{v}, \mathbf{w} \rangle = \mathbf{v}^\dagger \mathbf{w} =
    \sum_{i=1}^n \overline{v_i}w_i
  \]

  which measures the overlap between two quantum states.

  \ex{Inner Product Example}{

    Given two vectors:

    \[
      \mathbf{v} = \begin{bmatrix} 1 \\ i \end{bmatrix}, \quad
      \mathbf{w} = \begin{bmatrix} 2 \\ 1 \end{bmatrix}
    \]

    The inner product is:

    \[
      \langle \mathbf{v}, \mathbf{w} \rangle = \begin{bmatrix} 1 & -i
      \end{bmatrix}
      \begin{bmatrix} 2 \\ 1 \end{bmatrix} = 2 - i
    \]
  }

  We also have the following property that the inner product of a vector
  and itself is equivalent to the square of the Euclidean norm of a vector:
  \index{vector!Euclidean norm}

  \[
    \langle \mathbf{v}, \mathbf{v} \rangle = \|\mathbf{v}\|^2
  \]
}

\dfn{Outer Product}{The \textbf{outer product} \index{vector!outer product}
  of two vectors $\mathbf{v} \in \mathbb{C}^m$ and $\mathbf{w} \in
  \mathbb{C}^n$ produces an $m \times n$ matrix:

  \[
    \mathbf{v}\mathbf{w}^\dagger =
    \begin{bmatrix} v_1 \\ v_2 \\ \vdots \\ v_m \end{bmatrix}
    \begin{bmatrix} \overline{w_1} & \overline{w_2} & \dots & \overline{w_n}
    \end{bmatrix}
  \]

This operation is useful for constructing quantum operators.}

\dfn{Tensor Product}{The \textbf{tensor product} (or Kronecker product)
  allows us to describe multi-qubit systems. Given two vectors:
  \index{vector!tensor product}

  \[
    \mathbf{v} = \begin{bmatrix} v_1 \\ v_2 \end{bmatrix}, \quad
    \mathbf{w} = \begin{bmatrix} w_1 \\ w_2 \end{bmatrix}
  \]

  Their tensor product is:

  \[
    \mathbf{v} \otimes \mathbf{w} =
    \begin{bmatrix}
      v_1 w_1 \\
      v_1 w_2 \\
      v_2 w_1 \\
      v_2 w_2
    \end{bmatrix}
  \]

  The tensor product expands the state space, allowing representation of
entangled states.}

\subsection*{Orthagonality} \index{vector!orthogonality}
Two vectors $v, w \in \mathbb{C}^n$ are \textbf{orthogonal} if their
inner product is zero:

\[
  \langle \mathbf{v}, \mathbf{w} \rangle = 0
\]

Orthogonal vectors are linearly independent and span a subspace of the vector
space. As you might remember from linear algebra, a set of orthogonal vectors
can be used to construct an orthonormal basis, and any vector can be expressed
as a linear combination of the basis vectors.

\vspace{0.3cm}

This will be useful when we cover the quantum bases in \autoref{sec:lecture3}.

\vspace{0.3cm}

\dfn{Adjoint of a Matrix}{The \textbf{adjoint} (or Hermitian conjugate) of a
  matrix $A$ is obtained by taking the transpose and complex conjugate of
  each entry:
  \[
    A^\dagger = \overline{A^T}
  \]
  If $A$ is:
  \[
    A = \begin{bmatrix}
      1 & i \\
      2 & 3
    \end{bmatrix}
  \]
  Then its adjoint is:
  \[
    A^\dagger =
    \begin{bmatrix}
      1 & 2 \\
      - i & 3
    \end{bmatrix}
\]}

\dfn{Unitary Matrix}{A square matrix $U$ is called \textbf{unitary}
  \index{matrix!unitary} if its adjoint is equal to its
  inverse:

  \[
    U^\dagger U = I
  \]

  where $I$ is the identity matrix. Unitary matrices preserve the norm of
  quantum states and represent reversible quantum operations. Example:

  \[
    U = \frac{1}{\sqrt{2}}
    \begin{bmatrix}
      1 & 1 \\
      1 & -1
    \end{bmatrix}, \quad U^\dagger U = I
\]}

\dfn{Hermitian Matrix}{A square matrix $H$ is called \textbf{Hermitian}
  \index{matrix!Hermitian} if it is equal to its adjoint:

  \[
    H = H^\dagger
  \]

  Hermitian matrices represent observable quantities in quantum mechanics and
  have real eigenvalues. Example:

  \[
    H = \begin{bmatrix}
      2 & i \\
      - i & 2
    \end{bmatrix}
  \]

  Since $H^\dagger = H$, it is Hermitian.

  \nt{Hermitian matrices \textbf{can't} have complex numbers in their diagonal
    General case illustration:

    \[
      M = \begin{bmatrix} a + ib & c + id \\ e + if & g  + ih \end{bmatrix}
      \quad \Rightarrow \quad
      M^\dagger = \begin{bmatrix} a - ib & e - if \\ c - id & g - ih
      \end{bmatrix}
      \quad \Rightarrow \quad M \neq M^\dagger
    \]

    \vspace{0.3cm}

    $\therefore$ Hermitian matrices have real diagonal elements.

    \vspace{0.3cm}

    Additionally, the general matrix $M$ shown above is Hermitian iff. $c =
    e, d = -f$
  }
}

Hermitian matrices are not necessarily unitary, and unitary matrices are not
necessarily Hermitian. A matrix can be both, but these properties are distinct:

\[
  H = H^\dagger \quad (\text{Hermitian}), \quad U U^\dagger = I \quad (\text{unitary})
\]

In quantum computing, gates that are both Hermitian and unitary (e.g., Pauli
gates) satisfy $U = U^\dagger$ and $U^2 = I$, making them their own
inverses---they cancel out when applied twice in succession. However, not all
unitary gates are Hermitian (e.g., the phase gate, which is unitary but not
Hermitian):

\[
  \text{Pauli } X = \begin{bmatrix} 0 & 1 \\ 1 & 0 \end{bmatrix}, \quad X = X^\dagger, \quad X^2 = I; \quad
  \text{Phase } S = \begin{bmatrix} 1 & 0 \\ 0 & i \end{bmatrix}, \quad S S^\dagger = I, \quad S \neq S^\dagger
\]

\vspace{0.3cm}
%%%%%%%%%%%%%%

\dfn{Eigenvalues and Eigenvectors}{
  For a square matrix $A \in \mathbb{C}^{n\times n}$, a vector
  $\mathbf{v} \neq \mathbf{0}$ is an \textbf{eigenvector} if:
  \index{vector!eigenvector}

  \[
    A\mathbf{v} = \lambda\mathbf{v}
  \]

  where $\lambda \in \mathbb{C}$ is the \textbf{eigenvalue}. Eigenvalues
  provide insight into the structure of linear transformations.
  \index{matrix!eigenvalue}

  In Braket notation, the eigenvalue equation is: \index{matrix!eigenvalue equation}

  \[
    A|\mathbf{v}\rangle = \lambda|\mathbf{v}\rangle
  \]
}

\ex{Example: Eigenvalues}{For the matrix

  \[
    A = \begin{bmatrix} 1 & i \\ -i & 1 \end{bmatrix}
  \]

  The characteristic equation is:

  \[
    \det(A - \lambda I) = (1 - \lambda)^2 + 1 = 0
  \]

Solving gives eigenvalues $\lambda = 1 \pm i$.}

\dfn{Quantum Bits/ Qubits}{
  A \textbf{qubit} \index{qubit} can be defined mathematically as follows:

  \[
    \ket{\psi} = \bmatrix \alpha_1 \\ \alpha_2 \endbmatrix \in \mathbb{C}^2
  \]

  where:

  \[
    \alpha_1, \alpha_2 \in \mathbb{C} \quad \text{and} \quad |\alpha_1|^2 +
    |\alpha_2|^2 = 1
  \]
}

The first property ensures that the qubit is normalized, while the second
property ensures that the qubit is in a superposition of the basis states.

\vspace{0.3cm}

The first universal basis that we will look at is the computational basis,
which consists of the states \(\zero\) and \(\one\):
\index{universal bases!computational}

\[
  \text{Zero state} = \zero = \bmatrix 1 \\ 0 \endbmatrix
  \quad \text{One state} = \one = \bmatrix 0 \\ 1 \endbmatrix
\]

A quantum state vector $\ket{\psi}$ can be expressed as a linear
combination of the basis states:

\[
  \ket{\psi} = \alpha_1\zero + \alpha_2\one
\]

\nt{Properties of the computational basis:
  \index{universal bases!computational!properties@\textit{properties}}

  \begin{itemize}
    \item \textbf{The computational basis states are orthogonal:}

      \[
        \langle 0 | 1 \rangle = \zero^\dagger \one = \bmatrix 1 & 0 \endbmatrix
        \bmatrix 0 \\ 1 \endbmatrix = 0
      \]

    \item \textbf{The computational basis states are normalized:}

      \[
        \langle 0 | 0 \rangle = \zero^\dagger \zero = \bmatrix 1 & 0
        \endbmatrix \bmatrix 1 \\ 0 \endbmatrix = 1
      \]
  \end{itemize}
}

  %%%%%%%%%%%%%%%%%%%%%%%%%%%%%5

\qs{}{Show that any unitary matrix preserves the inner product of two
vectors.}

\sol{Since a unitary matrix satisfies \( U^\dagger U = I \), we have:
  \[
    \langle U\mathbf{v}, U\mathbf{w} \rangle = \mathbf{v}^\dagger
    (U^\dagger U) \mathbf{w} = \mathbf{v}^\dagger \mathbf{w}
  \]
Thus, inner products are preserved.}


%%%%%%%%%%%%%%%%%%%%%%%%%%%%%%%%%%%%%%%%%%%%%%
% End of Lecture 2
%%%%%%%%%%%%%%%%%%%%%%%%%%%%%%%%%%%%%%%%%%%%%%
