\section{Lecture 2: Review of Linear Algebra Concepts}

Linear algebra provides the foundation for manipulating quantum states, which are represented using vectors and matrices in a complex vector space.

\dfn{Vectors: Row and Column Vectors}{A \textbf{vector} is an ordered list of numbers, which can be represented as either a row or column vector. The components of vectors in quantum computing belong to the field of complex numbers ($\mathbb{C}$).}

\subsection*{Column Vectors}
A column vector is a vertical arrangement of numbers:
\[
\mathbf{v} =
\begin{bmatrix}
v_1 \\
v_2 \\
\vdots \\
v_n
\end{bmatrix}, \quad v_i \in \mathbb{C}.
\]

\subsection*{Row Vectors}
A row vector is the complex conjugate transpose (adjoint) of a column vector:
\[
\mathbf{v}^\dagger =
\begin{bmatrix}
\overline{v_1} & \overline{v_2} & \dots & \overline{v_n}
\end{bmatrix}.
\]

\subsection*{Dirac Notation}
In quantum computing, vectors are represented using \textbf{Dirac notation} (bra-ket notation):
\begin{itemize}
    \item \textbf{Ket} \( |v\rangle \): Represents a column vector.
    \item \textbf{Bra} \( \langle v | \): Represents the adjoint (conjugate transpose) of the ket.
    \item Example: \( |v\rangle = \begin{bmatrix} 1 + i \\ 2 \end{bmatrix}, \quad \langle v | = \begin{bmatrix} 1 - i & 2 \end{bmatrix} \).
\end{itemize}

\dfn{Euler's Formula}{Euler's formula relates complex exponentials to trigonometric functions:
\[
e^{i\omega} = \cos(\omega) + i\sin(\omega)
\]
This is fundamental in representing quantum states and transformations.}

\dfn{Inner Product}{The \textbf{inner product} of two vectors $\mathbf{v}, \mathbf{w} \in \mathbb{C}^n$ is defined as:
\[
\langle \mathbf{v}, \mathbf{w} \rangle = \mathbf{v}^\dagger \mathbf{w} = \sum_{i=1}^n \overline{v_i}w_i
\]
which measures the overlap between two quantum states.}

\dfn{Outer Product}{The \textbf{outer product} of two vectors $\mathbf{v} \in \mathbb{C}^m$ and $\mathbf{w} \in \mathbb{C}^n$ produces an $m \times n$ matrix:
\[
\mathbf{v}\mathbf{w}^\dagger =
\begin{bmatrix} v_1 \\ v_2 \\ \vdots \\ v_m \end{bmatrix}
\begin{bmatrix} \overline{w_1} & \overline{w_2} & \dots & \overline{w_n} \end{bmatrix}
\]
This operation is useful for constructing quantum operators.}

\dfn{Tensor Product}{The \textbf{tensor product} (or Kronecker product) allows us to describe multi-qubit systems. Given two vectors:
\[
\mathbf{v} = \begin{bmatrix} v_1 \\ v_2 \end{bmatrix}, \quad
\mathbf{w} = \begin{bmatrix} w_1 \\ w_2 \end{bmatrix}
\]
Their tensor product is:
\[
\mathbf{v} \otimes \mathbf{w} =
\begin{bmatrix}
v_1 w_1 \\
v_1 w_2 \\
v_2 w_1 \\
v_2 w_2
\end{bmatrix}
\]
The tensor product expands the state space, allowing representation of entangled states.}

\dfn{Adjoint of a Matrix}{The \textbf{adjoint} (or Hermitian conjugate) of a matrix $A$ is obtained by taking the transpose and complex conjugate of each entry:
\[
A^\dagger = \overline{A^T}
\]
If $A$ is:
\[
A = \begin{bmatrix}
1 & i \\
2 & 3
\end{bmatrix}
\]
Then its adjoint is:
\[
A^\dagger =
\begin{bmatrix}
1 & 2 \\
- i & 3
\end{bmatrix}
\]}

\dfn{Unitary Matrix}{A square matrix $U$ is called \textbf{unitary} if its adjoint is equal to its inverse:
\[
U^\dagger U = I
\]
where $I$ is the identity matrix. Unitary matrices preserve the norm of quantum states and represent reversible quantum operations. Example:
\[
U = \frac{1}{\sqrt{2}}
\begin{bmatrix}
1 & 1 \\
1 & -1
\end{bmatrix}, \quad U^\dagger U = I
\]}

\dfn{Hermitian Matrix}{A square matrix $H$ is called \textbf{Hermitian} if it is equal to its adjoint:
\[
H = H^\dagger
\]
Hermitian matrices represent observable quantities in quantum mechanics and have real eigenvalues. Example:
\[
H = \begin{bmatrix}
2 & i \\
- i & 2
\end{bmatrix}
\]
Since $H^\dagger = H$, it is Hermitian.}

\dfn{Eigenvalues and Eigenvectors}{For a square matrix $A \in \mathbb{C}^{n \times n}$, a vector $\mathbf{v} \neq \mathbf{0}$ is an \textbf{eigenvector} if:
\[
A\mathbf{v} = \lambda\mathbf{v}
\]
where $\lambda \in \mathbb{C}$ is the \textbf{eigenvalue}. Eigenvalues provide insight into the structure of linear transformations.}

\ex{Example: Eigenvalues}{For the matrix
\[
A = \begin{bmatrix} 1 & i \\ -i & 1 \end{bmatrix}
\]
The characteristic equation is:
\[
\det(A - \lambda I) = (1 - \lambda)^2 + 1 = 0
\]
Solving gives eigenvalues $\lambda = 1 \pm i$.}

\qs{}{Show that any unitary matrix preserves the inner product of two vectors.}

\sol{Since a unitary matrix satisfies \( U^\dagger U = I \), we have:
\[
\langle U\mathbf{v}, U\mathbf{w} \rangle = \mathbf{v}^\dagger (U^\dagger U) \mathbf{w} = \mathbf{v}^\dagger \mathbf{w}
\]
Thus, inner products are preserved.}

