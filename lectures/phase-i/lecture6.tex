\section{Lecture 6: Multi-Qubit Gates and Circuit Construction}
\label{sec:lecture6}

$n$-qubit gates are unitary transformations that operate on $n$ qubits. This
section explores key multi-qubit gates, their properties, and how to
construct quantum circuits using these gates.

\index{quantum gates!multi-qubit gates}
\subsection*{Controlled Gates}

\index{quantum gates!multi-qubit gates!controlled gates}
\dfn{Controlled X Gate}{(CNOT) is a two-qubit gate that flips the target qubit
  if the control qubit is in state $\ket{1}$, and does nothing if the control
  qubit is in state $\ket{0}$. The matrix representation of CNOT is given by:

  \[
    CNOT = CX = \begin{pmatrix}
      1 & 0 & 0 & 0 \\
      0 & 1 & 0 & 0 \\
      0 & 0 & 0 & 1 \\
      0 & 0 & 1 & 0
    \end{pmatrix}
  \]

  Representing the gate in a circuit diagram:

  \begin{align*}
    \begin{quantikz}
      \lstick{$\ket{c}$} & \ctrl{1} & \qw \\
      \lstick{$\ket{t}$} & \gate{X} & \qw \\
    \end{quantikz}
  \end{align*}

  The control qubit is denoted by $\ket{c}$ and the target qubit is denoted by
$\ket{t}$.}

\vspace{0.3cm}

Applying the CNOT gate to a two-qubit state $\ket{\psi} = \alpha\ket{00} +
\beta\ket{01} + \gamma\ket{10} + \delta\ket{11}$:

\begin{equation}
  \begin{aligned}
      % CX |00>
    CX \ket{00} & = \begin{bmatrix}
      1 & 0 & 0 & 0 \\
      0 & 1 & 0 & 0 \\
      0 & 0 & 0 & 1 \\
      0 & 0 & 1 & 0
      \end{bmatrix} \begin{bmatrix}
      1 \\ 0 \\ 0 \\ 0
      \end{bmatrix} & = \begin{bmatrix}
      1 \\ 0 \\ 0 \\ 0
    \end{bmatrix} = \ket{00} \\
      % CX |01>
    CX \ket{01} & = \begin{bmatrix}
      1 & 0 & 0 & 0 \\
      0 & 1 & 0 & 0 \\
      0 & 0 & 0 & 1 \\
      0 & 0 & 1 & 0
      \end{bmatrix} \begin{bmatrix}
      0 \\ 1 \\ 0 \\ 0
      \end{bmatrix} & = \begin{bmatrix}
      0 \\ 1 \\ 0 \\ 0
    \end{bmatrix} = \ket{01} \\
      % CX |10>
    CX \ket{10} & = \begin{bmatrix}
      1 & 0 & 0 & 0 \\
      0 & 1 & 0 & 0 \\
      0 & 0 & 0 & 1 \\
      0 & 0 & 1 & 0
      \end{bmatrix} \begin{bmatrix}
      0 \\ 0 \\ 1 \\ 0
      \end{bmatrix} & = \begin{bmatrix}
      0 \\ 0 \\ 0 \\ 1
    \end{bmatrix} = \ket{11} \\
      % CX |11>
    CX \ket{11} & = \begin{bmatrix}
      1 & 0 & 0 & 0 \\
      0 & 1 & 0 & 0 \\
      0 & 0 & 0 & 1 \\
      0 & 0 & 1 & 0
      \end{bmatrix} \begin{bmatrix}
      0 \\ 0 \\ 0 \\ 1
      \end{bmatrix} & = \begin{bmatrix}
      0 \\ 0 \\ 1 \\ 0
    \end{bmatrix} = \ket{10}
  \end{aligned}
\end{equation}

\subsection*{Swapping Control and Target}

The control and target qubits of a CNOT gate can be swapped using Hadamard
gates:

\[
  (H \otimes H) \cdot CNOT_{\text{control,target}} \cdot (H \otimes H) =
  CNOT_{\text{target,control}}
\]

This transformation can be visualized in a circuit diagram:

\begin{align*}
  \begin{quantikz}
    \lstick{$\ket{c}$} & \gate{H} & \ctrl{1} & \gate{H} & \qw \\
    \lstick{$\ket{t}$} & \gate{H} & \gate{X} & \gate{H} & \qw \\
  \end{quantikz}
\end{align*}


% TODO: move this to lecture 7
\index{quantum gates!multi-qubit gates!SWAP gate}
\subsection*{SWAP Gate}

\dfn{SWAP Gate}{exchanges the states of two qubits. Its matrix representation
  is:

  \[
    SWAP = \begin{pmatrix}
      1 & 0 & 0 & 0 \\
      0 & 0 & 1 & 0 \\
      0 & 1 & 0 & 0 \\
      0 & 0 & 0 & 1
    \end{pmatrix}
  \]

  The action of SWAP on basis states is given by:

  \[
    SWAP\ket{ab} = \ket{ba}, \quad a,b \in \{0,1\}
  \]
}

\index{quantum gates!multi-qubit gates!Controlled-Z gate}
\subsection*{Controlled-Z Gate}

\dfn{Controlled-Z Gate}{(CZ) applies a phase flip if both qubits are in state
  $\ket{1}$:

  \[
    CZ = \begin{pmatrix}
      1 & 0 & 0 & 0 \\
      0 & 1 & 0 & 0 \\
      0 & 0 & 1 & 0 \\
      0 & 0 & 0 & -1
    \end{pmatrix}
  \]
}

% TODO: move this to lecture 7
\index{quantum gates!multi-qubit gates!Toffoli gate}
\subsection*{Toffoli Gate}

\dfn{Toffoli Gate}{(CCX) is a three-qubit gate with two control qubits and
  one target qubit. Its matrix representation is:

  \[
    Toffoli = \begin{pmatrix}
      1 & 0 & 0 & 0 & 0 & 0 & 0 & 0 \\
      0 & 1 & 0 & 0 & 0 & 0 & 0 & 0 \\
      0 & 0 & 1 & 0 & 0 & 0 & 0 & 0 \\
      0 & 0 & 0 & 1 & 0 & 0 & 0 & 0 \\
      0 & 0 & 0 & 0 & 1 & 0 & 0 & 0 \\
      0 & 0 & 0 & 0 & 0 & 1 & 0 & 0 \\
      0 & 0 & 0 & 0 & 0 & 0 & 0 & 1 \\
      0 & 0 & 0 & 0 & 0 & 0 & 1 & 0
    \end{pmatrix}
  \]

  The Toffoli gate is Hermitian and only flips the target qubit if both control
  qubits are in state $\ket{1}$.
}

\index{Circuit notation!multi-qubit gates}
\subsection*{Circuit Representations}

The standard circuit representations for these multi-qubit gates are:

\begin{align*}
  \begin{quantikz}
    % CNOT
    \lstick{CNOT:} & \ctrl{1} & \qw \\
    & \targ{} & \qw \\[0.5cm]
    % SWAP
    \lstick{SWAP:} & \swap{1} & \qw \\
    & \targX{} & \qw \\[0.5cm]
    % CZ
    \lstick{CZ:} & \ctrl{1} & \qw \\
    & \gate{Z} & \qw \\[0.5cm]
    % Toffoli
    \lstick{Toffoli:} & \ctrl{1} & \qw \\
    & \ctrl{1} & \qw \\
    & \targ{} & \qw
  \end{quantikz}
\end{align*}

\subsection*{Tensor Product Ordering and Circuit Representations}

When converting quantum circuits to mathematical expressions, it's important
to understand that tensor products are not commutative: $A \otimes B \neq B
\otimes A$ in general. However, we can modify the ordering of tensor products
in our mathematical representation as long as we maintain the dependencies
established by the circuit diagram. This leads to multiple valid mathematical
representations of the same quantum circuit.

\dfn{Bit Ordering Convention}{Due to the non-commutativity of tensor
  products, we adopt the convention of representing qubits from most
  significant to least significant in our mathematical expressions. For an
  $n$-qubit system:

  \[
  \ket{q_{n-1}q_{n-2}...q_1q_0} = \ket{q_{n-1}} \otimes \ket{q_{n-2}} \otimes
  ... \otimes \ket{q_1} \otimes \ket{q_0}
  \]

  where $q_0$ is the least significant qubit.
}

\noindent

For example, consider a circuit with two Hadamard gates applied to different
qubits:

\begin{align*}
  \begin{quantikz}
    \lstick{$\ket{q_1}$} & \gate{H} & \qw \\
    \lstick{$\ket{q_0}$} & \gate{H} & \qw
  \end{quantikz}
\end{align*}

This circuit can be represented mathematically in equivalent ways:

\[
  (H \otimes H)\ket{q_1q_0} = (H\ket{q_1}) \otimes (H\ket{q_0}) = H\ket{q_1}
  \otimes H\ket{q_0}
\]

\noindent
When gates have dependencies (like controlled operations), the ordering must
respect these dependencies. For the CNOT gate:

\begin{align*}
  \begin{quantikz}
    \lstick{$\ket{q_1}$} & \ctrl{1} & \qw \\
    \lstick{$\ket{q_0}$} & \gate{X} & \qw
  \end{quantikz}
\end{align*}

\noindent
The mathematical representation must preserve the control-target
relationship, though intermediate calculations may use different but
equivalent orderings:

\[
  CNOT_{1,0}\ket{q_1q_0} = CNOT(\ket{q_1} \otimes \ket{q_0})
\]

This flexibility in representation, while maintaining functional equivalence,
is particularly useful when analyzing complex quantum circuits or optimizing
quantum computations.
