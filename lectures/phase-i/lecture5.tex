\section{Lecture 5: Other Quantum Gates, Measurement, Multi-Qubit Systems}

\dfn{Single-Qubit Gates}{Quantum gates manipulate individual qubits. Key
single-qubit gates include:}
\begin{itemize}
    \item \textbf{Hadamard Gate (H):}
    \[
        H = \frac{1}{\sqrt{2}} \begin{pmatrix} 1 & 1 \\ 1 & -1 \end{pmatrix}
    \]
    Creates superposition: \boxed{H\zero = \ket{+}, H\one = \ket{-}}

    Properties:
    \begin{itemize}
        \item Self-inverse: $H^2 = I$
        \item Maps computational basis to $\ket{\pm}$ basis:
        \begin{align*}
            \ket{+} &= \frac{1}{\sqrt{2}}(\zero + \one) \\
            \ket{-} &= \frac{1}{\sqrt{2}}(\zero - \one)
        \end{align*}
    \end{itemize}

    \item \textbf{Phase Gate (S):}
    \[
        S = \begin{pmatrix} 1 & 0 \\ 0 & i \end{pmatrix}
    \]
    Adds a $\pi/2$ phase to $\one$. Properties:
    \begin{itemize}
        \item Unitary but not Hermitian
        \item $S^2 = Z$
        \item Effect on $\ket{+}$ : \boxed{S\ket{+} =
          \frac{1}{\sqrt{2}}(\zero + i\one)}
    \end{itemize}

    \item \textbf{T Gate:}
    \[
        T = \begin{pmatrix} 1 & 0 \\ 0 & e^{i\pi/4} \end{pmatrix}
    \]
    Adds a $\pi/4$ phase to $\one$. Properties:
    \begin{itemize}
        \item $T^2 = S$
        \item $T^4 = Z$
        \item Often used in quantum error correction
    \end{itemize}

    \item \textbf{General Phase Gate } $P(\theta)$:
    \[
        P(\theta) = \begin{pmatrix} 1 & 0 \\ 0 & e^{i\theta} \end{pmatrix}
    \]
    Generalizes S and T gates: \boxed{S = P(\pi/2), T = P(\pi/4)}
\end{itemize}

\dfn{Important Relations}{
\begin{align*}
    Z &= H X H \\
    X &= H Z H
\end{align*}

Demonstrates duality between X and Z gates via Hadamard transformation.

Proof:
\[
HXH = \frac{1}{\sqrt{2}}\begin{pmatrix} 1 & 1 \\ 1 & -1 \end{pmatrix}
\begin{pmatrix} 0 & 1 \\ 1 & 0 \end{pmatrix}
\frac{1}{\sqrt{2}}\begin{pmatrix} 1 & 1 \\ 1 & -1 \end{pmatrix} =
\begin{pmatrix} 1 & 0 \\ 0 & -1 \end{pmatrix} = Z
\]
}

\dfn{Measurement}{Measurement collapses quantum states to basis states with
probabilities determined by amplitudes.}
\begin{itemize}
    \item \textbf{Z-basis:} Standard computational basis ($\zero$, $\one$)
        \begin{itemize}
            \item For state $\ket{\psi} = \alpha\zero + \beta\one$:
            \begin{align*}
                P(0) &= |\alpha|^2 \\
                P(1) &= |\beta|^2
            \end{align*}
        \end{itemize}
    \item \textbf{X-basis:} Hadamard basis ($\ket{+}$, $\ket{-}$)
      \begin{itemize}[label={*}]
            \item Measure in Z-basis after applying H gate
            \item $P(+) = |\bra{+}\psi\rangle|^2$
            \item $P(-) = |\bra{-}\psi\rangle|^2$
        \end{itemize}
    \item \textbf{Y-basis:} Eigenstates of Y
        \begin{itemize}
            \item $\ket{+i} = \frac{1}{\sqrt{2}}(\zero + i\one)$
            \item $\ket{-i} = \frac{1}{\sqrt{2}}(\zero - i\one)$
        \end{itemize}
\end{itemize}

\dfn{Multi-Qubit Systems}{States for multiple qubits are represented as
tensor products:}
\[
    \ket{\psi} = \sum_{x=0}^{2^n-1} \alpha_x \ket{x}, \quad \sum
    |\alpha_x|^2 = 1
\]

Properties of tensor products:
\begin{itemize}
    \item Not commutative: $(\ket{0} \otimes \ket{1} \neq \ket{1} \otimes
      \ket{0})$
    \item Associative: $((\ket{a} \otimes \ket{b}) \otimes \ket{c} = \ket{a}
      \otimes (\ket{b} \otimes \ket{c}))$
    \item Distributive: $((\alpha\ket{a} + \beta\ket{b}) \otimes \ket{c} =
      \alpha(\ket{a} \otimes \ket{c}) + \beta(\ket{b} \otimes \ket{c}))$
\end{itemize}

\vspace{0.3cm}

\ex{Example: Two-Qubit System}{
\[
    \ket{\psi} = \alpha_{00}\ket{00} + \alpha_{01}\ket{01} +
    \alpha_{10}\ket{10} + \alpha_{11}\ket{11}
\]
Bell state example:
\[
    \ket{\Phi^+} = \frac{1}{\sqrt{2}}(\ket{00} + \ket{11})
\]
This is a maximally entangled state.
}

\dfn{Circuit Diagrams}{Quantum circuits visually represent quantum operations
on qubits. They help to understand the sequence of quantum gates applied to
quantum states. The conventions are:}
\begin{itemize}
    \item Qubits are represented as horizontal lines (wires).
    \item Gates are shown as boxes or symbols on the wires.
    \item Time flows from left to right (the order of gate application).
    \item Measurement is depicted using a meter symbol.
    \item Classical information (post-measurement) is shown with double lines.
\end{itemize}

\subsubsection*{Single-Qubit Gate Example: Hadamard Gate}
\[
\begin{quantikz}
    \lstick{\ket{0}} & \gate{H} & \meter{}
\end{quantikz}
\]
This circuit applies the Hadamard gate $H$ to the state $\ket{0}$, creating
the superposition state $\frac{1}{\sqrt{2}}(\ket{0} + \ket{1})$, followed by
measurement.

\subsubsection*{Multi-Qubit System Example: CNOT Gate}
\[
\begin{quantikz}
    \lstick{\ket{0}} & \ctrl{1} & \qw \\
    \lstick{\ket{1}} & \targ{}  & \qw
\end{quantikz}
\]
This represents a controlled-NOT (CNOT) gate where the first qubit is the
control and the second is the target. If the control qubit is $\ket{1}$, the
target qubit flips; otherwise, it remains unchanged.

\subsubsection*{Bell State Preparation Circuit}
\[
\begin{quantikz}
    \lstick{\ket{0}} & \gate{H} & \ctrl{1} & \qw \\
    \lstick{\ket{0}} & \qw      & \targ{}  & \qw
\end{quantikz}
\]
This circuit prepares the Bell state $\ket{\Phi^+} =
\frac{1}{\sqrt{2}}(\ket{00} + \ket{11})$. The first qubit undergoes a
Hadamard gate, creating a superposition, followed by a CNOT gate that
entangles the qubits.

\subsubsection*{Measurement in Different Bases}
\[
\begin{quantikz}
    \lstick{\ket{\psi}} & \gate{H} & \meter{} & \cw \\
    \lstick{\ket{\phi}} & \gate{S} & \meter{} & \cw
\end{quantikz}
\]

\begin{itemize}
  \item The first qubit is measured in the X-basis by applying a Hadamard
    gate before measurement.
  \item The second qubit undergoes a phase shift with the $S$ gate and is
    measured in the computational (Z) basis.
\end{itemize}

\vspace{0.3cm}

\subsection*{Exercises}
\qs{Exercise 1}{Prove that the Hadamard gate is unitary and Hermitian.}

\qs{Exercise 2}{For $\ket{\psi} = \frac{1}{\sqrt{2}}(\ket{00} + \ket{11})$,
find measurement probabilities for $\ket{00}$ and $\ket{11}$.}

\qs{Exercise 3}{Determine if $U = \frac{1}{\sqrt{2}}\begin{pmatrix} 1 & i \\
 i & 1 \end{pmatrix}$ is unitary.}

\qs{Exercise 4}{If we apply H $\otimes$ H to $\ket{00}$, what state do we get?}

\sol{Exercise 1 Solution:
To prove H is unitary and Hermitian:
\begin{align*}
    H^\dagger &= \frac{1}{\sqrt{2}}\begin{pmatrix} 1 & 1 \\ 1 & -1
    \end{pmatrix} = H \\
    HH &= \frac{1}{2}\begin{pmatrix} 1 & 1 \\ 1 & -1
      \end{pmatrix}\begin{pmatrix} 1 & 1 \\ 1 & -1 \end{pmatrix} =
      \begin{pmatrix} 1 & 0 \\ 0 & 1 \end{pmatrix} = I
\end{align*}
Thus, H is both unitary ($HH^\dagger = I$) and Hermitian ($H = H^\dagger$).}

\vspace{0.3cm}

\sol{Exercise 2 Solution:
For $\ket{\psi} = \frac{1}{\sqrt{2}}(\ket{00} + \ket{11})$:
\begin{align*}
    P(00) &= |\bra{00}\psi\rangle|^2 = \left|\frac{1}{\sqrt{2}}\right|^2 =
    \frac{1}{2} \\
    P(11) &= |\bra{11}\psi\rangle|^2 = \left|\frac{1}{\sqrt{2}}\right|^2 =
    \frac{1}{2}
\end{align*}
}

\vspace{0.3cm}

\sol{Exercise 3 Solution:
To verify unitarity, compute $UU^\dagger$:
\begin{align*}
    U^\dagger &= \frac{1}{\sqrt{2}}\begin{pmatrix} 1 & -i \\ -i & 1
    \end{pmatrix} \\
    UU^\dagger &= \frac{1}{2}\begin{pmatrix} 1 & i \\ i & 1
      \end{pmatrix}\begin{pmatrix} 1 & -i \\ -i & 1 \end{pmatrix} =
      \begin{pmatrix} 1 & 0 \\ 0 & 1 \end{pmatrix}
\end{align*}
Therefore, U is unitary.}

\vspace{0.3cm}

\sol{Exercise 4 Solution:
\begin{align*}
    (H \otimes H)\ket{00} &= (H\ket{0}) \otimes (H\ket{0}) \\
    &= \frac{1}{\sqrt{2}}(\ket{0} + \ket{1}) \otimes
    \frac{1}{\sqrt{2}}(\ket{0} + \ket{1}) \\
    &= \frac{1}{2}(\ket{00} + \ket{01} + \ket{10} + \ket{11})
\end{align*}
This creates an equal superposition of all two-qubit basis states.}
