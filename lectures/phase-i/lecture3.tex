\section{Lecture 3: Quantum Bits and Quantum States} \label{sec:lecture3}

\dfn{Qubit}{A \textbf{qubit} is the fundamental unit of quantum information.
  Unlike a classical bit, which is either $0$ or $1$, a qubit can exist in a
  \vocab{superposition} of states:
  \[ |\psi\rangle = \alpha|0\rangle + \beta|1\rangle, \quad \text{where }
  \alpha, \beta \in \mathbb{C} \text{ and } ||\alpha||^2 + ||\beta||^2 = 1 \]
}


\vspace{0.3cm}

\noindent
Key features of qubits include:
  \begin{itemize}
    \ii \textbf{Superposition:} A qubit can exist simultaneously in multiple
    basis states.

    \ii \textbf{Complex Amplitudes:} Coefficients $\alpha$ and $\beta$ are
    complex numbers carrying magnitude and phase information.

    \ii \textbf{Interference:} Quantum states can interfere constructively or
    destructively.

    \ii \textbf{Entanglement:} Qubits can be correlated in ways that
    classical bits cannot.

\end{itemize}

\subsection*{Classical Computing Paradigms}

Quantum computing introduces a fundamentally different computational model:
  \begin{itemize}
    \ii \textbf{Deterministic Computing:} Uses discrete states (0 or 1) with
    predictable transitions.

    \ii \textbf{Analog Computing:} Uses continuous values susceptible to
    noise accumulation.

    \ii \textbf{Probabilistic Computing:} Represents probabilistic mixtures
    of states.

    \ii \textbf{Quantum Computing:} Allows coherent superposition with
    complex amplitudes and quantum interference.
\end{itemize}

\dfn{Dirac Notation}{Quantum states are represented using \vocab{Dirac
  notation} (bra-ket notation):
  \begin{itemize}
    \ii \textbf{Ket:} \( |0\rangle, |1\rangle \) represent computational
    basis states
    \ii Computational basis vectors:
    \[
      |0\rangle = \begin{bmatrix} 1 \\ 0 \end{bmatrix}, \quad
      |1\rangle = \begin{bmatrix} 0 \\ 1 \end{bmatrix}
    \]
    \ii General state: \( |\psi\rangle = \alpha|0\rangle + \beta|1\rangle \)
\end{itemize}}

\dfn{Basis States}{Common qubit bases include:
  \begin{itemize}
    \ii \textbf{Computational Basis:} \( |0\rangle, |1\rangle \)
    \ii \textbf{Hadamard Basis:}
    \[
      |+\rangle = \frac{1}{\sqrt{2}}(|0\rangle + |1\rangle) = \begin{bmatrix}
      1/\sqrt{2} \\ 1/\sqrt{2} \end{bmatrix}
    \]

    \[
      |-\rangle = \frac{1}{\sqrt{2}}(|0\rangle - |1\rangle) = \begin{bmatrix}
      1/\sqrt{2} \\ -1/\sqrt{2} \end{bmatrix}
    \]

    \ii \textbf{Phase/ Circular Polarization Basis:}
    \[ |L\rangle = \ket{+i} = \frac{1}{\sqrt{2}}(|0\rangle + i|1\rangle) \]
    \[|R\rangle = \ket{-i} = \frac{1}{\sqrt{2}}(|0\rangle - i|1\rangle) \]
\end{itemize}}

\subsection*{Bloch Sphere Representation}

\dfn{Bloch Sphere}{A geometric representation of a single qubit state:
  \[ |\psi\rangle = \lt[\cos\left(\frac{\theta}{2}\right)|0\rangle +
  e^{i\phi}\sin\left(\frac{\theta}{2}\right)|1\rangle\rt]e^{i\gamma} \]
  Where:
  \begin{itemize}
    \ii \( \theta \in [0,\pi] \) is the polar angle
    \ii \( \phi \in [0, 2\pi) \) is the azimuthal angle
    \ii \( \gamma \) is a global phase, often omitted since it cannot be
  represented on the Bloch sphere directly
\end{itemize}}
  \aside{Bloch Sphere Conversion to Cartesian Coordinates:
    \[ x = \sin\theta \cos\phi, \quad
      y = \sin\theta \sin\phi, \quad
      z = \cos\theta
    \]
  }

  Rearranging the Bloch sphere formula, we obtain that $\theta$ and
  $\phi$ can be expressed as:

  \[
    \theta = 2\arccos(\alpha_1), \quad \phi = -i
    \ln\left(\frac{\alpha_2}{\sin\left(\frac{\theta}{2}\right)}\right)
  \]

\ex{Example Bloch Sphere Representation}{
  \text{For the state } \(\theta = \frac{\pi}{2}, \phi = 0\):
  \bloch{90}{0}
}

\ex{Factoring Out the Global Phase}{
  Let's say that we have the following quantum state vector $\ket{\psi}$:

  \[
    \begin{aligned}
      \ket{\psi} & = \frac{1}{\sqrt{2}}\lt(i \zero + \one\rt) \\
      & = \frac{i}{\sqrt{2}}\zero + \frac{1}{\sqrt{2}}\one \\
      & = \underbrace{i}_{\text{global phase}}
      \lt(\frac{1}{\sqrt{2}}\zero + \frac{1}{\sqrt{2}}\one\rt) \\
    \end{aligned}
  \]
}


\subsection*{Quantum Measurement}
When a qubit is measured:
  \begin{itemize}
    \ii The quantum state \textit{collapses} to an eigenstate

    \ii Measurement probability depends on squared amplitude

    \ii Computational basis measurement probabilities:
    \[ P(0) = |\alpha|^2, \quad P(1) = |\beta|^2 \]

    \ii Post-measurement state:
    \[
      \boxed{|\psi_{\text{new}}\rangle = \frac{|b\rangle \langle b | \psi
        \rangle}{\sqrt{P(b)}}
      }
    \]

\end{itemize}

\ex{Measurement Example}{For the state \( |\psi\rangle =
  \frac{1}{\sqrt{3}}|0\rangle + \sqrt{\frac{2}{3}}|1\rangle \):
  \begin{itemize}
    \ii Probability of measuring \( |0\rangle \): \( P(0) = \frac{1}{3} \)
    \ii Probability of measuring \( |1\rangle \): \( P(1) = \frac{2}{3} \)
\end{itemize}}

\qs{Orthonormality Check}{Verify the inner products of basis states:
  \begin{align*}
    \langle 0 | 1 \rangle &= 0 \\
    \langle 0 | 0 \rangle &= 1 \\
    \langle + | + \rangle &= 1 \\
    \langle + | - \rangle &= 0
\end{align*}}

\sol{These relations hold due to the orthonormal nature of quantum basis
states.}

\nt{Quantum Bases and Their $\theta$ and $\phi$ Values:
  \begin{itemize}
    \item \textbf{Computational Basis:} \( \zero \rightarrow \theta = 0, \phi
      = 0, \quad \one \rightarrow \theta = \pi, \phi = 0 \)
    \item \textbf{Hadamard Basis:} \( \ket{+} \rightarrow \theta =
      \frac{\pi}{2}, \phi = 0, \quad \ket{-} \rightarrow \theta =
      \frac{\pi}{2}, \phi = \pi \)
  \item \textbf{Phase Basis:} \( \left|L\right\rangle = \ket{+i} \rightarrow
  \theta = \frac{\pi}{2}, \phi = \frac{\pi}{2}, \quad \left|R\right\rangle
  = \ket{-i} \rightarrow \theta = \frac{\pi}{2}, \phi = -\frac{\pi}{2} \)
\end{itemize}
}
