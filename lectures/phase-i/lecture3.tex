\section{Lecture 3: Quantum Bits and Quantum States}

\dfn{Qubit}{A \textbf{qubit} is the fundamental unit of quantum information. Unlike a classical bit, which is either $0$ or $1$, a qubit can exist in a \vocab{superposition} of states:
\[ |\psi\rangle = \alpha|0\rangle + \beta|1\rangle, \quad \text{where } \alpha, \beta \in \mathbb{C} \text{ and } |\alpha|^2 + |\beta|^2 = 1 \]

Key features of qubits include:
\begin{itemize}
    \ii \textbf{Superposition:} A qubit can exist simultaneously in multiple basis states.
    \ii \textbf{Complex Amplitudes:} Coefficients $\alpha$ and $\beta$ are complex numbers carrying magnitude and phase information.
    \ii \textbf{Interference:} Quantum states can interfere constructively or destructively.
    \ii \textbf{Entanglement:} Qubits can be correlated in ways that classical bits cannot.
\end{itemize}}

\dfn{Classical Computing Paradigms}{Quantum computing introduces a fundamentally different computational model:
\begin{itemize}
    \ii \textbf{Deterministic Computing:} Uses discrete states (0 or 1) with predictable transitions.
    \ii \textbf{Analog Computing:} Uses continuous values susceptible to noise accumulation.
    \ii \textbf{Probabilistic Computing:} Represents probabilistic mixtures of states.
    \ii \textbf{Quantum Computing:} Allows coherent superposition with complex amplitudes and quantum interference.
\end{itemize}}

\dfn{Dirac Notation}{Quantum states are represented using \vocab{Dirac notation} (bra-ket notation):
\begin{itemize}
    \ii \textbf{Ket:} \( |0\rangle, |1\rangle \) represent computational basis states
    \ii Computational basis vectors:
    \[
    |0\rangle = \begin{bmatrix} 1 \\ 0 \end{bmatrix}, \quad
    |1\rangle = \begin{bmatrix} 0 \\ 1 \end{bmatrix}
    \]
    \ii General state: \( |\psi\rangle = \alpha|0\rangle + \beta|1\rangle \)
\end{itemize}}

\dfn{Basis States}{Common qubit bases include:
\begin{itemize}
    \ii \textbf{Computational Basis:} \( |0\rangle, |1\rangle \)
    \ii \textbf{Hadamard Basis:}
    \[ |+\rangle = \frac{1}{\sqrt{2}}(|0\rangle + |1\rangle), \quad
       |-\rangle = \frac{1}{\sqrt{2}}(|0\rangle - |1\rangle) \]
    \ii \textbf{Circular Polarization Basis:}
    \[ |L\rangle = \frac{1}{\sqrt{2}}(|0\rangle + i|1\rangle), \quad
       |R\rangle = \frac{1}{\sqrt{2}}(|0\rangle - i|1\rangle) \]
\end{itemize}}

\dfn{Bloch Sphere}{A geometric representation of a single qubit state:
\[ |\psi\rangle = \cos\left(\frac{\theta}{2}\right)|0\rangle + e^{i\phi}\sin\left(\frac{\theta}{2}\right)|1\rangle \]
Where:
\begin{itemize}
    \ii \( \theta \in [0,\pi] \) is the polar angle
    \ii \( \phi \in [0, 2\pi) \) is the azimuthal angle
    \ii Cartesian coordinates:
    \[ x = \sin\theta \cos\phi, \quad
       y = \sin\theta \sin\phi, \quad
       z = \cos\theta \]

     \ex{Example Bloch Sphere Representation}{
       \text{For the state } \(\theta = \frac{\pi}{2}, \phi = 0\):
     \bloch{90}{0}
   }

\end{itemize}}

\dfn{Quantum Measurement}{When a qubit is measured:
\begin{itemize}
    \ii The quantum state \textit{collapses} to an eigenstate
    \ii Measurement probability depends on squared amplitude
    \ii Computational basis measurement probabilities:
    \[ P(0) = |\alpha|^2, \quad P(1) = |\beta|^2 \]
    \ii Post-measurement state:
    \[ |\psi_{\text{new}}\rangle = \frac{|b\rangle \langle b | \psi \rangle}{\sqrt{P(b)}} \]
\end{itemize}}

\ex{Measurement Example}{For the state \( |\psi\rangle = \frac{1}{\sqrt{3}}|0\rangle + \sqrt{\frac{2}{3}}|1\rangle \):
\begin{itemize}
    \ii Probability of measuring \( |0\rangle \): \( P(0) = \frac{1}{3} \)
    \ii Probability of measuring \( |1\rangle \): \( P(1) = \frac{2}{3} \)
\end{itemize}}

\qs{Orthonormality Check}{Verify the inner products of basis states:
\begin{align*}
    \langle 0 | 1 \rangle &= 0 \\
    \langle 0 | 0 \rangle &= 1 \\
    \langle + | + \rangle &= 1 \\
    \langle + | - \rangle &= 0
\end{align*}}

\sol{These relations hold due to the orthonormal nature of quantum basis states.}
