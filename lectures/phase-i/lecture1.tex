\section{Lecture 1: Overview of Quantum Computing Concepts}
\dfn{Quantum Computing}{Quantum computing is a computational paradigm
  leveraging quantum mechanical principles such as superposition, entanglement,
  and interference to perform computations that can surpass the capabilities of
  classical systems for specific tasks.\footnote{Superposition allows quantum
    bits (qubits) to exist in multiple states simultaneously, and entanglement
enables correlations between qubits even at a distance.}}

\subsection*{Historical Development of Quantum Computing}
\begin{itemize}
  \item \textbf{1980s-1990s:} Conception of quantum computing, with
    foundational ideas like the quantum Turing machine and quantum gates.
  \item \textbf{1990s-2000s:} Demonstration of key building blocks, such as
    quantum algorithms (e.g., Shor's and Grover's algorithms).
  \item \textbf{2016:} Emergence of quantum computing clouds, enabling
    access to quantum hardware via the internet.
  \item \textbf{2019:} First claims of \textbf{quantum advantage},
    showcasing tasks where quantum computers outperform classical
    counterparts.
  \item \textbf{2024:} Increasing qubit counts and improvements in quantum
    error correction techniques.
\end{itemize}

\subsection*{Applications of Quantum Computing}

Quantum computing offers \textbf{speedup} in areas such as:

\begin{enumerate}
  \item \textbf{Quantum Simulation:} Applications in chemistry, physics,
    and materials science, such as simulating molecular energy levels and
    drug discovery.
  \item \textbf{Security and Encryption:} Developing quantum-safe
    cryptographic protocols and random number generation.
  \item \textbf{Search and Optimization:} Enhancing solutions for weather
    forecasting, financial modeling, traffic planning, and resource
    allocation.
\end{enumerate}

\ex{Example: Quantum Speedup in Drug Discovery}{Drug discovery benefits from
  quantum simulation by enabling more accurate modeling of molecular
interactions, which classical computers struggle to achieve efficiently.}

\subsection*{Classical vs. Quantum Computing Paradigms}
\begin{itemize}
  \item \textbf{Classical Computing:} Utilizes traditional processing units
    (CPU, GPU, FPGA) and executes deterministic computations.
  \item \textbf{Quantum Computing:} Employs quantum processing units (QPU)
    with probabilistic computation based on quantum states.
\end{itemize}

\nt{Note: Classical computing paradigms still dominate in tasks that require
  precision and deterministic results. Quantum computing excels in
probabilistic or exponentially large state-space problems.}

