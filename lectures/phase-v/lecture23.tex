\section{Lecture 23: Neutral Atom Quantum Computers, Photonic Quantum Computers}
\label{sec:lecture23}

\index{neutral atom quantum computers}
\subsection*{Neutral Atom Quantum Computers}

Neutral atom quantum computers represent a promising platform for quantum
computing, leveraging individual neutral atoms as qubits. These atoms are
typically trapped using optical tweezers or other trapping methods, allowing
for precise control and manipulation.

\subsubsection*{Basic Principles}

In neutral atom quantum computing, qubits are encoded in the internal states
of the atoms, such as hyperfine levels or Rydberg states. Rydberg states, in
particular, are used for their strong interactions, which enable entangling
operations between distant atoms.

\vspace{0.3cm}

\noindent
The basic operations include:

\begin{itemize}
  \item \textbf{State Preparation:} Cooling and trapping atoms, then
    preparing them in the desired quantum state.
  \item \textbf{Single-Qubit Gates:} Manipulated using laser pulses to drive
    transitions between atomic states.
  \item \textbf{Two-Qubit Gates:} Achieved through Rydberg interactions,
    where the excitation of one atom affects another within the blockade
    radius.
  \item \textbf{Measurement:} Typically done by fluorescence detection or
    other methods to read out the atomic states.
\end{itemize}

\vspace{0.3cm}

\dfn{Rydberg Blockade}{
  The \textbf{Rydberg blockade} is a phenomenon where the excitation of one
  atom to a Rydberg state prevents nearby atoms from being excited due to
  strong van der Waals interactions. This enables entangling gates with a
  Hamiltonian of the form:

  \begin{equation*}
    H = \sum_i \frac{\Omega_i}{2} \sigma_x^i + \sum_i \frac{\Delta_i}{2}
    \sigma_z^i + \sum_{i<j} V_{ij} n_i n_j
  \end{equation*}

  where $\Omega_i$ is the Rabi frequency, $\Delta_i$ is the detuning,
  $V_{ij}$ is the interaction strength, and $n_i$ is the Rydberg state
  projector.
}

\subsubsection*{Hardware Implementations}

Leading companies in this field include QuEra and Pasqal. For example,
QuEra's Aquila system operates with up to 256 qubits in analogue mode,
utilizing Rydberg interactions for computation\footnote{See
\href{https://www.quera.com/aquila}{Aquila website}}. Pasqal offers systems
with up to 100 qubits, supporting both digital and analog computation modes
\footnote{See \href{https://www.pasqal.com/hardware}{Pasql website}}.

\begin{table}[h]
  \centering
  \caption{Key Neutral Atom Quantum Computing Systems}
  \begin{tabular}{|l|c|c|}
    \hline
    \textbf{Company} & \textbf{System} & \textbf{Qubit Count} \\
    \hline
    QuEra & Aquila & Up to 256 (analogue) \\
    Pasqal & Fresnel & Up to 100 \\
    \hline
  \end{tabular}
\end{table}

\index{neutral atom quantum computers!advantages}
\subsubsection*{Advantages}

\begin{itemize}
  \item \textbf{Scalability:} Neutral atom systems can potentially scale to
    thousands of qubits, limited mainly by laser power and optical system
    performance.
  \item \textbf{Long Coherence Times:} Trap lifetimes can be very long, with
    reported times up to 6000 seconds at cryogenic temperatures \footnote{See
    \href{https://epjquantumtechnology.springeropen.com/articles/10.1140/epjqt/s40507-023-00190-1}{Neutral atom quantum computing hardware: performance and end-user perspective}}.
  \item \textbf{High Gate Fidelities:} Two-qubit gate fidelities have reached
    $\sim$0.995, surpassing the threshold for error correction.
\end{itemize}

\index{neutral atom quantum computers!challenges}
\subsubsection*{Challenges}

\begin{itemize}
  \item \textbf{Preparation Times:} Rearranging atoms into desired
    configurations can be time-consuming, especially as the number of qubits
    increases.
  \item \textbf{Gate Speeds:} While gate fidelities are high, the speed of
    operations is slower compared to superconducting qubits, with two-qubit
    gates taking around 1 $\mu$s.
\end{itemize}

\index{neutral atom quantum computers!recent development@\textit{recent development}}
\subsubsection*{Recent Developments}

Recent advancements, particularly noted in 2024, include:

\begin{itemize}
  \item \textbf{Accurate Entangling Gates:} Demonstration of two-qubit
    operations with 99.5\% fidelity using reconfigurable atom arrays\footnote{
      See \href{https://phys.org/news/2024-09-neutral-atom-quantum-milestones.html}{
        Neutral atom innovations by quantum systems accelerator mark quantum
    computing milestones}}.

  \item \textbf{Error Correction:} Implementation of quantum low-density
    parity-check (qLDPC) codes, which reduce the resource overhead for error
    correction

  \item \textbf{Applications:} Successful use in quantum simulation for
    material science (e.g., high-temperature superconductors) and in solving
    combinatorial optimization problems, such as the Maximum Independent Set
\end{itemize}

\nt{
  Research suggests neutral atom quantum computers are gaining traction due
  to their scalability and high-fidelity operations. Companies like QuEra and
  Pasqal are pushing the boundaries, with applications already showing
  promise in specialized domains.
}

\index{photonic quantum computers}
\subsection*{Photonic Quantum Computers}

Photonic quantum computers use photons as qubits, offering a different
approach to quantum computing with unique advantages and challenges.

\subsubsection*{Basic Principles}

In photonic quantum computing, qubits are encoded in properties of photons,
such as polarization, path, or time-bin. Photons are particularly suitable
for quantum communication due to their ability to travel long distances with
low loss.

\vspace{0.3cm}

\noindent
Key operations include:

\begin{itemize}
  \item \textbf{State Preparation:} Generating single photons or entangled
    photon pairs.
  \item \textbf{Single-Qubit Gates:} Implemented using linear optical
    elements like beam splitters and phase shifters.
  \item \textbf{Two-Qubit Gates:} Achieved through interference and
    measurement, often requiring post-selection or feed-forward techniques.
  \item \textbf{Measurement:} Detected using photon-number-resolving
    detectors or other methods.
\end{itemize}

\vspace{0.3cm}

\dfn{Linear Optical Quantum Computing}{
  \textbf{Linear optical quantum computing} uses linear optical elements to
  manipulate photonic qubits. The Knill-Laflamme-Milburn (KLM) protocol
  enables universal quantum computing with linear optics, ancilla photons,
  and post-selection, described by the transformation:

  \begin{equation*}
    U = e^{i (a^\dagger b + a b^\dagger) \theta}
  \end{equation*}

  where $a$ and $b$ are photon creation operators, and $\theta$ is the phase
  shift.
}

\subsubsection*{Hardware Implementations}

Major players in photonic quantum computing include PsiQuantum and Xanadu.
PsiQuantum is developing a manufacturable photonic quantum computer using
silicon photonics, with plans for a commercial system by 2029\footnote{See
\href{https://thequantuminsider.com/2025/03/25/sources-psiquantum-raising-750-million-to-push-photonic-quantum-computers-toward-commercial-reality/}{Sources: PsiQuantum Raising \$750 Million to Push Photonic Quantum Computers Toward Commercial Reality}}
In 2025, PsiQuantum announced the Omega chipset, a utility-scale photonic
quantum computing chipset with improved components, such as better photon
detectors and reduced signal loss in optical waveguides
\footnote{See \href{https://thequantuminsider.com/2025/02/26/psiquantum-announces-omega-a-manufacturable-photonic-quantum-computing-chipset/}{PsiQuantum Announces Omega, a Manufacturable Photonic Quantum Computing Chipset}}

\vspace{0.3cm}

Additionally, a scale model of a photonic quantum computer called Aurora was
demonstrated in 2025, using 35 photonic chips to synthesize large cluster
states and implement quantum error correction \footnote{\href{https://www.nature.com/articles/s41586-024-08406-9}{Scaling and networking a modular photonic quantum computer
}}.

\begin{table}[h]
  \centering
  \caption{Key Photonic Quantum Computing Systems}
  \begin{tabular}{|l|c|c|}
    \hline
    \textbf{Company/Project} & \textbf{System} & \textbf{Status} \\
    \hline
    PsiQuantum & Omega & In development, commercial by 2029 \\
    Aurora & Scale model & 35 chips, demonstrated 2025 \\
    Xanadu & Borealis & Cloud-accessible prototype \\
    \hline
  \end{tabular}
\end{table}

\subsection*{Photonic Quantum Computers}

Photonic quantum computers use photons as qubits, offering a different
approach to quantum computing with unique advantages and challenges.

\index{photonic quantum computers!advantages}
\subsubsection*{Advantages}

\begin{itemize}
  \item \textbf{Room-Temperature Operation:} Most components operate at room
    temperature, reducing cooling requirements.
  \item \textbf{Integration with Networks:} Photons can be easily transmitted
    over fiber optics, facilitating integration with classical and quantum
    networks.
  \item \textbf{Low Decoherence:} Photons are less susceptible to decoherence
    compared to other qubit modalities.
\end{itemize}

\index{photonic quantum computers!challenges}
\subsubsection*{Challenges}

\begin{itemize}
  \item \textbf{High Loss Rates:} Current photonic systems suffer from
    significant optical losses, with total losses around 56\% in some
    components. Fault tolerance requires loss budgets of $\sim$1\%, which is
    a major hurdle.
  \item \textbf{Scalability:} While photons are ideal for communication,
    creating and manipulating large numbers of entangled photons is challenging.
\end{itemize}

\index{photonic quantum computers!recent development@\textit{recent development}}
\subsubsection*{Recent Developments}

Significant progress in 2025 includes:

\begin{itemize}
  \item \textbf{Aurora System:} A 35-chip photonic quantum computer that
    demonstrated the synthesis of a cluster state with 86.4 billion modes and
    implemented a quantum error correction code.
  \item \textbf{PsiQuantum's Funding and Progress:} In 2025, PsiQuantum
    raised at least \$750 million to advance its photonic quantum computing
    technology, with significant government support from the U.S., Australia,
    and the UK.
  \item \textbf{Component Improvements:} Developments in photon detectors,
    optical waveguides, and high-speed switches have improved the performance
    of photonic systems.
\end{itemize}

\index{photonic quantum computers!applications}
\subsubsection*{Applications}

Photonic quantum computers are particularly suited for:

\begin{itemize}
  \item \textbf{Quantum Communication:} Enabling secure communication
    protocols like quantum key distribution.
  \item \textbf{Quantum Simulation:} Simulating quantum systems that are
    difficult to model classically.
  \item \textbf{General Quantum Computing:} Once fault tolerance is achieved,
    they could perform a wide range of computational tasks.
\end{itemize}

\nt{
  The evidence leans toward photonic quantum computers being a strong
  contender for quantum communication and networking, with significant
  investments in 2025 signaling confidence in their commercial potential.
  However, overcoming loss rates remains a critical challenge.
}

%%%%%%%%%%%%%

\subsection*{Comparison and Future Outlook}

Both neutral atom and photonic quantum computers offer distinct advantages
and face unique challenges.

\begin{itemize}
  \item \textbf{Neutral Atom Systems:} More mature in terms of qubit numbers
    and gate fidelities, with systems already operating with hundreds of
    qubits. However, they require cryogenic temperatures for some operations
    and have slower gate speeds.
  \item \textbf{Photonic Systems:} Operate at room temperature and have
    natural advantages for communication and networking. However, they
    currently suffer from high loss rates and are less mature in terms of
    large-scale implementation.
\end{itemize}

The future of quantum computing may see a combination of these technologies,
with photonic systems handling communication and neutral atom systems
performing computations, or vice versa. Alternatively, one technology might
emerge as the leader once fault tolerance is achieved.

\vspace{0.3cm}

In terms of timelines, PsiQuantum aims to deliver a commercial photonic
quantum computer by 2029, while neutral atom systems are already being used
for specific applications today. Research suggests that hybrid approaches or
advancements in error correction could accelerate progress in both fields.

\vspace{0.3cm}

\aside{
  The race to build practical quantum computers continues, with both neutral
  atom and photonic approaches showing significant promise. As research
  progresses, we can expect to see larger systems, improved error correction,
  and new applications emerging in the coming years.
}

%%%%%%%%%%%%%%%%%%%%%%%%%%%%%%%%%%%%%%%%%%%%%%
% End of Lecture 23
%%%%%%%%%%%%%%%%%%%%%%%%%%%%%%%%%%%%%%%%%%%%%%

