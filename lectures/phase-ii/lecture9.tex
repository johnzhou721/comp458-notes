\section{Lecture 9: Grover's Search Algorithm}\label{sec:lecture9}

%%%%%%%%%%%%%%%%%%%%%%%%%%%%%%%%%%%%%%%%%%%%%%%%%%
% Problem Statement
%%%%%%%%%%%%%%%%%%%%%%%%%%%%%%%%%%%%%%%%%%%%%%%%%%
\index{Grover's Search Algorithm!problem statement}
\subsection*{Problem Statement}

Given a bitstring \( x \in \{0, 1\}^n \) and a function

\[
  f(x) =
  \begin{cases}
    1, & \text{if } x \text{ is the marked (or winning) state}, \\
    0, & \text{otherwise},
  \end{cases}
\]

our goal is to find the unique (or one of the) \( x \) such that \( f(x)=1 \).

\vspace{0.3cm}

In plain terms, imagine you have an unsorted database of \( N = 2^n \) items,
and only one item is “special.” Classically, you must check each item one by
one (on average, \( O(2^n) \) trials) to find the special item. Grover's
algorithm, however, uses quantum amplitude amplification to solve this
problem in only \( O(\sqrt{2^n}) \) iterations—a quadratic speedup over
classical search.

%%%%%%%%%%%%%%%%%%%%%%%%%%%%%%%%%%%%%%%%%%%%%%%%%%
% Grover's Algorithm Circuit (General n-qubit Version)
%%%%%%%%%%%%%%%%%%%%%%%%%%%%%%%%%%%%%%%%%%%%%%%%%%

\index{Grover's Search Algorithm!circuit}
\subsection*{Grover's Algorithm Circuit}

The general structure of Grover's algorithm is as follows:

\begin{enumerate}
  \item \textbf{State Preparation:} Initialize all qubits in the state
    \(\ket{0}^{\otimes n}\) and apply \( H^{\otimes n} \) to create a uniform
    superposition:

    \[
      \ket{s} = H^{\otimes n} \ket{0}^{\otimes n} = \frac{1}{\sqrt{2^n}}
      \sum_{x \in \{0,1\}^n} \ket{x}.
    \]

  \item \textbf{Grover Iteration:} Repeat the following two steps
    approximately \(\sqrt{2^n}\) times:

    \begin{itemize}
      \item \textbf{Oracle \(O\):} Flip the phase of the marked state(s);
        that is, for every \( x \),

        \[
          O \ket{x} = (-1)^{f(x)} \ket{x}.
        \]

        For example, if the winning state is \(\ket{w}\), then
        \( O\ket{w} = -\ket{w} \).

      \item \textbf{Diffusion Operator \(D\):} Reflect all amplitudes about
        the average amplitude. This operator is given by

        \[
          D = 2\ket{s}\bra{s} - I.
        \]

        In practice, \( D \) is implemented as

        \[
          D = H^{\otimes n} \; X^{\otimes n} \; (CZ) \; X^{\otimes n} \;
          H^{\otimes n},
        \]

        where the \(CZ\) gate here represents a controlled phase flip on
        \(\ket{1}^{\otimes n}\) (for \(n>2\), this is a multi-controlled \(Z\)
        gate).
    \end{itemize}
\end{enumerate}

The high-level circuit for an \( n \)-qubit Grover algorithm is illustrated as:

\[
\begin{quantikz}
  \lstick{$q_{n - 1}$} & \gate{H} &  \gate[5]{\shortstack{$f(x)$ \\ Oracle}} & \qw & \gate[5]{\text{Diffusion Circuit}} & \meter{} \\
  \lstick{$q_{n - 2}$} & \gate{H} &  &  & & \meter{} \\
  \vdots & \vdots & & \vdots & & \vdots \\
  \lstick{$q_1$} & \gate{H} & &  & & \meter{} \\
  \lstick{$q_0$} & \gate{H} &  & & & \meter{}
\end{quantikz}
\]

\[
  \hspace{1.5cm}
  \underbrace{\hspace{6cm}}_{\text{Repeat } \sqrt{2^n} \text{ times}}
\]

%%%%%%%%%%%%%%%%%%%%%%%%%%%%%%%%%%%%%%%%%%%%%%%%%%
% 2-Qubit Example
%%%%%%%%%%%%%%%%%%%%%%%%%%%%%%%%%%%%%%%%%%%%%%%%%%
\subsubsection*{2-Qubit Example}

To build intuition, consider the case \( n=2 \) (i.e., \( N=4 \)) with the winning
state chosen as \(\ket{11}\).

\vspace{0.3cm}

\textbf{Oracle:} For a 2-qubit system, the Oracle \(O\) can be implemented as a
controlled-\(Z\) (CZ) gate:
\[
  CZ =
  \begin{pmatrix}
    1 & 0 & 0 & 0 \\
    0 & 1 & 0 & 0 \\
    0 & 0 & 1 & 0 \\
    0 & 0 & 0 & -1
  \end{pmatrix},
\]

which multiplies the state \(\ket{11}\) by \(-1\).

\vspace{0.3cm}

\textbf{Diffusion Operator:} For 2 qubits, the diffusion operator is given by:

\[
  D = H^{\otimes 2}\; X^{\otimes 2}\; CZ\; X^{\otimes 2}\; H^{\otimes 2}.
\]

\vspace{0.3cm}

\textbf{2-Qubit Grover Circuit:}

\[
\begin{quantikz}
  \lstick{$\ket{0}$} & \gate{H} & \ctrl{1} & \gate{H} & \gate{X} & \qw & \ctrl{1} & \qw & \gate{X} & \gate{H} & \meter{} \\
  \lstick{$\ket{0}$} & \gate{H} & \ctrl{0} & \gate{H} & \gate{X} & \gate{H} & \gate{X} & \gate{H} & \gate{X} & \gate{H} & \meter{}
\end{quantikz}
\]

\vspace{0.3cm}

\noindent
\textbf{Explanation:}

\begin{itemize}
  \item \textbf{Oracle:} The Oracle adds a phase of \(-1\) to the winning
    state \(\ket{11}\), effectively marking it.
  \item \textbf{Diffusion Operator:} The diffusion circuit (implemented as
    \(H\;X\;CZ\;X\;H\)) reflects all amplitudes about the average. This
    inversion about the mean amplifies the amplitude of the marked state.
\end{itemize}

%%%%%%%%%%%%%%%%%%%%%%%%%%%%%%%%%%%%%%%%%%%%%%%%%%
% Algorithm Walkthrough and Pseudocode
%%%%%%%%%%%%%%%%%%%%%%%%%%%%%%%%%%%%%%%%%%%%%%%%%%

\index{Grover's Search Algorithm!pseudocode}
\subsection*{Walkthrough of the Algorithm and Generalization to \(n\) Qubits}

Grover's algorithm can be summarized in the following pseudocode:

\begin{algorithm}[H]
  \caption{Grover's Search Algorithm}
  \KwIn{A function \( f: \{0,1\}^n \rightarrow \{0,1\} \) with a unique
  marked state \( x_0 \) such that \( f(x_0) = 1 \).}
  \KwOut{The marked element \( x_0 \).}

  \textbf{Initialize:} Set the quantum state to \( \ket{\psi} \gets
  \ket{0}^{\otimes n} \).\\

  Apply \( H^{\otimes n} \) to obtain the uniform superposition:

  \[
    \ket{s} = H^{\otimes n}\ket{0}^{\otimes n} = \frac{1}{\sqrt{2^n}} \sum_{x
    \in \{0,1\}^n} \ket{x}.
  \]

  \For{\( i = 1 \) \KwTo \( \left\lfloor \frac{\pi}{4}\sqrt{2^n}
  \right\rfloor \)}{

    Apply the Oracle \( O \) which performs:

    \[
      O\ket{x} = (-1)^{f(x)}\ket{x},
    \]
    (i.e., flip the phase of the marked state).\;

    Apply the Diffusion Operator \( D = 2\ket{s}\bra{s} - I \)
    (which reflects amplitudes about the average).\;
  }

  Measure the state in the computational basis to obtain \( x_0 \).\;

\end{algorithm}

\vspace{0.3cm}

\noindent
\textbf{Detailed Explanation:}
\begin{enumerate}
  \item \textbf{Initialization:} All qubits are set to \(\ket{0}\) and then
    put into an equal superposition via \(H^{\otimes n}\).

  \item \textbf{Oracle Application:} The Oracle selectively flips the phase
    of the winning state(s) (e.g., for the marked state \(\ket{w}\),
    \(O\ket{w} = -\ket{w}\)).

  \item \textbf{Diffusion (Inversion about the Mean):} This operator reflects
    the state vector about the average amplitude. Geometrically, one can
    view the process as a rotation in a two-dimensional subspace (see
    below).

  \item \textbf{Iteration:} Repeating the Oracle and Diffusion steps
    approximately \( \sqrt{2^n} \) times amplifies the probability amplitude of
    the marked state.

  \item \textbf{Measurement:} Finally, a measurement in the computational
    basis yields the marked element with high probability.
\end{enumerate}

This algorithm demonstrates how quantum algorithms can achieve a quadratic
speedup over classical brute-force search methods.

%%%%%%%%%%%%%%%%%%%%%%%%%%%%%%%%%%%%%%%%%%%%%%%%%%
% End of Lecture 9
%%%%%%%%%%%%%%%%%%%%%%%%%%%%%%%%%%%%%%%%%%%%%%%%%%
