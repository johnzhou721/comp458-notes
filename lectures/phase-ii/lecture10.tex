\section{Lecture 10: Quantum Logic Gates, Applying Grover's Search Algorithm
to SAT}\label{sec:lecture10}

%%%%%%%%%%%%%%

\subsection*{Review Question from Last Lecture}

\qs{Function of the Hadamard Layer in Grover's Algorithm}{
  Why does Grover's algorithm apply a Hadamard ($H$) gate to each of the $n$
  qubits at the beginning of the circuit?
}

\sol{
  The purpose of the Hadamard layer is that it allows up to work with all
  bitstring permutations simultaneously by putting all states in equal
  superposition.

  The Hadamard layer initializes the system in a uniform superposition of all
  possible $2^n$ states, allowing Grover's algorithm to explore all bitstring
  permutations simultaneously. For $n$ qubits starting at $|0\rangle^{\otimes
  n}$, applying $H^{\otimes n}$ yields:

  \[
    |s\rangle = H^{\otimes n} |0\rangle^{\otimes n} = \frac{1}{\sqrt{2^n}}
    \sum_{x \in \{0,1\}^n} |x\rangle.
  \]

  This equal superposition ensures that the algorithm can amplify the
  amplitude of the solution state(s) efficiently.
}

%%%%%%%%%%%%%

\subsection*{Quantum Implementations of Classical Logic Gates}

\index{quantum gates!OR gate}
\subsubsection*{OR Gate}

The quantum OR gate computes $c = a \lor b$ using a combination of CNOT and
Toffoli gates. Below is the two-way OR implementation:

\[
  \begin{quantikz}
    \lstick{$|a\rangle$} & \ctrl{1} & \ctrl{2} & \qw \\
    \lstick{$|b\rangle$} & \ctrl{1} & \targ{} & \qw \\
    \lstick{$|c\rangle$} & \targ{} & \targ{} & \qw
  \end{quantikz}
\]

\textbf{Truth Table:}
\[
  \begin{array}{ccc|c}
    a & b & c_{\text{in}} & c_{\text{out}} = a \lor b \\
    \hline
    0 & 0 & 0 & 0 \\
    0 & 1 & 0 & 1 \\
    1 & 0 & 0 & 1 \\
    1 & 1 & 0 & 1 \\
  \end{array}
\]

The circuit uses CCNOT (Toffoli) to set $c$ if both $a$ and $b$ are 1, then
CNOTs to flip $c$ if either $a$ or $b$ is 1, effectively computing OR. NOT
gates ($X$) can be added to inputs for negated terms (e.g., $\neg a \lor \neg
b$).

\index{quantum gates!NOR gate}
\subsubsection*{NOR Gate}

A NOR gate ($c = \neg (a \lor b)$) can be implemented by adding an $X$ gate
on the output qubit after the OR circuit:

\[
  \begin{quantikz}
    \lstick{$|a\rangle$} & \ctrl{1} & \ctrl{2} & \qw \\
    \lstick{$|b\rangle$} & \ctrl{1} & \targ{} & \qw \\
    \lstick{$|c\rangle$} & \targ{} & \targ{} & \gate{X} & \qw
  \end{quantikz}
\]


\textbf{Truth Table:}
\[
  \begin{array}{ccc|c}
    a & b & c_{\text{in}} & c_{\text{out}} = \neg (a \lor b) \\
    \hline
    0 & 0 & 0 & 1 \\
    0 & 1 & 0 & 0 \\
    1 & 0 & 0 & 0 \\
    1 & 1 & 0 & 0 \\
  \end{array}
\]

\index{quantum gates!AND gate}
\subsubsection*{AND Gate}

An AND gate ($c = a \land b$) is directly implemented with a Toffoli gate:

\[
  \begin{quantikz}
    \lstick{$| a \rangle$} & \ctrl{1} & \qw \\
    \lstick{$| b \rangle$} & \ctrl{1} & \qw \\
    \lstick{$| c \rangle$} & \targ{} & \qw
  \end{quantikz}
\]

\textbf{Truth Table:}
\[
  \begin{array}{ccc|c}
    a & b & c_{\text{in}} & c_{\text{out}} = a \land b \\
    \hline
    0 & 0 & 0 & 0 \\
    0 & 1 & 0 & 0 \\
    1 & 0 & 0 & 0 \\
    1 & 1 & 0 & 1 \\
  \end{array}
\]

\index{quantum gates!NAND gate}
\subsubsection*{NAND Gate}
A NAND gate ($c = \neg (a \land b)$) adds an $X$ gate after the AND:

\[
  \begin{quantikz}
    \lstick{$|a\rangle$} & \ctrl{1} & \qw \\
    \lstick{$|b\rangle$} & \ctrl{1} & \qw \\
    \lstick{$|c\rangle$} & \targ{} & \gate{X} & \qw
  \end{quantikz}
\]

\textbf{Truth Table:}
\[
  \begin{array}{ccc|c}
    a & b & c_{\text{in}} & c_{\text{out}} = \neg (a \land b) \\
    \hline
    0 & 0 & 0 & 1 \\
    0 & 1 & 0 & 1 \\
    1 & 0 & 0 & 1 \\
    1 & 1 & 0 & 0 \\
  \end{array}
\]

%%%%%%%%%%%%%

\subsection*{Applying Grover's Algorithm to SAT}

\index{quantum kickback}
\dfn{Quantum Kickback}{
  Quantum kickback occurs when a controlled operation transfers a phase from
  the target qubit to the control qubit(s). In Grover's algorithm, the oracle
  uses this to mark solution states. Consider a simple example with a
  controlled-$X$ and phase:

  \[
    \begin{quantikz}
      \lstick{$|+\rangle$} & \ctrl{1} & \qw \\
      \lstick{$|-\rangle$} & \targ{} & \qw
    \end{quantikz}
  \]

  Initial state: $|+\rangle|-\rangle = \frac{1}{\sqrt{2}}(|0\rangle +
  |1\rangle) \otimes \frac{1}{\sqrt{2}}(|0\rangle - |1\rangle) =
  \frac{1}{2}(|00\rangle - |01\rangle + |10\rangle - |11\rangle)$.

  After CNOT:

  \[
    \frac{1}{2}(|00\rangle - |01\rangle + |11\rangle - |10\rangle) =
    \frac{1}{\sqrt{2}}(|0\rangle|-\rangle - |1\rangle|+\rangle).
  \]

  The phase of the target ($|-\rangle$) "kicks back" to the control, flipping
  its relative phase. In Grover’s oracle, a multi-controlled $X$ on an
  ancilla in $|-\rangle$ state marks solutions with a $-1$ phase.
}

\paragraph{Implications of Quantum Kickback}\label{par:Implications of
Quantum Kickback}

Quantum kickback enables efficient phase marking in Grover’s algorithm
without altering the computational basis states directly. It implies that:

\begin{itemize}
  \item Ancilla qubits in superposition (e.g., $|-\rangle$) can encode
    solution conditions.\footnote{\textbf{ancilla}: extra qubits (beyond the
    clauses) that we need to complete the complete the computation}

  \item The oracle’s design must ensure correct phase inversion for all
    satisfying assignments.

  \item When designing algorithms, we leverage kickback to avoid classical
    computation of each clause, reducing circuit depth and enhancing quantum
    speedup.
\end{itemize}

\index{Grover's Search Algorithm!Alexandria Ship Example@\textit{Alexandria
Ship Example}}
\ex{Alexandria Ship Example}{

  In 48 BCE Alexandria, a merchant must maximize passengers on a ship to
  Italy under these constraints:

  \begin{enumerate}
    \item \textit{Arsinoe (A)} won’t go if Cleopatra (C) is on the ship:
      $\neg A \lor \neg C$.

    \item \textit{Arsinoe} won’t go alone: $\neg A \lor J \lor P$.

    \item \textit{Ptolemy (P)} only goes if Julius Caesar (J) goes: $\neg P
      \lor J$.

    \item \textit{Ptolemy} won’t go if Cleopatra is on: $\neg P \lor \neg C$.

    \item \textit{Julius Caesar} must go: $J$.

    \item \textit{Cleopatra} won’t go if Julius Caesar is on: $\neg C \lor
      \neg J$.
  \end{enumerate}

  \textbf{Goal:} Find the maximum number of passengers (A, C, J, P)
  satisfying all constraints using Grover’s algorithm.
}

\paragraph{Creating SAT Clauses}\label{par:Creating SAT Clauses} % (fold)

The problem is a SAT instance with clauses:

\begin{enumerate}
  \item $\neg A \lor \neg C$ (Arsinoe-Cleopatra conflict).
  \item $\neg A \lor J \lor P$ (Arsinoe not alone).
  \item $\neg P \lor J$ (Ptolemy needs Caesar).
  \item $\neg P \lor \neg C$ (Ptolemy-Cleopatra conflict).
  \item $J$ (Caesar must go).
  \item $\neg C \lor \neg J$ (Cleopatra-Caesar conflict).
\end{enumerate}

These are implemented as OR gates in the quantum circuit, with an
ancilla qubit per clause. The final ancilla ($q_9$) computes the AND of all
clauses using a multi-controlled $X$, marking satisfying assignments with a
phase flip.

% paragraph Creating SAT Clauses (end)

\subsubsection*{\texttt{Cirq} Implementation}

Below is the Cirq code with detailed explanations:

\begin{minted}[linenos,highlightlines={6-10},highlightcolor=yellow!20]{python}
import cirq

# Qubits: A=0, C=1, J=2, P=3, clause ancillas=4-8, final output=9
qubits = cirq.LineQubit.range(10)

def twoway_OR(circuit, a, b, c, not_a, not_b):
    if not_a: circuit.append(cirq.X(a))
    if not_b: circuit.append(cirq.X(b))
    circuit.append(cirq.CCNOT(a, b, c))  # AND to target
    circuit.append(cirq.CNOT(a, c))      # OR logic
    circuit.append(cirq.CNOT(b, c))
    if not_a: circuit.append(cirq.X(a))
    if not_b: circuit.append(cirq.X(b))
    return circuit
\end{minted}
\textit{Lines 6-10}: Implements $c = a \lor b$ by computing AND then adding
single-qubit terms, with optional inversions for negated inputs.

\begin{minted}[linenos,highlightlines={11-14}]{python}
clauses_circuit = cirq.Circuit()
twoway_OR(clauses_circuit, qubits[0], qubits[1], qubits[4], 1, 1)  # A or not C
threeway_OR(clauses_circuit, qubits[0], qubits[2], qubits[3], qubits[5], 1, 0, 0)  # not A or J or P
twoway_OR(clauses_circuit, qubits[1], qubits[2], qubits[6], 1, 1)  # not C or not J
twoway_OR(clauses_circuit, qubits[1], qubits[3], qubits[7], 1, 1)  # not C or not P
twoway_OR(clauses_circuit, qubits[2], qubits[3], qubits[8], 0, 1)  # J or not P

oracle_circuit = cirq.Circuit()
oracle_circuit.append(clauses_circuit)
oracle_circuit.append(cirq.X(qubits[9]).controlled_by(qubits[2], qubits[4], qubits[5],
                                                      qubits[6], qubits[7], qubits[8]))
oracle_circuit.append(clauses_circuit)
\end{minted}
\textit{Lines 11-14}: The oracle computes all clauses, then uses a
multi-controlled $X$ on $q_9$ (in $|-\rangle$ state) to flip the phase of
satisfying states, leveraging kickback.

\begin{minted}[linenos,highlightlines={5-7}]{python}
final_circuit = cirq.Circuit()
final_circuit.append(cirq.H.on_each(qubits[0:4]))  # Superposition
final_circuit.append(cirq.X(qubits[9]))            # Prepare ancilla
final_circuit.append(cirq.H(qubits[9]))
final_circuit.append(oracle_circuit)
final_circuit.append(diffusion_circuit)            # From notebook
final_circuit.append(cirq.measure(qubits[0:4][::-1], key='PJCA'))
\end{minted}

\textit{Lines 5-7}: Initializes input qubits in superposition and ancilla in
$|-\rangle$, applies oracle and diffusion, then measures.

\paragraph{Results from the Circuit}\label{par:Results from the Circuit}
Running the circuit with 8192 repetitions yields:

\begin{verbatim}
Counter({5: 2079, 13: 2043, 12: 2037, 4: 2033})
\end{verbatim}

\noindent
Converting to binary (PJCA order):

\begin{itemize}
  \item $5 = 0101$: $P=0, J=1, C=0, A=1$ (J, A).
  \item $13 = 1101$: $P=1, J=1, C=0, A=1$ (P, J, A).
  \item $12 = 1100$: $P=1, J=1, C=0, A=0$ (P, J).
  \item $4 = 0100$: $P=0, J=1, C=0, A=0$ (J).
\end{itemize}

All solutions have $J=1$ (Caesar goes), $C=0$ (Cleopatra stays), and vary in
A and P, satisfying all constraints. The maximum passengers (3) is $P=1, J=1,
A=1$, excluding Cleopatra, optimizing the merchant’s goal.

We have shown how Grover’s Search Algorithm efficiently solves this SAT
problem, amplifying valid configurations quadratically faster than classical
search!

\emph{However}, when we look at the representation of this circuit, it is
clear that there are some optimizations that we can do to make it have a much
shallower depth and not rely on so many ancilla qubits--- which we will fix
in the next chapter.

%%%%%%%%%%%%%%%%%%%%%%%%%%%%%%%%%%%%%%%%%%%%%%
% End of Lecture 10
%%%%%%%%%%%%%%%%%%%%%%%%%%%%%%%%%%%%%%%%%%%%%%
