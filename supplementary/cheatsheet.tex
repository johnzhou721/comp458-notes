\section{Cheatsheet}\label{sec:cheatsheet}

% Local formatting for the cheatsheet
\begingroup
\setlength{\parindent}{0pt}
\setlength{\parskip}{0pt plus 0.5ex}
\raggedright
\footnotesize

\makeatother

\begin{multicols}{3}

  \section*{Linear Algebra Notation}
  \begin{itemize}[leftmargin=*,nosep,topsep=0pt]
    \item \textbf{Bra-Ket}: Ket $|\psi\rangle = \begin{bmatrix} \alpha \\
      \beta \end{bmatrix}$, Bra $\langle\psi| = [\alpha^* \; \beta^*]$

    \item \textbf{Inner Product}: $\langle\phi|\psi\rangle = \phi^\dagger\psi
      = \sum_i \phi_i^*\psi_i$

    \item \textbf{Norm}: $\|\psi\|^2 = \langle\psi|\psi\rangle = \sum_i
      |\psi_i|^2$

    \item \textbf{Outer Product}: $|\psi\rangle\langle\phi| = \begin{bmatrix}
        \psi_1\phi_1^* & \psi_1\phi_2^* \\ \psi_2\phi_1^* & \psi_2\phi_2^*
      \end{bmatrix}$

    \item \textbf{Tensor Product}: $|\psi\rangle \otimes |\phi\rangle =
      \begin{bmatrix} \psi_1\phi_1 \\ \psi_1\phi_2 \\ \psi_2\phi_1 \\
      \psi_2\phi_2 \end{bmatrix}$

    \item \textbf{Matrix Mult}: $(AB)_{ij} = \sum_k A_{ik}B_{kj}$

    \item \textbf{Unitary}: $U^\dagger U = UU^\dagger = I$, preserves norms
      and inner products

    \item \textbf{Hermitian}: $H = H^\dagger$, eigenvalues are real

    \item \textbf{Magnitude}: Real: $\|v\| = \sqrt{\sum_i v_i^2}$, Complex:
      $\|v\| = \sqrt{\sum_i |v_i|^2}$

  \end{itemize}

  \conceptbox{Important Property}{Tensor products are \textit{not commutative}:
  $A \otimes B \neq B \otimes A$ generally}

  \section*{Qubit Representation}
  \begin{itemize}[leftmargin=*,nosep,topsep=0pt]
    \item \textbf{Dirac Notation}: $|\psi\rangle = \alpha|0\rangle +
      \beta|1\rangle$, where $\alpha,\beta \in \mathbb{C}$ and $|\alpha|^2 +
      |\beta|^2 = 1$

    \item \textbf{Computational Basis}: $|0\rangle = \begin{bmatrix} 1 \\ 0
        \end{bmatrix}$, $|1\rangle = \begin{bmatrix} 0 \\ 1 \end{bmatrix}$
    \end{itemize}

    \section*{Universal Bases}
    \begin{itemize}[leftmargin=*,nosep,topsep=0pt]
      \item \textbf{Computational}: $\{|0\rangle, |1\rangle\}$ \\
        $(\theta=0, \phi=0; \theta=\pi, \phi=0)$

      \item \textbf{Hadamard}: $\{|+\rangle, |-\rangle\}$ where\\
        $|+\rangle = \frac{1}{\sqrt{2}}(|0\rangle + |1\rangle)$\\
        $|-\rangle = \frac{1}{\sqrt{2}}(|0\rangle - |1\rangle)$\\
        $(\theta=\frac{\pi}{2}, \phi=0; \theta=\frac{\pi}{2}, \phi=\pi)$

      \item \textbf{Phase}: $\{|+i\rangle, |-i\rangle\}$ where\\
        $|+i\rangle = \frac{1}{\sqrt{2}}(|0\rangle + i|1\rangle)$\\
        $|-i\rangle = \frac{1}{\sqrt{2}}(|0\rangle - i|1\rangle)$\\
        $(\theta=\frac{\pi}{2}, \phi=\frac{\pi}{2}; \theta=\frac{\pi}{2},
        \phi=-\frac{\pi}{2})$
    \end{itemize}

    \section*{Bloch Sphere}
    \begin{itemize}[leftmargin=*,nosep,topsep=0pt]
      \item \textbf{Equation}: $|\psi\rangle = \cos\frac{\theta}{2}|0\rangle +
        e^{i\phi}\sin\frac{\theta}{2}|1\rangle$\\
        where $\theta \in [0,\pi]$, $\phi \in [0,2\pi)$

      \item \textbf{Cartesian Projection}:
        \begin{align*}
          x &= \sin\theta\cos\phi\\
          y &= \sin\theta\sin\phi\\
          z &= \cos\theta
        \end{align*}

    \end{itemize}

    \section*{Quantum Measurement}
    \begin{itemize}[leftmargin=*,nosep,topsep=0pt]
      \item \textbf{Probability}: For measuring state $|\psi\rangle$ in basis
        $|b\rangle$:
        \begin{align*}
          P(b) = |\langle b|\psi\rangle|^2
        \end{align*}

      \item \textbf{Post-measurement state}:
        \begin{align*}
          |\psi_{\text{new}}\rangle = \frac{|b\rangle\langle
          b|\psi\rangle}{\sqrt{P(b)}}
        \end{align*}

      \item For $|\psi\rangle = \alpha|0\rangle + \beta|1\rangle$:
        \begin{align*}
          P(0) = |\alpha|^2, \quad P(1) = |\beta|^2
        \end{align*}

    \end{itemize}

    \section*{Single-Qubit Gates}
    \begin{itemize}[leftmargin=*,nosep,topsep=0pt]
      \item \textbf{Properties}:
        \begin{itemize}[nosep]
          \item Reversible: $U^\dagger U = UU^\dagger = I$

          \item Preserve norm: $\|U|\psi\rangle\| = \||\psi\rangle\|$

          \item Linear: $U(\alpha|\psi\rangle + \beta|\phi\rangle) = \alpha
            U|\psi\rangle + \beta U|\phi\rangle$

        \end{itemize}
    \end{itemize}

    \section*{Pauli Gates}
    \begin{itemize}[leftmargin=*,nosep,topsep=0pt]
      \item \textbf{Pauli-X (NOT)}: $X = \begin{bmatrix} 0 & 1 \\ 1 & 0
        \end{bmatrix}$
        \begin{itemize}[nosep]
          \item $X|0\rangle = |1\rangle$, $X|1\rangle = |0\rangle$
          \item $X|+\rangle = |+\rangle$, $X|-\rangle = -|-\rangle$
        \end{itemize}

        \item \textbf{Pauli-Y}: $Y = \begin{bmatrix} 0 & -i \\ i & 0 \end{bmatrix}$
          \begin{itemize}[nosep]
            \item $Y|0\rangle = i|1\rangle$, $Y|1\rangle = -i|0\rangle$
          \end{itemize}

        \item \textbf{Pauli-Z (Phase Flip)}: $Z = \begin{bmatrix} 1 & 0 \\ 0 & -1
          \end{bmatrix}$
          \begin{itemize}[nosep]
            \item $Z|0\rangle = |0\rangle$, $Z|1\rangle = -|1\rangle$
            \item $Z|+\rangle = |-\rangle$, $Z|-\rangle = |+\rangle$
          \end{itemize}
        \end{itemize}

        \section*{Rotation Gates}
        \begin{itemize}[leftmargin=*,nosep,topsep=0pt]
          \item \textbf{X-rotation}: $R_X(\theta) = e^{-i\theta X/2} =
            \begin{bmatrix} \cos\frac{\theta}{2} & -i\sin\frac{\theta}{2} \\
            -i\sin\frac{\theta}{2} & \cos\frac{\theta}{2} \end{bmatrix}$

          \item \textbf{Y-rotation}: $R_Y(\theta) = e^{-i\theta Y/2} =
              \begin{bmatrix} \cos\frac{\theta}{2} & -\sin\frac{\theta}{2} \\
              \sin\frac{\theta}{2} & \cos\frac{\theta}{2} \end{bmatrix}$

            \item \textbf{Z-rotation}: $R_Z(\theta) = e^{-i\theta Z/2} =
                \begin{bmatrix} e^{-i\theta/2} & 0 \\ 0 & e^{i\theta/2} \end{bmatrix}$
            \end{itemize}

            \section*{Other Important Gates}
            \begin{itemize}[leftmargin=*,nosep,topsep=0pt]
              \item \textbf{Hadamard}: $H = \frac{1}{\sqrt{2}}\begin{bmatrix} 1 & 1 \\
                1 & -1 \end{bmatrix}$
                \begin{itemize}[nosep]
                  \item $H|0\rangle = |+\rangle$, $H|1\rangle = |-\rangle$

                  \item $H|+\rangle = |0\rangle$, $H|-\rangle = |1\rangle$

                  \item $H^2 = I$
                \end{itemize}

                \item \textbf{Phase (S)}: $S = \begin{bmatrix} 1 & 0 \\ 0 & i \end{bmatrix}$
                  \begin{itemize}[nosep]
                    \item $S|+\rangle = |+i\rangle = \frac{1}{\sqrt{2}}(|0\rangle +
                      i|1\rangle)$

                    \item $S^2 = Z$
                  \end{itemize}

                \item \textbf{T Gate}: $T = \begin{bmatrix} 1 & 0 \\ 0 & e^{i\pi/4}
                  \end{bmatrix}$
                  \begin{itemize}[nosep]
                    \item $T|+\rangle = \frac{1}{\sqrt{2}}(|0\rangle + e^{i\pi/4}|1\rangle)$
                    \item $T^2 = S$, $T^4 = Z$
                  \end{itemize}

                  \item \textbf{General Phase}: $P(\theta) = \begin{bmatrix} 1 & 0 \\ 0 &
                    e^{i\theta} \end{bmatrix}$
                    \begin{itemize}[nosep]
                      \item $S = P(\pi/2)$, $T = P(\pi/4)$, $Z = P(\pi)$
                    \end{itemize}
                  \end{itemize}

                  \section*{Circuit Notation}
                  \begin{itemize}[leftmargin=*,nosep,topsep=0pt]
                    \item \textbf{Elements}:
                      \begin{itemize}[nosep]
                        \item Qubit: Horizontal line
                        \item Gates: Boxes with labels
                        \item Measurement: Meter symbol (if available)
                        \item Time: Left $\to$ Right
                        \item Controlled operations: Vertical line with dot
                        \item Control on $|1\rangle$: Filled dot $\bullet$
                        \item Control on $|0\rangle$: Empty dot $\circ$
                        \item Endiannes: Top (MSB) $\to$ Bottom (LSB)
                      \end{itemize}
                  \end{itemize}
                  \conceptbox{Important}{Matrix order in circuit is opposite to
                    mathematical notation:

                    $U_1 U_2$ in circuit $=$ $U_2 U_1$ in matrix form
                  }

                  \section*{Multi-Qubit Gates}
                  \begin{itemize}[leftmargin=*,nosep,topsep=0pt]
                    \item \textbf{CNOT} (Controlled-NOT):
                      \begin{align*}
                        \text{CNOT} = \begin{bmatrix}
                          1 & 0 & 0 & 0 \\
                          0 & 1 & 0 & 0 \\
                          0 & 0 & 0 & 1 \\
                          0 & 0 & 1 & 0
                        \end{bmatrix}
                      \end{align*}
                      \begin{itemize}[nosep]
                        \item Effect: $|c,t\rangle \rightarrow |c, t \oplus c\rangle$
                        \item $|00\rangle \to |00\rangle$, $|01\rangle \to |01\rangle$
                        \item $|10\rangle \to |11\rangle$, $|11\rangle \to |10\rangle$
                      \end{itemize}

                    \item \textbf{Controlled-Z}:
                      \begin{align*}
                        \text{CZ} = \begin{bmatrix}
                          1 & 0 & 0 & 0 \\
                          0 & 1 & 0 & 0 \\
                          0 & 0 & 1 & 0 \\
                          0 & 0 & 0 & -1
                        \end{bmatrix}
                      \end{align*}
                      \begin{itemize}[nosep]
                        \item Effect: $|x,y\rangle \to (-1)^{xy}|x,y\rangle$
                        \item Only changes $|11\rangle \to -|11\rangle$
                        \item Symmetric: Either qubit can be control
                      \end{itemize}

                    \item \textbf{Toffoli} (CCNOT):
                      \begin{itemize}[nosep]
                        \item Effect: $|x,y,z\rangle \to |x,y,z \oplus (x \cdot y)\rangle$
                        \item Flips target only if both controls are $|1\rangle$
                      \end{itemize}

                    \item \textbf{SWAP}:
                      \begin{align*}
                        \text{SWAP} = \begin{bmatrix}
                          1 & 0 & 0 & 0 \\
                          0 & 0 & 1 & 0 \\
                          0 & 1 & 0 & 0 \\
                          0 & 0 & 0 & 1
                        \end{bmatrix}
                      \end{align*}
                      \begin{itemize}[nosep]
                        \item Effect: $|a,b\rangle \to |b,a\rangle$

                        \item Implementation: $\text{CNOT}_{1,2} \cdot \text{CNOT}_{2,1}
                          \cdot \text{CNOT}_{1,2}$
                      \end{itemize}

                    \item \textbf{Flipped CNOT}:
                      \begin{align*}
                        (H \otimes H) \cdot \text{CNOT} \cdot (H \otimes H) = \begin{bmatrix}
                          1 & 0 & 0 & 0 \\
                          0 & 0 & 0 & 1 \\
                          0 & 0 & 1 & 0 \\
                          0 & 1 & 0 & 0
                        \end{bmatrix}
                      \end{align*}
                  \end{itemize}

                  \section*{Key Quantum Properties}
                  \begin{itemize}[leftmargin=*,nosep,topsep=0pt]
                    \item \textbf{Reversibility}: All quantum gates are reversible
                      \begin{itemize}[nosep]
                        \item To reverse a circuit, apply $U^\dagger$ gates in reverse order
                      \end{itemize}

                    \item \textbf{No-Cloning Theorem}: Cannot create an identical copy of an
                      unknown quantum state
                      \begin{align*}
                        \nexists U: U(|\psi\rangle \otimes |0\rangle) = |\psi\rangle \otimes
                        |\psi\rangle
                      \end{align*}

                    \item \textbf{Entanglement}: States that cannot be factored as tensor
                      products of individual states
                      \begin{itemize}[nosep]
                        \item E.g., Bell state: $\frac{1}{\sqrt{2}}(|00\rangle + |11\rangle)$
                      \end{itemize}
                  \end{itemize}

                  \section*{Bell Pair}
                  \begin{itemize}[leftmargin=*,nosep,topsep=0pt]
                    \item \textbf{Bell States} (maximally entangled):
                      \begin{align*}
                        |\Phi^+\rangle &= \frac{1}{\sqrt{2}}(|00\rangle + |11\rangle)\\
                        |\Phi^-\rangle &= \frac{1}{\sqrt{2}}(|00\rangle - |11\rangle)\\
                        |\Psi^+\rangle &= \frac{1}{\sqrt{2}}(|01\rangle + |10\rangle)\\
                        |\Psi^-\rangle &= \frac{1}{\sqrt{2}}(|01\rangle - |10\rangle)
                      \end{align*}

                    \item Bell-Pair Circuit
                      \[
                        \begin{quantikz}
                          q_1 \quad \lstick{$\zero$} & \gate{H} & \ctrl{1} & \meter{} \\
                          q_0 \quad \lstick{$\zero$} & \qw & \gate{X} & \qw \\
                        \end{quantikz}
                      \]
                  \end{itemize}

                  \section*{Grover's Algorithm}
                  \begin{itemize}[leftmargin=*,nosep,topsep=0pt]
                    \item \textbf{Purpose}: Search unstructured database of $N$ items in
                      $O(\sqrt{N})$ time

                    \item \textbf{Algorithm}:
                      \begin{enumerate}[nosep]
                        \item Initialize: $|s\rangle = H^{\otimes n}|0\rangle^{\otimes n}$
                        \item Repeat $O(\sqrt{N})$ times:
                          \begin{itemize}[nosep]
                            \item Oracle ($O$): $O|x\rangle = (-1)^{f(x)}|x\rangle$ where
                              $f(x)=1$ for solution

                            \item Diffusion ($D$): $D = 2|s\rangle\langle s| - I$
                          \end{itemize}

                        \item Measure to find solution with high probability
                      \end{enumerate}

                      \conceptbox{Important Insight}{Grover's algorithm
                        provides quadratic speedup over classical search, which
                      is proven to be optimal for quantum algorithms}

                    \item \textbf{Circuit}
                      \[
                        \resizebox{0.25\textwidth}{!}{
                          \begin{quantikz}
                            \lstick{$q_{n - 1}$} & \gate{H} & \gate[5]{\shortstack{$f(x)$ \\ Oracle}} & \qw & \gate[5]{\text{Diffusion Circuit}} & \meter{} \\
                            \lstick{$q_{n - 2}$} & \gate{H} & & & & \meter{} \\
                            \vdots & \vdots & & \vdots & & \vdots \\
                            \lstick{$q_1$} & \gate{H} & & & & \meter{} \\
                            \lstick{$q_0$} & \gate{H} & & & & \meter{}
                          \end{quantikz}
                        }
                      \]

                      \[
                        \hspace{1cm}
                        \underbrace{\hspace{3cm}}_{\text{Repeat } \sqrt{2^n} \text{ times}}
                      \]

                    \item \textbf{General Diffusion Circuit}: Layer of H,
                      then layer of X, H on LSB, then Controlled-X layer over
                      all qubits, another H on LSB, then layer of X, and then
                      layer of H

                    \item \textbf{phase kickback}: when a controlled
                      operation transfers a phase from the target qubit to
                      the control qubit(s)
                  \end{itemize}

                  \section*{Quantum Implementations of Classical Logic Gates}
                  \begin{itemize}[leftmargin=*,nosep,topsep=0pt]
                    \item \textbf{OR Gate}: Computes $c = a \lor b$
                      \begin{align*}
                        \begin{quantikz}
                          \lstick{$|a\rangle$} & \ctrl{1} & \ctrl{2} & \qw \\
                          \lstick{$|b\rangle$} & \ctrl{1} & \targ{} & \qw \\
                          \lstick{$|c\rangle$} & \targ{} & \targ{} & \qw
                        \end{quantikz}
                      \end{align*}
                      \begin{itemize}[nosep]
                        \item Effect: $|a,b,c\rangle \to |a,b,c \oplus (a \lor b)\rangle$
                        \item Truth Table: $|000\rangle \to |000\rangle$, $|010\rangle \to |011\rangle$
                        \item $|100\rangle \to |101\rangle$, $|110\rangle \to |111\rangle$
                      \end{itemize}

                    \item \textbf{NOR Gate}: Computes $c = \lnot(a \lor b)$
                      \begin{align*}
                        \begin{quantikz}
                          \lstick{$|a\rangle$} & \ctrl{1} & \ctrl{2} & \qw \\
                          \lstick{$|b\rangle$} & \ctrl{1} & \targ{} & \qw \\
                          \lstick{$|c\rangle$} & \targ{} & \targ{} & \gate{X} & \qw
                        \end{quantikz}
                      \end{align*}
                      \begin{itemize}[nosep]
                        \item Effect: $|a,b,c\rangle \to |a,b,c \oplus \lnot(a \lor b)\rangle$
                        \item Truth Table: $|000\rangle \to |001\rangle$, $|010\rangle \to |010\rangle$
                        \item $|100\rangle \to |100\rangle$, $|110\rangle \to |110\rangle$
                      \end{itemize}

                    \item \textbf{AND Gate}: Computes $c = a \land b$
                      \begin{align*}
                        \begin{quantikz}
                          \lstick{$|a\rangle$} & \ctrl{1} & \qw \\
                          \lstick{$|b\rangle$} & \ctrl{1} & \qw \\
                          \lstick{$|c\rangle$} & \targ{} & \qw
                        \end{quantikz}
                      \end{align*}
                      \begin{itemize}[nosep]
                        \item Effect: $|a,b,c\rangle \to |a,b,c \oplus (a \land b)\rangle$
                        \item Truth Table: $|000\rangle \to |000\rangle$, $|010\rangle \to |010\rangle$
                        \item $|100\rangle \to |100\rangle$, $|110\rangle \to |111\rangle$
                        \item Direct implementation using Toffoli gate
                      \end{itemize}

                    \item \textbf{NAND Gate}: Computes $c = \lnot(a \land b)$
                      \begin{align*}
                        \begin{quantikz}
                          \lstick{$|a\rangle$} & \ctrl{1} & \qw \\
                          \lstick{$|b\rangle$} & \ctrl{1} & \qw \\
                          \lstick{$|c\rangle$} & \targ{} & \gate{X} & \qw
                        \end{quantikz}
                      \end{align*}
                      \begin{itemize}[nosep]
                        \item Effect: $|a,b,c\rangle \to |a,b,c \oplus \lnot(a \land b)\rangle$
                        \item Truth Table: $|000\rangle \to |001\rangle$, $|010\rangle \to |011\rangle$
                        \item $|100\rangle \to |101\rangle$, $|110\rangle \to |110\rangle$
                      \end{itemize}

                    \item \textbf{XOR Gate}: Computes $c = a \oplus b$
                      \begin{align*}
                        \begin{quantikz}
                          \lstick{$|a\rangle$} & \ctrl{1} & \qw \\
                          \lstick{$|b\rangle$} & \targ{} & \qw
                        \end{quantikz}
                      \end{align*}
                      \begin{itemize}[nosep]
                        \item For 3-qubit implementation:
                          \begin{align*}
                            \begin{quantikz}
                              \lstick{$|a\rangle$} & \ctrl{2} & \qw \\
                              \lstick{$|b\rangle$} & \qw & \ctrl{1} & \qw \\
                              \lstick{$|c\rangle$} & \targ{} & \targ{} & \qw
                            \end{quantikz}
                          \end{align*}
                        \item Effect: $|a,b,c\rangle \to |a,b,c \oplus (a \oplus b)\rangle$
                        \item Truth Table: $|000\rangle \to |000\rangle$, $|010\rangle \to |011\rangle$
                        \item $|100\rangle \to |101\rangle$, $|110\rangle \to |110\rangle$
                      \end{itemize}

                    \item \textbf{XNOR Gate}: Computes $c = \lnot(a \oplus b)$
                      \begin{align*}
                        \begin{quantikz}
                          \lstick{$|a\rangle$} & \ctrl{2} & \qw \\
                          \lstick{$|b\rangle$} & \qw & \ctrl{1} & \qw \\
                          \lstick{$|c\rangle$} & \targ{} & \targ{} & \gate{X} & \qw
                        \end{quantikz}
                      \end{align*}

                      \begin{itemize}[nosep]
                        \item Effect: $|a,b,c\rangle \to |a,b,c \oplus \lnot(a \oplus b)\rangle$

                        \item Truth Table: $|000\rangle \to |001\rangle$, $|010\rangle \to |010\rangle$

                        \item $|100\rangle \to |100\rangle$, $|110\rangle \to |111\rangle$

                      \end{itemize}
                  \end{itemize}

                  \conceptbox{Important Note}{The output qubit $|c\rangle$
                    should be initialized to $|0\rangle$ for classical logic
                    gate operation. The circuits preserve input states
                    ($|a\rangle$ and $|b\rangle$) allowing for reversible
                  computation.}

                  \section*{CNF and ANF for SAT}
                  \begin{itemize}[leftmargin=*,nosep,topsep=0pt]
                    \item \textbf{CNF (Conjunctive Normal Form)}:
                      \begin{itemize}[nosep]
                        \item Form: $\bigwedge_i (\lor_j l_{ij})$, $l_{ij} =
                          x_k$ or $\neg x_k$ (literals)

                        \item E.g., $(\neg A \lor \neg C) \land (J \lor P)$

                        \item Quantum: OR gates per clause, multi-controlled
                          $X$ for AND, $O(\text{\# clauses})$ ancilla

                        \item Circuit: \[
                            \begin{quantikz}
                              \lstick{$x_1$} & \ctrl{2} & \qw \\
                              \lstick{$x_2$} & \ctrl{1} & \qw \\
                              \lstick{$a_1$} & \targ{} & \ctrl{2} \\
                              \lstick{$x_3$} & \ctrl{1} & \qw \\
                              \lstick{$a_2$} & \targ{} & \targ{}
                            \end{quantikz}
                          \]
                      \end{itemize}

                    \item \textbf{ANF (Algebraic Normal Form)}:
                      \begin{itemize}[nosep]
                        \item Form: XOR of AND terms, e.g., $f = J \oplus C J$

                        \item Conversion: From CNF via truth table or Boolean
                          algebra

                        \item Quantum: CNOT/Toffoli for XOR/AND, $O(1)$
                          ancilla (typically 1)

                        \item Circuit: \[
                            \begin{quantikz}
                              \lstick{$x_1$} & \qw & \qw \\
                              \lstick{$x_2$} & \ctrl{2} & \qw \\
                              \lstick{$x_3$} & \ctrl{1} & \qw \\
                              \lstick{$a$} & \targ{} & \targ{}
                            \end{quantikz}
                          \]
                      \end{itemize}

                    \item \textbf{Comparison}:
                      \begin{itemize}[nosep]
                        \item CNF: Intuitive, scales with clauses, higher depth
                        \item ANF: Compact, typically fewer ancilla, lower
                          depth, needs conversion
                      \end{itemize}
                  \end{itemize}

                  \conceptbox{ANF Advantage}{Can reduce ancilla qubits,
                  simplifies oracle for Grover’s SAT}

                  \section*{Quantum Approximate Optimization Algorithm (QAOA)}
                  \begin{itemize}[leftmargin=*,nosep,topsep=0pt]
                    \item \textbf{Purpose}: Approximate solutions to
                      combinatorial optimization problems (e.g., \textsc{Max-Cut})
                      using a hybrid quantum-classical approach

                    \item \textbf{Key Components}:
                      \begin{itemize}[nosep]
                        \item \textit{Cost Hamiltonian} ($H_C$): Encodes
                          problem, e.g., $H_C = \sum_{\langle i,j \rangle}
                          Z_i Z_j$ for \textsc{Max-Cut}

                        \item \textit{Mixing Hamiltonian} ($H_M$): Drives
                          exploration, typically $H_M = \sum_i X_i$

                        \item \textit{Ansatz}: $\ket{\psi(\gamma, \beta)} =
                          \prod_{p=1}^P e^{-i\beta_p H_M} e^{-i\gamma_p H_C}
                          \ket{s}$, where $\ket{s} = H^{\otimes n}
                          \ket{0}^{\otimes n}$

                        \item \textit{Goal}: Minimize $\langle \psi(\gamma,
                          \beta) | H_C | \psi(\gamma, \beta) \rangle$
                      \end{itemize}

                    \item \textbf{Steps}:
                      \begin{enumerate}[nosep]
                        \item Initialize: $\ket{s} = H^{\otimes n}
                          \ket{0}^{\otimes n}$ (e.g., $n=10$ for `$10
                          \mathrm{~N} \rightarrow \mathrm{H}$`)

                        \item Apply $P$ layers of $e^{-i\gamma_p H_C}$ and
                          $e^{-i\beta_p H_M}$

                        \item Measure $\langle H_C \rangle$

                        \item Optimize $\gamma_p, \beta_p$ classically

                        \item Repeat until convergence
                      \end{enumerate}

                    \item \textbf{\textsc{Max-Cut} Circuit ($P=1$)}:
                      \[
                        \resizebox{0.25\textwidth}{!}{
                          \begin{quantikz}
                            \lstick{$q_0$} & \gate{H} & \gate{e^{-i\gamma Z Z}} & \gate{e^{-i\gamma Z Z}} & \gate{R_X(2\beta)} & \meter{} \\
                            \lstick{$q_1$} & \gate{H} & \gate[2]{Z Z} & \qw & \gate{R_X(2\beta)} & \meter{} \\
                            \lstick{$q_2$} & \gate{H} & \qw & \gate[2]{Z Z} & \gate{R_X(2\beta)} & \meter{} \\
                            \lstick{$q_3$} & \gate{H} & \qw & \qw & \gate{R_X(2\beta)} & \meter{}
                          \end{quantikz}
                        }
                      \]

                    \item \textbf{Gate Implementation}:
                      \begin{itemize}[nosep]
                        \item $e^{-i\gamma Z_i Z_j}$: CNOT, $R_Z(2\gamma)$, CNOT
                        \item $e^{-i\beta X_i}$: $R_X(2\beta)$
                      \end{itemize}

                    \item \textbf{Properties}:
                      \begin{itemize}[nosep]
                        \item \textit{Advantages}: NISQ-friendly (small $P$),
                          scales to exact solution as $P \to \infty$
                        \item \textit{Limitations}: Barren plateaus, unproven
                          speedup
                      \end{itemize}
                  \end{itemize}

                  \conceptbox{QAOA Insight}{Hybrid approach leverages quantum
                    circuits for state preparation and classical optimization
                    for parameter tuning, balancing NISQ constraints with
                  computational power}

                  \section*{Variational Quantum Algorithms (VQA)}
                  \begin{itemize}[leftmargin=*,nosep,topsep=0pt]
                    \item \textbf{Overview}: Hybrid quantum–classical
                      algorithms that iteratively optimize a parameterized
                      quantum circuit.
                    \item \textbf{Process}:
                      \begin{enumerate}[nosep]
                        \item \textit{State Preparation}: Create an ansatz
                          $\ket{\psi(\theta)}$.
                        \item \textit{Measurement}: Evaluate a cost function
                          $C(\theta)$ based on quantum measurement outcomes.
                        \item \textit{Optimization}: Adjust the parameters
                          $\theta$ using a classical optimizer.
                      \end{enumerate}
                    \item \textbf{Example} (Bell State Preparation):
                      \[
                        \ket{\psi(\theta)} = \mathrm{CNOT}\Bigl(R_y(\theta)\ket{0}\otimes\ket{0}\Bigr)
                      \]
                      with cost function
                      \[
                        C(\theta)=1-|\langle\Phi^+|\psi(\theta)\rangle|^2,
                      \]
                      where $\ket{\Phi^+}=\frac{1}{\sqrt{2}}(\ket{00}+\ket{11})$.
                  \end{itemize}

                  \section*{Advanced QAOA Concepts}
                  \begin{itemize}[leftmargin=*,nosep,topsep=0pt]
                    \item \textbf{Cost Hamiltonian} ($H_C$):
                      \begin{itemize}[nosep]
                        \item Encodes the optimization objective. For any computational basis state \(\ket{x}\), the operator satisfies
                          \[
                          H_C \ket{x} = C(x) \ket{x},
                          \]
                          where \(C(x)\) is the classical cost evaluated on bitstring \(x\).
                        \item \textbf{Max-Cut Example}: The cost function is defined as
                          \[
                          C = \frac{1}{2} \sum_{(i,j) \in E} \Bigl(I - Z_i Z_j\Bigr),
                          \]
                          which assigns a cost based on the number of edges cut by a partition.
                      \end{itemize}
                    \item \textbf{Mixing Hamiltonian} ($H_M$):
                      \begin{itemize}[nosep]
                        \item Drives exploration of the solution space; standard form:
                          \[
                            H_M = \sum_i X_i.
                          \]
                        \item Advanced variants (e.g., XY mixers) can preserve additional problem constraints.
                      \end{itemize}
                    \item \textbf{Circuit Structure}:
                      \begin{enumerate}[nosep]
                        \item \textit{Initialization}: Prepare the uniform superposition via \(\ket{s}=H^{\otimes n}\ket{0}^{\otimes n}\).
                        \item \textit{Alternating Layers}: For each layer \(p=1,\dots,P\) apply:
                          \begin{itemize}[nosep]
                            \item Cost evolution: \(e^{-i\gamma_p H_C}\).
                            \item Mixer evolution: \(e^{-i\beta_p H_M}\).
                          \end{itemize}
                        \item \textit{Measurement}: Read out in the computational basis.
                      \end{enumerate}
                    \item \textbf{Parameter Optimization}:
                      \begin{itemize}[nosep]
                        \item Classical methods (e.g., COBYLA, BFGS) adjust \(\{\gamma_p,\beta_p\}\).
                        \item Initialization strategies include random starts, linear ramps, or layer-by-layer training.
                      \end{itemize}
                    \item \textbf{Scaling \& Limitations}:
                      \begin{itemize}[nosep]
                        \item Increasing \(P\) can improve approximation quality (approaching adiabatic dynamics) but raises challenges such as barren plateaus and hardware noise.
                      \end{itemize}
                  \end{itemize}
                  \conceptbox{QAOA Advanced Insight}{Optimal parameter tuning in QAOA balances circuit depth with performance, leveraging both quantum state preparation and classical optimization.}

                  \section*{Cirq Implementation of QAOA (Max-Cut)}
                  \begin{itemize}[leftmargin=*,nosep,topsep=0pt]
                    \item \textbf{Circuit Steps}:
                      \begin{enumerate}[nosep]
                        \item \textit{Initialization}: Apply Hadamard gates to all qubits to create \(\ket{s}\).
                        \item \textit{Cost Evolution}: For each edge \((i,j)\), implement
                            $e^{-i\gamma Z_i Z_j}$ using a CNOT, $R_Z(2\gamma)$, and a CNOT.
                        \item \textit{Mixing Evolution}: Apply \(R_X(2\beta)\) to each qubit.
                        \item \textit{Measurement}: Measure all qubits to estimate the cut value.
                      \end{enumerate}
                    \item \textbf{Optimization}: Utilize a classical optimizer (e.g., COBYLA) to tune the parameters by minimizing the negative cut value.
                  \end{itemize}
                  \conceptbox{Cirq QAOA Insight}{Efficient circuit implementations leverage native gate sets to balance circuit depth and operational fidelity on NISQ hardware.}
                \end{multicols}
                \endgroup
