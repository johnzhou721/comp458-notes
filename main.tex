\documentclass{report}

% ==============================

% Configuration files for custom packages and macros
%%%%%%%%%%%%%%%%%%%%%%%%%%%%%%%%
% PACKAGE IMPORTS
%%%%%%%%%%%%%%%%%%%%%%%%%%%%%%%%%
\usepackage[tmargin=2cm,rmargin=1in,lmargin=1in,margin=0.85in,bmargin=2cm,footskip=.2in]{geometry}
\usepackage{amsmath,amsfonts,amsthm,amssymb,mathtools}
\usepackage[varbb]{newpxmath}
\usepackage{xfrac}
\usepackage[makeroom]{cancel}
\usepackage{mathtools}
\usepackage{bookmark}
\usepackage{appendix}
\usepackage{imakeidx}
\usepackage{enumitem}
\usepackage[most,many,breakable]{tcolorbox}
\usepackage{xcolor}
\usepackage{varwidth}
\usepackage{varwidth}
\usepackage{etoolbox}
\usepackage{nameref}
\usepackage{multicol,array}
\usepackage{quantikz}
\usepackage{circuitikz}
\usepackage{tikz-cd}
\usepackage{tikz-3dplot}
\usetikzlibrary{arrows.meta}
\usepackage[ruled,vlined,linesnumbered]{algorithm2e}
\usepackage{comment} % enables the use of multi-line comments (\ifx \fi)
\usepackage{import}
\usepackage{xifthen}
\usepackage{pdfpages}
\usepackage{transparent}
\usepackage{hyperref,theoremref}
\hypersetup{
	pdftitle={COMP 458 Quantum Computing Handbook},
  pdfauthor={Micah Kepe},
  hyperindex=true,
	colorlinks=true, linkcolor=doc!90,
	bookmarksnumbered=true,
	bookmarksopen=true
}

\usepackage{fancyhdr}  % Package for custom headers/footers
\pagestyle{fancy}

% Define the header style
\fancyhf{}  % Clear default header/footer settings

% Define headers for odd and even pages
\fancyhead[LE]{\textbf{\thepage}}         % Left side (Even pages) - Page number
\fancyhead[RE]{\textbf{\rightmark}}       % Right side (Even pages) - Section name
\fancyhead[LO]{\textbf{\leftmark}}        % Left side (Odd pages) - Chapter name
\fancyhead[RO]{\textbf{\thepage}}         % Right side (Odd pages) - Page number

% Remove header/footer on the first page of each chapter
\fancypagestyle{plain}{%
    \fancyhf{}  % Clear header/footer for first page of chapter
    \renewcommand{\headrulewidth}{0pt}  % Remove header rule
    \renewcommand{\footrulewidth}{0pt}  % Remove footer rule
}

% Ensure that LaTeX recognizes chapter and section names for headers
\renewcommand{\chaptermark}[1]{\markboth{\textbf{\thechapter. #1 }}{}}
\renewcommand{\sectionmark}[1]{\markright{\textbf{\thesection. #1 }}}

% Adjust header rule thickness
\renewcommand{\headrulewidth}{0.4pt} % Line under header

% Ensure footer is empty
\renewcommand{\footrulewidth}{0pt}

\newcommand\mycommfont[1]{\footnotesize\ttfamily\textcolor{blue}{#1}}
\SetCommentSty{mycommfont}
\newcommand{\incfig}[1]{%
    \def\svgwidth{\columnwidth}
    \import{./figures/}{#1.pdf_tex}
}

\usepackage{tikzsymbols}
\renewcommand\qedsymbol{$\blacksquare$}



% Minted Package Configuration %%%%%%%%%%%%%%%%%%%%%%%%%%%%%%%%%%
\usepackage{minted}
\usemintedstyle{monokai} % More modern and readable style
\setminted{
    linenos=true,              % Enable line numbers
    frame=single,              % Single line frame (cleaner look)
    framesep=3mm,             % Slightly more padding
    fontsize=\small,          % Maintain small font size
    tabsize=4,                % Standard tab width
    breaklines=true,          % Enable line breaking
    breakanywhere=true,       % Break lines anywhere
    bgcolor=gray!10,          % Light gray background
    numberblanklines=false,   % Don't number blank lines
    autogobble=true,          % Remove common leading whitespace
    style=friendly,           % Base style
    python3=true,             % Enforce Python 3 syntax
    xleftmargin=20pt,         % Left margin for line numbers
    linenos=true,             % Show line numbers
    numbersep=8pt,            % Space between numbers and code
    formatcom=\setlength{\baselineskip}{0.8\baselineskip} % Tighter line spacing
}

%%%%%%%%%%%%%%%%%%%%%%%%%%%%%%
% SELF MADE COLORS
%%%%%%%%%%%%%%%%%%%%%%%%%%%%%%
\definecolor{myg}{RGB}{56, 140, 70}
\definecolor{myb}{RGB}{45, 111, 177}
\definecolor{myr}{RGB}{199, 68, 64}
\definecolor{mytheorembg}{HTML}{F2F2F9}
\definecolor{mytheoremfr}{HTML}{00007B}
\definecolor{mylenmabg}{HTML}{FFFAF8}
\definecolor{mylenmafr}{HTML}{983b0f}
\definecolor{mypropbg}{HTML}{f2fbfc}
\definecolor{mypropfr}{HTML}{191971}
\definecolor{myexamplebg}{HTML}{F2FBF8}
\definecolor{myexamplefr}{HTML}{88D6D1}
\definecolor{myexampleti}{HTML}{2A7F7F}
\definecolor{mydefinitbg}{HTML}{E5E5FF}
\definecolor{mydefinitfr}{HTML}{3F3FA3}
\definecolor{notesgreen}{RGB}{0,162,0}
\definecolor{myp}{RGB}{197, 92, 212}
\definecolor{mygr}{HTML}{2C3338}
\definecolor{myred}{RGB}{127,0,0}
\definecolor{myyellow}{RGB}{169,121,69}
\definecolor{myexercisebg}{HTML}{F2FBF8}
\definecolor{myexercisefg}{HTML}{88D6D1}


%%%%%%%%%%%%%%%%%%%%%%%%%%%%
% TCOLORBOX SETUPS
%%%%%%%%%%%%%%%%%%%%%%%%%%%%
\setlength{\parindent}{1cm}

%================================
% THEOREM BOX
%================================

\tcbuselibrary{theorems,skins,hooks}
\newtcbtheorem[number within=section]{Theorem}{Theorem}
{%
	enhanced,
	breakable,
	colback = mytheorembg,
	frame hidden,
	boxrule = 0sp,
	borderline west = {2pt}{0pt}{mytheoremfr},
	sharp corners,
	detach title,
	before upper = \tcbtitle\par\smallskip,
	coltitle = mytheoremfr,
	fonttitle = \bfseries\sffamily,
	description font = \mdseries,
	separator sign none,
	segmentation style={solid, mytheoremfr},
}
{th}

\tcbuselibrary{theorems,skins,hooks}
\newtcbtheorem[number within=chapter]{theorem}{Theorem}
{%
	enhanced,
	breakable,
	colback = mytheorembg,
	frame hidden,
	boxrule = 0sp,
	borderline west = {2pt}{0pt}{mytheoremfr},
	sharp corners,
	detach title,
	before upper = \tcbtitle\par\smallskip,
	coltitle = mytheoremfr,
	fonttitle = \bfseries\sffamily,
	description font = \mdseries,
	separator sign none,
	segmentation style={solid, mytheoremfr},
}
{th}


\tcbuselibrary{theorems,skins,hooks}
\newtcolorbox{Theoremcon}
{%
	enhanced
	,breakable
	,colback = mytheorembg
	,frame hidden
	,boxrule = 0sp
	,borderline west = {2pt}{0pt}{mytheoremfr}
	,sharp corners
	,description font = \mdseries
	,separator sign none
}

%================================
% Corollery
%================================
\tcbuselibrary{theorems,skins,hooks}
\newtcbtheorem[number within=section]{Corollary}{Corollary}
{%
	enhanced
	,breakable
	,colback = myp!10
	,frame hidden
	,boxrule = 0sp
	,borderline west = {2pt}{0pt}{myp!85!black}
	,sharp corners
	,detach title
	,before upper = \tcbtitle\par\smallskip
	,coltitle = myp!85!black
	,fonttitle = \bfseries\sffamily
	,description font = \mdseries
	,separator sign none
	,segmentation style={solid, myp!85!black}
}
{th}
\tcbuselibrary{theorems,skins,hooks}
\newtcbtheorem[number within=chapter]{corollary}{Corollary}
{%
	enhanced
	,breakable
	,colback = myp!10
	,frame hidden
	,boxrule = 0sp
	,borderline west = {2pt}{0pt}{myp!85!black}
	,sharp corners
	,detach title
	,before upper = \tcbtitle\par\smallskip
	,coltitle = myp!85!black
	,fonttitle = \bfseries\sffamily
	,description font = \mdseries
	,separator sign none
	,segmentation style={solid, myp!85!black}
}
{th}


%================================
% LENMA
%================================

\tcbuselibrary{theorems,skins,hooks}
\newtcbtheorem[number within=section]{Lenma}{Lenma}
{%
	enhanced,
	breakable,
	colback = mylenmabg,
	frame hidden,
	boxrule = 0sp,
	borderline west = {2pt}{0pt}{mylenmafr},
	sharp corners,
	detach title,
	before upper = \tcbtitle\par\smallskip,
	coltitle = mylenmafr,
	fonttitle = \bfseries\sffamily,
	description font = \mdseries,
	separator sign none,
	segmentation style={solid, mylenmafr},
}
{th}

\tcbuselibrary{theorems,skins,hooks}
\newtcbtheorem[number within=chapter]{lenma}{Lenma}
{%
	enhanced,
	breakable,
	colback = mylenmabg,
	frame hidden,
	boxrule = 0sp,
	borderline west = {2pt}{0pt}{mylenmafr},
	sharp corners,
	detach title,
	before upper = \tcbtitle\par\smallskip,
	coltitle = mylenmafr,
	fonttitle = \bfseries\sffamily,
	description font = \mdseries,
	separator sign none,
	segmentation style={solid, mylenmafr},
}
{th}


%================================
% PROPOSITION
%================================

\tcbuselibrary{theorems,skins,hooks}
\newtcbtheorem[number within=section]{Prop}{Proposition}
{%
	enhanced,
	breakable,
	colback = mypropbg,
	frame hidden,
	boxrule = 0sp,
	borderline west = {2pt}{0pt}{mypropfr},
	sharp corners,
	detach title,
	before upper = \tcbtitle\par\smallskip,
	coltitle = mypropfr,
	fonttitle = \bfseries\sffamily,
	description font = \mdseries,
	separator sign none,
	segmentation style={solid, mypropfr},
}
{th}

\tcbuselibrary{theorems,skins,hooks}
\newtcbtheorem[number within=chapter]{prop}{Proposition}
{%
	enhanced,
	breakable,
	colback = mypropbg,
	frame hidden,
	boxrule = 0sp,
	borderline west = {2pt}{0pt}{mypropfr},
	sharp corners,
	detach title,
	before upper = \tcbtitle\par\smallskip,
	coltitle = mypropfr,
	fonttitle = \bfseries\sffamily,
	description font = \mdseries,
	separator sign none,
	segmentation style={solid, mypropfr},
}
{th}


%================================
% CLAIM
%================================

\tcbuselibrary{theorems,skins,hooks}
\newtcbtheorem[number within=section]{claim}{Claim}
{%
	enhanced
	,breakable
	,colback = myg!10
	,frame hidden
	,boxrule = 0sp
	,borderline west = {2pt}{0pt}{myg}
	,sharp corners
	,detach title
	,before upper = \tcbtitle\par\smallskip
	,coltitle = myg!85!black
	,fonttitle = \bfseries\sffamily
	,description font = \mdseries
	,separator sign none
	,segmentation style={solid, myg!85!black}
}
{th}



%================================
% Exercise
%================================

\tcbuselibrary{theorems,skins,hooks}
\newtcbtheorem[number within=section]{Exercise}{Exercise}
{%
	enhanced,
	breakable,
	colback = myexercisebg,
	frame hidden,
	boxrule = 0sp,
	borderline west = {2pt}{0pt}{myexercisefg},
	sharp corners,
	detach title,
	before upper = \tcbtitle\par\smallskip,
	coltitle = myexercisefg,
	fonttitle = \bfseries\sffamily,
	description font = \mdseries,
	separator sign none,
	segmentation style={solid, myexercisefg},
}
{th}

\tcbuselibrary{theorems,skins,hooks}
\newtcbtheorem[number within=chapter]{exercise}{Exercise}
{%
	enhanced,
	breakable,
	colback = myexercisebg,
	frame hidden,
	boxrule = 0sp,
	borderline west = {2pt}{0pt}{myexercisefg},
	sharp corners,
	detach title,
	before upper = \tcbtitle\par\smallskip,
	coltitle = myexercisefg,
	fonttitle = \bfseries\sffamily,
	description font = \mdseries,
	separator sign none,
	segmentation style={solid, myexercisefg},
}
{th}

%================================
% ASIDE BOX
%================================
\newcommand{\aside}[1]{%
    \begin{tcolorbox}[
      enhanced,
      breakable,
      title=Aside,
      colback=gray!10,
      colframe=gray!50,
      sharp corners=southwest]
    #1
    \end{tcolorbox}
}

%================================
% EXAMPLE BOX
%================================

\newtcbtheorem[number within=section]{Example}{Example}
{%
	colback = myexamplebg
	,breakable
	,colframe = myexamplefr
	,coltitle = myexampleti
	,boxrule = 1pt
	,sharp corners
	,detach title
	,before upper=\tcbtitle\par\smallskip
	,fonttitle = \bfseries
	,description font = \mdseries
	,separator sign none
	,description delimiters parenthesis
}
{ex}

\newtcbtheorem[number within=chapter]{example}{Example}
{%
	colback = myexamplebg
	,breakable
	,colframe = myexamplefr
	,coltitle = myexampleti
	,boxrule = 1pt
	,sharp corners
	,detach title
	,before upper=\tcbtitle\par\smallskip
	,fonttitle = \bfseries
	,description font = \mdseries
	,separator sign none
	,description delimiters parenthesis
}
{ex}

%================================
% DEFINITION BOX
%================================

\newtcbtheorem[number within=section]{Definition}{Definition}{enhanced,
	before skip=2mm,after skip=2mm, colback=red!5,colframe=red!80!black,boxrule=0.5mm,
	attach boxed title to top left={xshift=1cm,yshift*=1mm-\tcboxedtitleheight}, varwidth boxed title*=-3cm,
	boxed title style={frame code={
					\path[fill=tcbcolback]
					([yshift=-1mm,xshift=-1mm]frame.north west)
					arc[start angle=0,end angle=180,radius=1mm]
					([yshift=-1mm,xshift=1mm]frame.north east)
					arc[start angle=180,end angle=0,radius=1mm];
					\path[left color=tcbcolback!60!black,right color=tcbcolback!60!black,
						middle color=tcbcolback!80!black]
					([xshift=-2mm]frame.north west) -- ([xshift=2mm]frame.north east)
					[rounded corners=1mm]-- ([xshift=1mm,yshift=-1mm]frame.north east)
					-- (frame.south east) -- (frame.south west)
					-- ([xshift=-1mm,yshift=-1mm]frame.north west)
					[sharp corners]-- cycle;
				},interior engine=empty,
		},
	fonttitle=\bfseries,
	title={#2},#1}{def}
\newtcbtheorem[number within=chapter]{definition}{Definition}{enhanced,
	before skip=2mm,after skip=2mm, colback=red!5,colframe=red!80!black,boxrule=0.5mm,
	attach boxed title to top left={xshift=1cm,yshift*=1mm-\tcboxedtitleheight}, varwidth boxed title*=-3cm,
	boxed title style={frame code={
					\path[fill=tcbcolback]
					([yshift=-1mm,xshift=-1mm]frame.north west)
					arc[start angle=0,end angle=180,radius=1mm]
					([yshift=-1mm,xshift=1mm]frame.north east)
					arc[start angle=180,end angle=0,radius=1mm];
					\path[left color=tcbcolback!60!black,right color=tcbcolback!60!black,
						middle color=tcbcolback!80!black]
					([xshift=-2mm]frame.north west) -- ([xshift=2mm]frame.north east)
					[rounded corners=1mm]-- ([xshift=1mm,yshift=-1mm]frame.north east)
					-- (frame.south east) -- (frame.south west)
					-- ([xshift=-1mm,yshift=-1mm]frame.north west)
					[sharp corners]-- cycle;
				},interior engine=empty,
		},
	fonttitle=\bfseries,
	title={#2},#1}{def}

%================================
% SOLUTION BOX
%================================

\makeatletter
\newtcolorbox{solution}{enhanced,
	breakable,
	colback=white,
	colframe=myg!80!black,
	attach boxed title to top left={yshift*=-\tcboxedtitleheight},
	title=Solution,
	boxed title size=title,
	boxed title style={%
			sharp corners,
			rounded corners=northwest,
			colback=tcbcolframe,
			boxrule=0pt,
		},
	underlay boxed title={%
			\path[fill=tcbcolframe] (title.south west)--(title.south east)
			to[out=0, in=180] ([xshift=5mm]title.east)--
			(title.center-|frame.east)
			[rounded corners=\kvtcb@arc] |-
			(frame.north) -| cycle;
		},
}
\makeatother

%================================
% QUESTION BOX
%================================

\makeatletter
\newtcbtheorem{question}{Question}{enhanced,
	breakable,
	colback=white,
	colframe=mygr,
	attach boxed title to top left={yshift*=-\tcboxedtitleheight},
	fonttitle=\bfseries,
	title={#2},
	boxed title size=title,
	boxed title style={%
			sharp corners,
			rounded corners=northwest,
			colback=tcbcolframe,
			boxrule=0pt,
		},
	underlay boxed title={%
			\path[fill=tcbcolframe] (title.south west)--(title.south east)
			to[out=0, in=180] ([xshift=5mm]title.east)--
			(title.center-|frame.east)
			[rounded corners=\kvtcb@arc] |-
			(frame.north) -| cycle;
		},
	#1
}{def}
\makeatother

%================================
% WRONG CONCEPT BOX
%================================

\newtcbtheorem[number within=chapter]{wconc}{Wrong Concept}{
	breakable,
	enhanced,
	colback=white,
	colframe=myr,
	arc=0pt,
	outer arc=0pt,
	fonttitle=\bfseries\sffamily\large,
	colbacktitle=myr,
	attach boxed title to top left={},
	boxed title style={
			enhanced,
			skin=enhancedfirst jigsaw,
			arc=3pt,
			bottom=0pt,
			interior style={fill=myr}
		},
	#1
}{def}



%================================
% NOTE BOX
%================================

\usetikzlibrary{arrows,calc,shadows.blur}
\tcbuselibrary{skins}
\newtcolorbox{note}[1][]{%
	enhanced jigsaw,
	colback=gray!20!white,%
	colframe=gray!80!black,
	size=small,
	boxrule=1pt,
	title=\textbf{Note:-},
	halign title=flush center,
	coltitle=black,
	breakable,
	drop shadow=black!50!white,
	attach boxed title to top left={xshift=1cm,yshift=-\tcboxedtitleheight/2,yshifttext=-\tcboxedtitleheight/2},
	minipage boxed title=1.5cm,
	boxed title style={%
			colback=white,
			size=fbox,
			boxrule=1pt,
			boxsep=2pt,
			underlay={%
					\coordinate (dotA) at ($(interior.west) + (-0.5pt,0)$);
					\coordinate (dotB) at ($(interior.east) + (0.5pt,0)$);
					\begin{scope}
						\clip (interior.north west) rectangle ([xshift=3ex]interior.east);
						\filldraw [white, blur shadow={shadow opacity=60, shadow yshift=-.75ex}, rounded corners=2pt] (interior.north west) rectangle (interior.south east);
					\end{scope}
					\begin{scope}[gray!80!black]
						\fill (dotA) circle (2pt);
						\fill (dotB) circle (2pt);
					\end{scope}
				},
		},
	#1,
}

%%%%%%%%%%%%%%%%%%%%%%%%%%%%%%
% SELF MADE COMMANDS
%%%%%%%%%%%%%%%%%%%%%%%%%%%%%%


\newcommand{\thm}[2]{\begin{Theorem}{#1}{}#2\end{Theorem}}
\newcommand{\cor}[2]{\begin{Corollary}{#1}{}#2\end{Corollary}}
\newcommand{\mlenma}[2]{\begin{Lenma}{#1}{}#2\end{Lenma}}
\newcommand{\mprop}[2]{\begin{Prop}{#1}{}#2\end{Prop}}
\newcommand{\clm}[3]{\begin{claim}{#1}{#2}#3\end{claim}}
\newcommand{\wc}[2]{\begin{wconc}{#1}{}\setlength{\parindent}{1cm}#2\end{wconc}}
\newcommand{\thmcon}[1]{\begin{Theoremcon}{#1}\end{Theoremcon}}
\newcommand{\ex}[2]{\begin{Example}{#1}{}#2\end{Example}}
\newcommand{\dfn}[2]{\begin{Definition}[colbacktitle=red!75!black]{#1}{}#2\end{Definition}}
\newcommand{\dfnc}[2]{\begin{definition}[colbacktitle=red!75!black]{#1}{}#2\end{definition}}
\newcommand{\qs}[2]{\begin{question}{#1}{}#2\end{question}}
\newcommand{\pf}[2]{\begin{myproof}[#1]#2\end{myproof}}
\newcommand{\nt}[1]{\begin{note}#1\end{note}}

\newcommand*\circled[1]{\tikz[baseline=(char.base)]{
		\node[shape=circle,draw,inner sep=1pt] (char) {#1};}}
\newcommand\getcurrentref[1]{%
	\ifnumequal{\value{#1}}{0}
	{??}
	{\the\value{#1}}%
}
\newcommand{\getCurrentSectionNumber}{\getcurrentref{section}}
\newenvironment{myproof}[1][\proofname]{%
	\proof[\bfseries #1: ]%
}{\endproof}

\newcommand{\mclm}[2]{\begin{myclaim}[#1]#2\end{myclaim}}
\newenvironment{myclaim}[1][\claimname]{\proof[\bfseries #1: ]}{}

\newcounter{mylabelcounter}

\makeatletter
\newcommand{\setword}[2]{%
	\phantomsection
	#1\def\@currentlabel{\unexpanded{#1}}\label{#2}%
}
\makeatother




\tikzset{
	symbol/.style={
			draw=none,
			every to/.append style={
					edge node={node [sloped, allow upside down, auto=false]{$#1$}}}
		}
}


% deliminators
\DeclarePairedDelimiter{\abs}{\lvert}{\rvert}
\DeclarePairedDelimiter{\norm}{\lVert}{\rVert}

\DeclarePairedDelimiter{\ceil}{\lceil}{\rceil}
\DeclarePairedDelimiter{\floor}{\lfloor}{\rfloor}
\DeclarePairedDelimiter{\round}{\lfloor}{\rceil}

\newsavebox\diffdbox
\newcommand{\slantedromand}{{\mathpalette\makesl{d}}}
\newcommand{\makesl}[2]{%
\begingroup
\sbox{\diffdbox}{$\mathsurround=0pt#1\mathrm{#2}$}%
\pdfsave
\pdfsetmatrix{1 0 0.2 1}%
\rlap{\usebox{\diffdbox}}%
\pdfrestore
\hskip\wd\diffdbox
\endgroup
}
\newcommand{\dd}[1][]{\ensuremath{\mathop{}\!\ifstrempty{#1}{%
\slantedromand\@ifnextchar^{\hspace{0.2ex}}{\hspace{0.1ex}}}%
{\slantedromand\hspace{0.2ex}^{#1}}}}
\ProvideDocumentCommand\dv{o m g}{%
  \ensuremath{%
    \IfValueTF{#3}{%
      \IfNoValueTF{#1}{%
        \frac{\dd #2}{\dd #3}%
      }{%
        \frac{\dd^{#1} #2}{\dd #3^{#1}}%
      }%
    }{%
      \IfNoValueTF{#1}{%
        \frac{\dd}{\dd #2}%
      }{%
        \frac{\dd^{#1}}{\dd #2^{#1}}%
      }%
    }%
  }%
}
\providecommand*{\pdv}[3][]{\frac{\partial^{#1}#2}{\partial#3^{#1}}}
%  - others
\DeclareMathOperator{\Lap}{\mathcal{L}}
\DeclareMathOperator{\Var}{Var} % varience
\DeclareMathOperator{\Cov}{Cov} % covarience
\DeclareMathOperator{\E}{E} % expected

% Since the amsthm package isn't loaded

% I prefer the slanted \leq
\let\oldleq\leq % save them in case they're every wanted
\let\oldgeq\geq
\renewcommand{\leq}{\leqslant}
\renewcommand{\geq}{\geqslant}

%%%%%%%%%%%%%%%%%%%%%%%%%%%%%%%%%%%%%%%%%%%
% TABLE OF CONTENTS
%%%%%%%%%%%%%%%%%%%%%%%%%%%%%%%%%%%%%%%%%%%

\usepackage{tikz}
\definecolor{doc}{RGB}{0,60,110}
\usepackage{titletoc}
\contentsmargin{0cm}
\titlecontents{chapter}[3.7pc]
{\addvspace{30pt}%
	\begin{tikzpicture}[remember picture, overlay]%
		\draw[fill=doc!60,draw=doc!60] (-7,-.1) rectangle (-0.9,.5);%
		\pgftext[left,x=-3.5cm,y=0.2cm]{\color{white}\Large\sc\bfseries Chapter\ \thecontentslabel};%
	\end{tikzpicture}\color{doc!60}\large\sc\bfseries}%
{}
{}
{\;\titlerule\;\large\sc\bfseries Page \thecontentspage
	\begin{tikzpicture}[remember picture, overlay]
		\draw[fill=doc!60,draw=doc!60] (2pt,0) rectangle (4,0.1pt);
	\end{tikzpicture}}%
\titlecontents{section}[3.7pc]
{\addvspace{2pt}}
{\contentslabel[\thecontentslabel]{2pc}}
{}
{\hfill\small \thecontentspage}
[]
\titlecontents*{subsection}[3.7pc]
{\addvspace{-1pt}\small}
{}
{}
{\ --- \small\thecontentspage}
[ \textbullet\ ][]

\makeatletter
\renewcommand{\tableofcontents}{%
	\chapter*{%
	  \vspace*{-20\p@}%
	  \begin{tikzpicture}[remember picture, overlay]%
		  \pgftext[right,x=15cm,y=0.2cm]{\color{doc!60}\Huge\sc\bfseries \contentsname};%
		  \draw[fill=doc!60,draw=doc!60] (13,-.75) rectangle (20,1);%
		  \clip (13,-.75) rectangle (20,1);
		  \pgftext[right,x=15cm,y=0.2cm]{\color{white}\Huge\sc\bfseries \contentsname};%
	  \end{tikzpicture}}%
	\@starttoc{toc}}
\makeatother


%%%%%%%%%%%%%%%%%%%%%%%%%%%%%%%%%%%%%%%%%%%
% CODE LISTINGS
%%%%%%%%%%%%%%%%%%%%%%%%%%%%%%%%%%%%%%%%%%%

\lstdefinelanguage{Rust}{
    keywords={
        fn, let, mut, struct, impl, trait, enum, match, if, else, return, break,
        continue, where, for, in, type, move, ref, Option, Result, Some, None, Ok, Err
    },
    sensitive=true,
    comment=[l]{//},
    morecomment=[s]{/*}{*/},
    morestring=[b]",
    morestring=[b]',
    basicstyle=\ttfamily\footnotesize\color{mygr}, % Base text style
    keywordstyle=\bfseries\color{myb},            % Keywords in blue
    stringstyle=\color{myr},                      % Strings in red
    commentstyle=\itshape\color{myg},             % Comments in green
    numbers=left,
    numberstyle=\tiny\color{myyellow},            % Line numbers in yellow-brown
    stepnumber=1,
    numbersep=8pt,
    backgroundcolor=\color{gray!10},              % Light gray background
    showspaces=false,
    showstringspaces=false,
    showtabs=false,
    frame=single,                                 % Box frame around the code
    framesep=5pt,
    rulecolor=\color{mygr},                       % Border color for the frame
    tabsize=4,                                    % Tab spacing
    captionpos=b,
    breaklines=true,
    breakatwhitespace=false,
    title=\lstname,
    escapeinside={(*@}{@*)},                      % Allows escaping to LaTeX
    emph={panic, println, dbg},                  % Highlight specific functions
    emphstyle=\color{myp},                        % Highlighted functions in purple
}


\lstset{language=Rust}

% Description: Custom macros for LaTeX documents

%%%%%%%%%%%%%%%%%%%%%%%%%%%%%%%%%%%%%%%%%%%%%%%%%%%%%%%%
% Custom macros for quantum mechanics
%%%%%%%%%%%%%%%%%%%%%%%%%%%%%%%%%%%%%%%%%%%%%%%%%%%%%%%%

\newcommand{\bloch}[2]{%
    \begin{center}
    \tdplotsetmaincoords{60}{120}
    \begin{tikzpicture}[tdplot_main_coords, scale=2]

        % Sphere with 3D effect and an outer border
        \shade[ball color=white!90,opacity=0.2] (0,0,0) circle (1);

        % Equatorial circle
        \draw[thick] (0,0,0) circle (1);

        % Draw major latitude circles for 3D effect
        \foreach \angle in {0,30,...,150} {
            \tdplotsetrotatedcoords{0}{\angle}{0}
            \draw[tdplot_rotated_coords,black!30] (1,0,0) arc (0:360:1);
        }

        % Main axes with arrows
        \draw[-{Stealth[length=3mm]}, thick, blue] (-1.5,0,0) -- (1.5,0,0) node[below] {\large $x$};
        \draw[-{Stealth[length=3mm]}, thick, blue] (0,-1.5,0) -- (0,1.5,0) node[right] {\large $y$};
        \draw[-{Stealth[length=3mm]}, thick, blue] (0,0,-1.5) -- (0,0,1.5) node[above] {\large $z$};

        % State vector using parameters
        % #1 is theta, #2 is phi (in degrees)
        \draw[red,-{Stealth[length=3mm]}, very thick] (0,0,0) --
            ({sin(#1)*cos(#2)},{sin(#1)*sin(#2)},{cos(#1)});
        \node[red] at
            ({sin(#1)*cos(#2)+0.3},{sin(#1)*sin(#2)+0.2},{cos(#1) + 0.05})
            {\large $|\psi\rangle$};

        % Labels for poles with better visibility
        \fill[black] (0,0,1) circle (0.5pt);
        \node[above=3pt] at (-0.4,0,1.1) {\large $|0\rangle$};

        \fill[black] (0,0,-1) circle (0.5pt);
        \node[below=3pt] at (-0.4,0,-1.1) {\large $|1\rangle$};

        % Additional circular guides to emphasize perspective
        \draw[dashed] (1,0,0) arc (0:180:1 and 0.4);
        \draw (1,0,0) arc (0:-180:1 and 0.4);
    \end{tikzpicture}%
    \end{center}
}

% Math things
\renewcommand{\norm}[1]{\left\lVert#1\right\rVert}
\renewcommand{\abs}[1]{\left\lvert#1\right\rvert}
\newcommand{\inner}[2]{\left\langle#1,#2\right\rangle}
\newcommand{\set}[1]{\left\{#1\right\}}

% Quantum mechanics things
\newcommand{\zero}{\left|0\right\rangle}
\newcommand{\one}{\left|1\right\rangle}
\newcommand{\bra}[1]{\left\langle#1\right|}
\newcommand{\ket}[1]{\left|#1\right\rangle}


%%%%%%%%%%%%%%%%%%%%%%%%%%%%%%%%%%%%%%%%%%%%%%%%%%%%%%%%
% Custom macros I already had
%
% Adapted from:
%   https://github.com/SeniorMars/dotfiles/blob/main/latex_template/macros.tex
%%%%%%%%%%%%%%%%%%%%%%%%%%%%%%%%%%%%%%%%%%%%%%%%%%%%%%%%

%From M275 ``Topology'' at SJSU
\newcommand{\id}{\mathrm{id}}
\newcommand{\taking}[1]{\xrightarrow{#1}}
\newcommand{\inv}{^{-1}}

%From M170 ``Introduction to Graph Theory'' at SJSU
\DeclareMathOperator{\diam}{diam}
\DeclareMathOperator{\ord}{ord}
\newcommand{\defeq}{\overset{\mathrm{def}}{=}}

%From the USAMO .tex files
\newcommand{\ts}{\textsuperscript}
\newcommand{\dg}{^\circ}
\newcommand{\ii}{\item}

% % From Math 55 and Math 145 at Harvard
\newenvironment{subproof}[1][Proof]{%
\begin{proof}[#1] \renewcommand{\qedsymbol}{$\blacksquare$}}%
{\end{proof}}

\newcommand{\liff}{\leftrightarrow}
\newcommand{\lthen}{\rightarrow}
\newcommand{\opname}{\operatorname}
\newcommand{\surjto}{\twoheadrightarrow}
\newcommand{\injto}{\hookrightarrow}
\newcommand{\On}{\mathrm{On}} % ordinals
\DeclareMathOperator{\img}{im} % Image
\DeclareMathOperator{\Img}{Im} % Image
\DeclareMathOperator{\coker}{coker} % Cokernel
\DeclareMathOperator{\Coker}{Coker} % Cokernel
\DeclareMathOperator{\Ker}{Ker} % Kernel
\DeclareMathOperator{\rank}{rank}
\DeclareMathOperator{\Spec}{Spec} % spectrum
\DeclareMathOperator{\Tr}{Tr} % trace
\DeclareMathOperator{\pr}{pr} % projection
\DeclareMathOperator{\ext}{ext} % extension
\DeclareMathOperator{\pred}{pred} % predecessor
\DeclareMathOperator{\dom}{dom} % domain
\DeclareMathOperator{\ran}{ran} % range
\DeclareMathOperator{\Hom}{Hom} % homomorphism
\DeclareMathOperator{\Mor}{Mor} % morphisms
\DeclareMathOperator{\End}{End} % endomorphism

\newcommand{\eps}{\epsilon}
\newcommand{\veps}{\varepsilon}
\newcommand{\ol}{\overline}
\newcommand{\ul}{\underline}
\newcommand{\wt}{\widetilde}
\newcommand{\wh}{\widehat}
\newcommand{\vocab}[1]{\textbf{\color{blue} #1}}
\providecommand{\half}{\frac{1}{2}}
\newcommand{\dang}{\measuredangle} %% Directed angle
\newcommand{\ray}[1]{\overrightarrow{#1}}
\newcommand{\seg}[1]{\overline{#1}}
\newcommand{\arc}[1]{\wideparen{#1}}
\DeclareMathOperator{\cis}{cis}
\DeclareMathOperator*{\lcm}{lcm}
\DeclareMathOperator*{\argmin}{arg min}
\DeclareMathOperator*{\argmax}{arg max}
\newcommand{\cycsum}{\sum_{\mathrm{cyc}}}
\newcommand{\symsum}{\sum_{\mathrm{sym}}}
\newcommand{\cycprod}{\prod_{\mathrm{cyc}}}
\newcommand{\symprod}{\prod_{\mathrm{sym}}}
\newcommand{\Qed}{\begin{flushright}\qed\end{flushright}}
\newcommand{\parinn}{\setlength{\parindent}{1cm}}
\newcommand{\parinf}{\setlength{\parindent}{0cm}}
\newcommand{\inorm}{\norm_{\infty}}
\newcommand{\opensets}{\{V_{\alpha}\}_{\alpha\in I}}
\newcommand{\oset}{V_{\alpha}}
\newcommand{\opset}[1]{V_{\alpha_{#1}}}
\newcommand{\lub}{\text{lub}}
\newcommand{\del}[2]{\frac{\partial #1}{\partial #2}}
\newcommand{\Del}[3]{\frac{\partial^{#1} #2}{\partial^{#1} #3}}
\newcommand{\deld}[2]{\dfrac{\partial #1}{\partial #2}}
\newcommand{\Deld}[3]{\dfrac{\partial^{#1} #2}{\partial^{#1} #3}}
\newcommand{\lm}{\lambda}
\newcommand{\uin}{\mathbin{\rotatebox[origin=c]{90}{$\in$}}}
\newcommand{\usubset}{\mathbin{\rotatebox[origin=c]{90}{$\subset$}}}
\newcommand{\lt}{\left}
\newcommand{\rt}{\right}
\newcommand{\bs}[1]{\boldsymbol{#1}}
\newcommand{\exs}{\exists}
\newcommand{\st}{\strut}
\newcommand{\dps}[1]{\displaystyle{#1}}

\newcommand{\sol}{\setlength{\parindent}{0cm}\textbf{\textit{Solution:}}\setlength{\parindent}{1cm} }
\newcommand{\solve}[1]{\setlength{\parindent}{0cm}\textbf{\textit{Solution: }}\setlength{\parindent}{1cm}#1 \Qed}

% Adapted from:
%   https://github.com/SeniorMars/dotfiles/blob/main/latex_template/letterfonts.tex

% Things Lie
\newcommand{\kb}{\mathfrak b}
\newcommand{\kg}{\mathfrak g}
\newcommand{\kh}{\mathfrak h}
\newcommand{\kn}{\mathfrak n}
\newcommand{\ku}{\mathfrak u}
\newcommand{\kz}{\mathfrak z}
\DeclareMathOperator{\Ext}{Ext} % Ext functor
\DeclareMathOperator{\Tor}{Tor} % Tor functor
\newcommand{\gl}{\opname{\mathfrak{gl}}} % frak gl group
\renewcommand{\sl}{\opname{\mathfrak{sl}}} % frak sl group chktex 6

% More script letters etc.
\newcommand{\SA}{\mathcal A}
\newcommand{\SB}{\mathcal B}
\newcommand{\SC}{\mathcal C}
\newcommand{\SF}{\mathcal F}
\newcommand{\SG}{\mathcal G}
\newcommand{\SH}{\mathcal H}
\newcommand{\OO}{\mathcal O}

\newcommand{\SCA}{\mathscr A}
\newcommand{\SCB}{\mathscr B}
\newcommand{\SCC}{\mathscr C}
\newcommand{\SCD}{\mathscr D}
\newcommand{\SCE}{\mathscr E}
\newcommand{\SCF}{\mathscr F}
\newcommand{\SCG}{\mathscr G}
\newcommand{\SCH}{\mathscr H}

% Mathfrak primes
\newcommand{\km}{\mathfrak m}
\newcommand{\kp}{\mathfrak p}
\newcommand{\kq}{\mathfrak q}

% number sets
\newcommand{\RR}[1][]{\ensuremath{\ifstrempty{#1}{\mathbb{R}}{\mathbb{R}^{#1}}}}
\newcommand{\NN}[1][]{\ensuremath{\ifstrempty{#1}{\mathbb{N}}{\mathbb{N}^{#1}}}}
\newcommand{\ZZ}[1][]{\ensuremath{\ifstrempty{#1}{\mathbb{Z}}{\mathbb{Z}^{#1}}}}
\newcommand{\QQ}[1][]{\ensuremath{\ifstrempty{#1}{\mathbb{Q}}{\mathbb{Q}^{#1}}}}
\newcommand{\CC}[1][]{\ensuremath{\ifstrempty{#1}{\mathbb{C}}{\mathbb{C}^{#1}}}}
\newcommand{\PP}[1][]{\ensuremath{\ifstrempty{#1}{\mathbb{P}}{\mathbb{P}^{#1}}}}
\newcommand{\HH}[1][]{\ensuremath{\ifstrempty{#1}{\mathbb{H}}{\mathbb{H}^{#1}}}}
\newcommand{\FF}[1][]{\ensuremath{\ifstrempty{#1}{\mathbb{F}}{\mathbb{F}^{#1}}}}
% expected value
\newcommand{\EE}{\ensuremath{\mathbb{E}}}
\newcommand{\charin}{\text{ char }}
\DeclareMathOperator{\sign}{sign}
\DeclareMathOperator{\Aut}{Aut}
\DeclareMathOperator{\Inn}{Inn}
\DeclareMathOperator{\Syl}{Syl}
\DeclareMathOperator{\Gal}{Gal}
\DeclareMathOperator{\GL}{GL} % General linear group
\DeclareMathOperator{\SL}{SL} % Special linear group

%---------------------------------------
% BlackBoard Math Fonts :-
%---------------------------------------

%Captital Letters
\newcommand{\bbA}{\mathbb{A}}	\newcommand{\bbB}{\mathbb{B}}
\newcommand{\bbC}{\mathbb{C}}	\newcommand{\bbD}{\mathbb{D}}
\newcommand{\bbE}{\mathbb{E}}	\newcommand{\bbF}{\mathbb{F}}
\newcommand{\bbG}{\mathbb{G}}	\newcommand{\bbH}{\mathbb{H}}
\newcommand{\bbI}{\mathbb{I}}	\newcommand{\bbJ}{\mathbb{J}}
\newcommand{\bbK}{\mathbb{K}}	\newcommand{\bbL}{\mathbb{L}}
\newcommand{\bbM}{\mathbb{M}}	\newcommand{\bbN}{\mathbb{N}}
\newcommand{\bbO}{\mathbb{O}}	\newcommand{\bbP}{\mathbb{P}}
\newcommand{\bbQ}{\mathbb{Q}}	\newcommand{\bbR}{\mathbb{R}}
\newcommand{\bbS}{\mathbb{S}}	\newcommand{\bbT}{\mathbb{T}}
\newcommand{\bbU}{\mathbb{U}}	\newcommand{\bbV}{\mathbb{V}}
\newcommand{\bbW}{\mathbb{W}}	\newcommand{\bbX}{\mathbb{X}}
\newcommand{\bbY}{\mathbb{Y}}	\newcommand{\bbZ}{\mathbb{Z}}

%---------------------------------------
% MathCal Fonts :-
%---------------------------------------

%Captital Letters
\newcommand{\mcA}{\mathcal{A}}	\newcommand{\mcB}{\mathcal{B}}
\newcommand{\mcC}{\mathcal{C}}	\newcommand{\mcD}{\mathcal{D}}
\newcommand{\mcE}{\mathcal{E}}	\newcommand{\mcF}{\mathcal{F}}
\newcommand{\mcG}{\mathcal{G}}	\newcommand{\mcH}{\mathcal{H}}
\newcommand{\mcI}{\mathcal{I}}	\newcommand{\mcJ}{\mathcal{J}}
\newcommand{\mcK}{\mathcal{K}}	\newcommand{\mcL}{\mathcal{L}}
\newcommand{\mcM}{\mathcal{M}}	\newcommand{\mcN}{\mathcal{N}}
\newcommand{\mcO}{\mathcal{O}}	\newcommand{\mcP}{\mathcal{P}}
\newcommand{\mcQ}{\mathcal{Q}}	\newcommand{\mcR}{\mathcal{R}}
\newcommand{\mcS}{\mathcal{S}}	\newcommand{\mcT}{\mathcal{T}}
\newcommand{\mcU}{\mathcal{U}}	\newcommand{\mcV}{\mathcal{V}}
\newcommand{\mcW}{\mathcal{W}}	\newcommand{\mcX}{\mathcal{X}}
\newcommand{\mcY}{\mathcal{Y}}	\newcommand{\mcZ}{\mathcal{Z}}


%---------------------------------------
% Bold Math Fonts :-
%---------------------------------------

%Captital Letters
\newcommand{\bmA}{\boldsymbol{A}}	\newcommand{\bmB}{\boldsymbol{B}}
\newcommand{\bmC}{\boldsymbol{C}}	\newcommand{\bmD}{\boldsymbol{D}}
\newcommand{\bmE}{\boldsymbol{E}}	\newcommand{\bmF}{\boldsymbol{F}}
\newcommand{\bmG}{\boldsymbol{G}}	\newcommand{\bmH}{\boldsymbol{H}}
\newcommand{\bmI}{\boldsymbol{I}}	\newcommand{\bmJ}{\boldsymbol{J}}
\newcommand{\bmK}{\boldsymbol{K}}	\newcommand{\bmL}{\boldsymbol{L}}
\newcommand{\bmM}{\boldsymbol{M}}	\newcommand{\bmN}{\boldsymbol{N}}
\newcommand{\bmO}{\boldsymbol{O}}	\newcommand{\bmP}{\boldsymbol{P}}
\newcommand{\bmQ}{\boldsymbol{Q}}	\newcommand{\bmR}{\boldsymbol{R}}
\newcommand{\bmS}{\boldsymbol{S}}	\newcommand{\bmT}{\boldsymbol{T}}
\newcommand{\bmU}{\boldsymbol{U}}	\newcommand{\bmV}{\boldsymbol{V}}
\newcommand{\bmW}{\boldsymbol{W}}	\newcommand{\bmX}{\boldsymbol{X}}
\newcommand{\bmY}{\boldsymbol{Y}}	\newcommand{\bmZ}{\boldsymbol{Z}}
%Small Letters
\newcommand{\bma}{\boldsymbol{a}}	\newcommand{\bmb}{\boldsymbol{b}}
\newcommand{\bmc}{\boldsymbol{c}}	\newcommand{\bmd}{\boldsymbol{d}}
\newcommand{\bme}{\boldsymbol{e}}	\newcommand{\bmf}{\boldsymbol{f}}
\newcommand{\bmg}{\boldsymbol{g}}	\newcommand{\bmh}{\boldsymbol{h}}
\newcommand{\bmi}{\boldsymbol{i}}	\newcommand{\bmj}{\boldsymbol{j}}
\newcommand{\bmk}{\boldsymbol{k}}	\newcommand{\bml}{\boldsymbol{l}}
\newcommand{\bmm}{\boldsymbol{m}}	\newcommand{\bmn}{\boldsymbol{n}}
\newcommand{\bmo}{\boldsymbol{o}}	\newcommand{\bmp}{\boldsymbol{p}}
\newcommand{\bmq}{\boldsymbol{q}}	\newcommand{\bmr}{\boldsymbol{r}}
\newcommand{\bms}{\boldsymbol{s}}	\newcommand{\bmt}{\boldsymbol{t}}
\newcommand{\bmu}{\boldsymbol{u}}	\newcommand{\bmv}{\boldsymbol{v}}
\newcommand{\bmw}{\boldsymbol{w}}	\newcommand{\bmx}{\boldsymbol{x}}
\newcommand{\bmy}{\boldsymbol{y}}	\newcommand{\bmz}{\boldsymbol{z}}

%---------------------------------------
% Scr Math Fonts :-
%---------------------------------------

\newcommand{\sA}{{\mathscr{A}}}   \newcommand{\sB}{{\mathscr{B}}}
\newcommand{\sC}{{\mathscr{C}}}   \newcommand{\sD}{{\mathscr{D}}}
\newcommand{\sE}{{\mathscr{E}}}   \newcommand{\sF}{{\mathscr{F}}}
\newcommand{\sG}{{\mathscr{G}}}   \newcommand{\sH}{{\mathscr{H}}}
\newcommand{\sI}{{\mathscr{I}}}   \newcommand{\sJ}{{\mathscr{J}}}
\newcommand{\sK}{{\mathscr{K}}}   \newcommand{\sL}{{\mathscr{L}}}
\newcommand{\sM}{{\mathscr{M}}}   \newcommand{\sN}{{\mathscr{N}}}
\newcommand{\sO}{{\mathscr{O}}}   \newcommand{\sP}{{\mathscr{P}}}
\newcommand{\sQ}{{\mathscr{Q}}}   \newcommand{\sR}{{\mathscr{R}}}
\newcommand{\sS}{{\mathscr{S}}}   \newcommand{\sT}{{\mathscr{T}}}
\newcommand{\sU}{{\mathscr{U}}}   \newcommand{\sV}{{\mathscr{V}}}
\newcommand{\sW}{{\mathscr{W}}}   \newcommand{\sX}{{\mathscr{X}}}
\newcommand{\sY}{{\mathscr{Y}}}   \newcommand{\sZ}{{\mathscr{Z}}}


%---------------------------------------
% Math Fraktur Font
%---------------------------------------

%Captital Letters
\newcommand{\mfA}{\mathfrak{A}}	\newcommand{\mfB}{\mathfrak{B}}
\newcommand{\mfC}{\mathfrak{C}}	\newcommand{\mfD}{\mathfrak{D}}
\newcommand{\mfE}{\mathfrak{E}}	\newcommand{\mfF}{\mathfrak{F}}
\newcommand{\mfG}{\mathfrak{G}}	\newcommand{\mfH}{\mathfrak{H}}
\newcommand{\mfI}{\mathfrak{I}}	\newcommand{\mfJ}{\mathfrak{J}}
\newcommand{\mfK}{\mathfrak{K}}	\newcommand{\mfL}{\mathfrak{L}}
\newcommand{\mfM}{\mathfrak{M}}	\newcommand{\mfN}{\mathfrak{N}}
\newcommand{\mfO}{\mathfrak{O}}	\newcommand{\mfP}{\mathfrak{P}}
\newcommand{\mfQ}{\mathfrak{Q}}	\newcommand{\mfR}{\mathfrak{R}}
\newcommand{\mfS}{\mathfrak{S}}	\newcommand{\mfT}{\mathfrak{T}}
\newcommand{\mfU}{\mathfrak{U}}	\newcommand{\mfV}{\mathfrak{V}}
\newcommand{\mfW}{\mathfrak{W}}	\newcommand{\mfX}{\mathfrak{X}}
\newcommand{\mfY}{\mathfrak{Y}}	\newcommand{\mfZ}{\mathfrak{Z}}

%Small Letters
\newcommand{\mfa}{\mathfrak{a}}	\newcommand{\mfb}{\mathfrak{b}}
\newcommand{\mfc}{\mathfrak{c}}	\newcommand{\mfd}{\mathfrak{d}}
\newcommand{\mfe}{\mathfrak{e}}	\newcommand{\mff}{\mathfrak{f}}
\newcommand{\mfg}{\mathfrak{g}}	\newcommand{\mfh}{\mathfrak{h}}
\newcommand{\mfi}{\mathfrak{i}}	\newcommand{\mfj}{\mathfrak{j}}
\newcommand{\mfk}{\mathfrak{k}}	\newcommand{\mfl}{\mathfrak{l}}
\newcommand{\mfm}{\mathfrak{m}}	\newcommand{\mfn}{\mathfrak{n}}
\newcommand{\mfo}{\mathfrak{o}}	\newcommand{\mfp}{\mathfrak{p}}
\newcommand{\mfq}{\mathfrak{q}}	\newcommand{\mfr}{\mathfrak{r}}
\newcommand{\mfs}{\mathfrak{s}}	\newcommand{\mft}{\mathfrak{t}}
\newcommand{\mfu}{\mathfrak{u}}	\newcommand{\mfv}{\mathfrak{v}}
\newcommand{\mfw}{\mathfrak{w}}	\newcommand{\mfx}{\mathfrak{x}}
\newcommand{\mfy}{\mathfrak{y}}	\newcommand{\mfz}{\mathfrak{z}}


% ==============================

\title{\Huge{COMP 458/558}\\Quantum Computing Algorithms}
\author{\huge{Micah Kepe}}
\date{}

% ==============================

\makeindex[title=Appendix, intoc]

% ==============================

\begin{document}

\maketitle
\newpage  % or `\cleardoublepage` if using two-sided layout
\renewcommand*\contentsname{Table of Contents}
\pdfbookmark[section]{\contentsname}{toc}
\tableofcontents
\pagebreak

% ==============================
% PREFACE
% ==============================
\chapter*{Preface}
Quantum computing is an emerging field poised to revolutionize multiple
disciplines, including cryptography, optimization, machine learning, and
scientific simulations. This handbook serves as a structured compilation of
notes from \textit{COMP 458/558: Quantum Computing Algorithms}, offered at Rice
University during Spring 2025. The content is derived from lectures,
discussions, and supplementary readings throughout the semester, aiming to
reinforce core concepts and provide a comprehensive reference for students
and enthusiasts alike.

\vspace{0.3cm}

\noindent
The notes begin with foundational mathematical principles, including linear
algebra and probability theory, before delving into the core tenets of
quantum mechanics that underpin quantum computing. Subsequent sections
explore quantum circuits, essential quantum gates, multi-qubit systems, and
fundamental quantum algorithms such as Grover’s search and Shor’s factoring
algorithm. The latter phases of the handbook introduce advanced topics,
including variational quantum algorithms, quantum simulation techniques, and
emerging paradigms in quantum machine learning and optimization.

\vspace{0.3cm}

\noindent
I would like to extend my deepest gratitude to \textbf{Professor Tirthak Patel},
whose guidance and expertise have been instrumental in shaping my
understanding of quantum computing. I am also grateful to my peers for the
collaborative discussions that enriched my learning experience.

\vspace{0.3cm}

\noindent
As quantum computing continues to evolve, these notes will remain a living
document, subject to refinement and expansion. I welcome feedback,
corrections, and contributions from readers to ensure the accuracy and
completeness of this resource.

\vspace{1cm}

\noindent \textbf{Micah Kepe}  \\
Rice University  \\
Spring 2025


% ==============================
% PHASE I: INTRODUCTION
% ==============================
\chapter{Phase I: Introduction and Background}

% ● Overview of quantum computing concepts and
% applications
% ● Historical background and current state of quantum
% computing
% ● Review of linear algebra concepts, notation, and
% vector spaces
\section{Lecture 1: Overview of Quantum Computing Concepts}\label{sec:lecture1}
\dfn{Quantum Computing}{\textbf{Quantum computing} is a computational paradigm
  leveraging quantum mechanical principles such as superposition, entanglement,
  and interference to perform computations that can surpass the capabilities of
  classical systems for specific tasks.\footnote{Superposition
    \index{superposition} allows quantum bits (qubits)
    to exist in multiple states simultaneously, and entanglement enables
correlations between qubits even at a distance.}}

\subsection*{Historical Development of Quantum Computing}
\begin{itemize}
  \item \textbf{1980s-1990s:} Conception of quantum computing, with
    foundational ideas like the quantum Turing machine and quantum gates.

  \item \textbf{1990s-2000s:} Demonstration of key building blocks, such as
    quantum algorithms (e.g., Shor's and Grover's algorithms).

  \item \textbf{2016:} Emergence of quantum computing clouds, enabling
    access to quantum hardware via the internet.

  \item \textbf{2019:} First claims of \textbf{quantum advantage},
    showcasing tasks where quantum computers outperform classical
    counterparts.

  \item \textbf{2024:} Increasing qubit counts and improvements in quantum
    error correction techniques.

\end{itemize}

\subsection*{Applications of Quantum Computing}

Quantum computing offers \textbf{speedup} in areas such as:

\begin{enumerate}
  \item \textbf{Quantum Simulation:} Applications in chemistry, physics,
    and materials science, such as simulating molecular energy levels and
    drug discovery.

  \item \textbf{Security and Encryption:} Developing quantum-safe
    cryptographic protocols and random number generation.

  \item \textbf{Search and Optimization:} Enhancing solutions for weather
    forecasting, financial modeling, traffic planning, and resource
    allocation.

\end{enumerate}

\ex{Example: Quantum Speedup in Drug Discovery}{Drug discovery benefits from
  quantum simulation by enabling more accurate modeling of molecular
interactions, which classical computers struggle to achieve efficiently.}

\subsection*{Classical vs. Quantum Computing Paradigms}
\begin{itemize}

  \item \textbf{Classical Computing:\index{classical computing}} Utilizes
    traditional processing units (CPU, GPU, FPGA) and executes deterministic
    computations.

  \item \textbf{Quantum Computing:} Employs quantum processing units (QPU)
    with probabilistic computation based on quantum states.

\end{itemize}

\nt{Note: Classical computing paradigms still dominate in tasks that require
  precision and deterministic results. Quantum computing excels in
probabilistic or exponentially large state-space problems.}

%%%%%%%%%%%%%%%%%%%%%%%%%%%%%%%%%%%%%%%%%%%%%%
% End of Lecture 1
%%%%%%%%%%%%%%%%%%%%%%%%%%%%%%%%%%%%%%%%%%%%%%

\section{Lecture 2: Review of Linear Algebra Concepts}\label{sec:lecture2}

Linear algebra provides the foundation for manipulating quantum states, which
are represented using vectors and matrices in a complex vector space.

\dfn{Vectors: Row and Column Vectors}{A \textbf{vector} \index{vector} is an
  ordered list of numbers, which can be represented as either a row or column
  vector. The components of vectors in quantum computing belong to the field of
complex numbers ($\mathbb{C}$).}

\subsection*{Column Vectors} \index{vector!column| see{ ket }}

A column vector is a vertical arrangement of numbers:
\[
  \mathbf{v} =
  \begin{bmatrix}
    v_1 \\
    v_2 \\
    \vdots \\
    v_n
  \end{bmatrix}, \quad v_i \in \mathbb{C}.
\]

\subsection*{Row Vectors}

A row vector is the complex conjugate transpose \textbf{vector}
\index{vector!adjoint} of a column vector:
\[
  \mathbf{v}^\dagger =
  \begin{bmatrix}
    \overline{v_1} & \overline{v_2} & \dots & \overline{v_n}
  \end{bmatrix}.
\]

The adjoint of a column vector is a row vector, and vice versa. We represent
the adjoint of a vector using the dagger symbol ($\dagger$).
\index{vector!dagger@\textsl{dagger}| see{ adjoint }}

\subsection*{Dirac Notation \index{vector!Dirac notation}}

In quantum computing, vectors are represented using \textbf{Dirac notation}
(bra-ket notation):

\begin{itemize}
  \item \textbf{Ket} \index{vector!Dirac notation!ket@\textsl{ket}} \(
    |v\rangle \): Represents a column vector.

  \item \textbf{Bra} \index{vector!Dirac notation!bra@\textsl{bra}} \(
    \langle v | \): Represents the adjoint (conjugate transpose) of the ket.

  \item Example: \( |v\rangle = \begin{bmatrix} 1 + i \\ 2 \end{bmatrix},
    \quad \langle v | = \begin{bmatrix} 1 - i & 2 \end{bmatrix} \).
\end{itemize}

\dfn{Euler's Formula}{Euler's formula \index{Euler's formula} relates complex
  exponentials to trigonometric functions:
  \[
    e^{i\omega} = \cos(\omega) + i\sin(\omega)
  \]

This is fundamental in representing quantum states and transformations.}

\dfn{Inner Product}{The \textbf{inner product} \index{vector!inner product}
  of two vectors $\mathbf{v}, \mathbf{w} \in \mathbb{C}^n$ is defined as:
  \[
    \langle \mathbf{v}, \mathbf{w} \rangle = \mathbf{v}^\dagger \mathbf{w} =
    \sum_{i=1}^n \overline{v_i}w_i
  \]

  which measures the overlap between two quantum states.

  \ex{Inner Product Example}{

    Given two vectors:
    \[
      \mathbf{v} = \begin{bmatrix} 1 \\ i \end{bmatrix}, \quad
      \mathbf{w} = \begin{bmatrix} 2 \\ 1 \end{bmatrix}
    \]

    The inner product is:
    \[
      \langle \mathbf{v}, \mathbf{w} \rangle = \begin{bmatrix} 1 & -i
      \end{bmatrix}
      \begin{bmatrix} 2 \\ 1 \end{bmatrix} = 2 - i
    \]
  }

  We also have the following property that the inner product is equivalent to
  the square of the Euclidean norm of a vector:
  \index{vector!Euclidean norm}
  \[
    \langle \mathbf{v}, \mathbf{v} \rangle = \|\mathbf{v}\|^2
  \]
}

\dfn{Outer Product}{The \textbf{outer product} \index{vector!outer product}
  of two vectors $\mathbf{v} \in \mathbb{C}^m$ and $\mathbf{w} \in
  \mathbb{C}^n$ produces an $m \times n$ matrix:
  \[
    \mathbf{v}\mathbf{w}^\dagger =
    \begin{bmatrix} v_1 \\ v_2 \\ \vdots \\ v_m \end{bmatrix}
    \begin{bmatrix} \overline{w_1} & \overline{w_2} & \dots & \overline{w_n}
    \end{bmatrix}
  \]

This operation is useful for constructing quantum operators.}

\dfn{Tensor Product}{The \textbf{tensor product} (or Kronecker product)
  allows us to describe multi-qubit systems. Given two vectors:
  \index{vector!tensor product}
  \[
    \mathbf{v} = \begin{bmatrix} v_1 \\ v_2 \end{bmatrix}, \quad
    \mathbf{w} = \begin{bmatrix} w_1 \\ w_2 \end{bmatrix}
  \]

  Their tensor product is:
  \[
    \mathbf{v} \otimes \mathbf{w} =
    \begin{bmatrix}
      v_1 w_1 \\
      v_1 w_2 \\
      v_2 w_1 \\
      v_2 w_2
    \end{bmatrix}
  \]

  The tensor product expands the state space, allowing representation of
entangled states.}

\subsection*{Orthagonality} \index{vector!orthogonality}
Two vectors $v, w \in \mathbb{C}^n$ are \textbf{orthogonal} if their
inner product is zero:
\[
  \langle \mathbf{v}, \mathbf{w} \rangle = 0
\]

Orthogonal vectors are linearly independent and span a subspace of the vector
space. As you might remember from linear algebra, a set of orthogonal vectors
can be used to construct an orthonormal basis, and any vector can be expressed
as a linear combination of the basis vectors.

\vspace{0.3cm}

This will be useful when we cover the quantum bases in \autoref{sec:lecture3}.

\vspace{0.3cm}

\dfn{Adjoint of a Matrix}{The \textbf{adjoint} (or Hermitian conjugate) of a
  matrix $A$ is obtained by taking the transpose and complex conjugate of
  each entry:
  \[
    A^\dagger = \overline{A^T}
  \]
  If $A$ is:
  \[
    A = \begin{bmatrix}
      1 & i \\
      2 & 3
    \end{bmatrix}
  \]
  Then its adjoint is:
  \[
    A^\dagger =
    \begin{bmatrix}
      1 & 2 \\
      - i & 3
    \end{bmatrix}
\]}

\dfn{Unitary Matrix}{A square matrix $U$ is called \textbf{unitary}
  \index{matrix!unitary} if its adjoint is equal to its
  inverse:
  \[
    U^\dagger U = I
  \]

  where $I$ is the identity matrix. Unitary matrices preserve the norm of
  quantum states and represent reversible quantum operations. Example:
  \[
    U = \frac{1}{\sqrt{2}}
    \begin{bmatrix}
      1 & 1 \\
      1 & -1
    \end{bmatrix}, \quad U^\dagger U = I
\]}

\dfn{Hermitian Matrix}{A square matrix $H$ is called \textbf{Hermitian}
  \index{matrix!Hermitian} if it is equal to its adjoint:
  \[
    H = H^\dagger
  \]

  Hermitian matrices represent observable quantities in quantum mechanics and
  have real eigenvalues. Example:
  \[
    H = \begin{bmatrix}
      2 & i \\
      - i & 2
    \end{bmatrix}
  \]

  Since $H^\dagger = H$, it is Hermitian.

  \nt{Hermitian matrices \textbf{can't} have complex numbers in their diagonal
    General case illustration:
    \[
      M = \begin{bmatrix} a + ib & c + id \\ e + if & g  + ih \end{bmatrix}
      \quad \Rightarrow \quad
      M^\dagger = \begin{bmatrix} a - ib & e - if \\ c - id & g - ih
      \end{bmatrix}
      \quad \Rightarrow \quad M \neq M^\dagger
    \]

    \vspace{0.3cm}

    $\therefore$ Hermitian matrices have real diagonal elements.

    \vspace{0.3cm}

    Additionally, the general matrix $M$ shown above is Hermitian iff. $c =
    e, d = -f$
  }
}

Hermitian matrices are not necessarily unitary, and unitary matrices are not
necessarily Hermitian. A matrix can be both, but these properties are distinct:
\[
  H = H^\dagger \quad (\text{Hermitian}), \quad U U^\dagger = I \quad (\text{unitary})
\]

In quantum computing, gates that are both Hermitian and unitary (e.g., Pauli
gates) satisfy $U = U^\dagger$ and $U^2 = I$, making them their own
inverses---they cancel out when applied twice in succession. However, not all
unitary gates are Hermitian (e.g., the phase gate, which is unitary but not
Hermitian):
\[
  \text{Pauli } X = \begin{bmatrix} 0 & 1 \\ 1 & 0 \end{bmatrix}, \quad X = X^\dagger, \quad X^2 = I; \quad
  \text{Phase } S = \begin{bmatrix} 1 & 0 \\ 0 & i \end{bmatrix}, \quad S S^\dagger = I, \quad S \neq S^\dagger
\]

\vspace{0.3cm}
%%%%%%%%%%%%%%

\dfn{Eigenvalues and Eigenvectors}{
  For a square matrix $A \in \mathbb{C}^{n\times n}$, a vector
  $\mathbf{v} \neq \mathbf{0}$ is an \textbf{eigenvector} if:
  \index{vector!eigenvector}
  \[
    A\mathbf{v} = \lambda\mathbf{v}
  \]

  where $\lambda \in \mathbb{C}$ is the \textbf{eigenvalue}. Eigenvalues
  provide insight into the structure of linear transformations.
  \index{matrix!eigenvalue}

  In Braket notation, the eigenvalue equation is: \index{matrix!eigenvalue equation}
  \[
    A|\mathbf{v}\rangle = \lambda|\mathbf{v}\rangle
  \]
}

\ex{Example: Eigenvalues}{For the matrix
  \[
    A = \begin{bmatrix} 1 & i \\ -i & 1 \end{bmatrix}
  \]

  The characteristic equation is:
  \[
    \det(A - \lambda I) = (1 - \lambda)^2 + 1 = 0
  \]

Solving gives eigenvalues $\lambda = 1 \pm i$.}

\dfn{Quantum Bits/ Qubits}{
  A \textbf{qubit} \index{qubit} can be defined mathematically as follows:
  \[
    \ket{\psi} = \bmatrix \alpha_1 \\ \alpha_2 \endbmatrix \in \mathbb{C}^2
  \]

  where:
  \[
    \alpha_1, \alpha_2 \in \mathbb{C} \quad \text{and} \quad |\alpha_1|^2 +
    |\alpha_2|^2 = 1
  \]
}

The first property ensures that the qubit is normalized, while the second
property ensures that the qubit is in a superposition of the basis states.

\vspace{0.3cm}

The first universal basis that we will look at is the computational basis,
which consists of the states \(\zero\) and \(\one\):
\index{universal bases!computational}
\[
  \text{Zero state} = \zero = \bmatrix 1 \\ 0 \endbmatrix
  \quad \text{One state} = \one = \bmatrix 0 \\ 1 \endbmatrix
\]

A quantum state vector $\ket{\psi}$ can be expressed as a linear
combination of the basis states:
\[
  \ket{\psi} = \alpha_1\zero + \alpha_2\one
\]

\nt{Properties of the computational basis:
  \index{universal bases!computational!properties@\textit{properties}}

  \begin{itemize}
    \item \textbf{The computational basis states are orthogonal:}
      \[
        \langle 0 | 1 \rangle = \zero^\dagger \one = \bmatrix 1 & 0 \endbmatrix
        \bmatrix 0 \\ 1 \endbmatrix = 0
      \]

    \item \textbf{The computational basis states are normalized:}
      \[
        \langle 0 | 0 \rangle = \zero^\dagger \zero = \bmatrix 1 & 0
        \endbmatrix \bmatrix 1 \\ 0 \endbmatrix = 1
      \]
  \end{itemize}
}

  %%%%%%%%%%%%%%%%%%%%%%%%%%%%%5

\qs{}{Show that any unitary matrix preserves the inner product of two
vectors.}

\sol{Since a unitary matrix satisfies \( U^\dagger U = I \), we have:
  \[
    \langle U\mathbf{v}, U\mathbf{w} \rangle = \mathbf{v}^\dagger
    (U^\dagger U) \mathbf{w} = \mathbf{v}^\dagger \mathbf{w}
  \]
Thus, inner products are preserved.}


%%%%%%%%%%%%%%%%%%%%%%%%%%%%%%%%%%%%%%%%%%%%%%
% End of Lecture 2
%%%%%%%%%%%%%%%%%%%%%%%%%%%%%%%%%%%%%%%%%%%%%%


%%%%

% ● Quantum state representation and quantum bits (qubits)
% ● Probability theory and its application to quantum systems
% ● Measurement and quantum state collapse
% ● Statistical analysis of quantum measurements
\section{Lecture 3: Quantum Bits and Quantum States}

\dfn{Qubit}{A \textbf{qubit} is the fundamental unit of quantum information. Unlike a classical bit, which is either $0$ or $1$, a qubit can exist in a \vocab{superposition} of states:
\[ |\psi\rangle = \alpha|0\rangle + \beta|1\rangle, \quad \text{where } \alpha, \beta \in \mathbb{C} \text{ and } |\alpha|^2 + |\beta|^2 = 1 \]

Key features of qubits include:
\begin{itemize}
    \ii \textbf{Superposition:} A qubit can exist simultaneously in multiple basis states.
    \ii \textbf{Complex Amplitudes:} Coefficients $\alpha$ and $\beta$ are complex numbers carrying magnitude and phase information.
    \ii \textbf{Interference:} Quantum states can interfere constructively or destructively.
    \ii \textbf{Entanglement:} Qubits can be correlated in ways that classical bits cannot.
\end{itemize}}

\dfn{Classical Computing Paradigms}{Quantum computing introduces a fundamentally different computational model:
\begin{itemize}
    \ii \textbf{Deterministic Computing:} Uses discrete states (0 or 1) with predictable transitions.
    \ii \textbf{Analog Computing:} Uses continuous values susceptible to noise accumulation.
    \ii \textbf{Probabilistic Computing:} Represents probabilistic mixtures of states.
    \ii \textbf{Quantum Computing:} Allows coherent superposition with complex amplitudes and quantum interference.
\end{itemize}}

\dfn{Dirac Notation}{Quantum states are represented using \vocab{Dirac notation} (bra-ket notation):
\begin{itemize}
    \ii \textbf{Ket:} \( |0\rangle, |1\rangle \) represent computational basis states
    \ii Computational basis vectors:
    \[
    |0\rangle = \begin{bmatrix} 1 \\ 0 \end{bmatrix}, \quad
    |1\rangle = \begin{bmatrix} 0 \\ 1 \end{bmatrix}
    \]
    \ii General state: \( |\psi\rangle = \alpha|0\rangle + \beta|1\rangle \)
\end{itemize}}

\dfn{Basis States}{Common qubit bases include:
\begin{itemize}
    \ii \textbf{Computational Basis:} \( |0\rangle, |1\rangle \)
    \ii \textbf{Hadamard Basis:}
    \[ |+\rangle = \frac{1}{\sqrt{2}}(|0\rangle + |1\rangle), \quad
       |-\rangle = \frac{1}{\sqrt{2}}(|0\rangle - |1\rangle) \]
    \ii \textbf{Circular Polarization Basis:}
    \[ |L\rangle = \frac{1}{\sqrt{2}}(|0\rangle + i|1\rangle), \quad
       |R\rangle = \frac{1}{\sqrt{2}}(|0\rangle - i|1\rangle) \]
\end{itemize}}

\dfn{Bloch Sphere}{A geometric representation of a single qubit state:
\[ |\psi\rangle = \cos\left(\frac{\theta}{2}\right)|0\rangle + e^{i\phi}\sin\left(\frac{\theta}{2}\right)|1\rangle \]
Where:
\begin{itemize}
    \ii \( \theta \in [0,\pi] \) is the polar angle
    \ii \( \phi \in [0, 2\pi) \) is the azimuthal angle
    \ii Cartesian coordinates:
    \[ x = \sin\theta \cos\phi, \quad
       y = \sin\theta \sin\phi, \quad
       z = \cos\theta \]

     \ex{Example Bloch Sphere Representation}{
       \text{For the state } \(\theta = \frac{\pi}{2}, \phi = 0\):
     \bloch{90}{0}
   }

\end{itemize}}

\dfn{Quantum Measurement}{When a qubit is measured:
\begin{itemize}
    \ii The quantum state \textit{collapses} to an eigenstate
    \ii Measurement probability depends on squared amplitude
    \ii Computational basis measurement probabilities:
    \[ P(0) = |\alpha|^2, \quad P(1) = |\beta|^2 \]
    \ii Post-measurement state:
    \[ |\psi_{\text{new}}\rangle = \frac{|b\rangle \langle b | \psi \rangle}{\sqrt{P(b)}} \]
\end{itemize}}

\ex{Measurement Example}{For the state \( |\psi\rangle = \frac{1}{\sqrt{3}}|0\rangle + \sqrt{\frac{2}{3}}|1\rangle \):
\begin{itemize}
    \ii Probability of measuring \( |0\rangle \): \( P(0) = \frac{1}{3} \)
    \ii Probability of measuring \( |1\rangle \): \( P(1) = \frac{2}{3} \)
\end{itemize}}

\qs{Orthonormality Check}{Verify the inner products of basis states:
\begin{align*}
    \langle 0 | 1 \rangle &= 0 \\
    \langle 0 | 0 \rangle &= 1 \\
    \langle + | + \rangle &= 1 \\
    \langle + | - \rangle &= 0
\end{align*}}

\sol{These relations hold due to the orthonormal nature of quantum basis states.}



%%%%

% ● Quantum gates and transformations
% ● Unitary matrices and their properties
% ● Quantum circuits, systems, and properties
% ● Multi-qubit systems
\section{Lecture 4: Quantum Gates and Transformations}\label{sec:lecture4}

Quantum gates manipulate qubits through unitary transformations, preserving
quantum information and enabling quantum computation. This section explores
key quantum operations, their mathematical properties, and circuit
representations.

\dfn{Qubit Superposition and Hilbert Space}{A \textbf{qubit} exists in a
  complex vector space called a \textbf{Hilbert space}. The state of a qubit
  is given by:

  \index{qubit!superposition}

  \[
    |\psi\rangle = \alpha |0\rangle + \beta |1\rangle, \quad \text{where }
    \alpha, \beta \in \mathbb{C} \text{ and } |\alpha|^2 + |\beta|^2 = 1.
  \]

}

\subsection*{Measurement and Superposition Collapse}

When a qubit is measured in the computational basis $\{|0\rangle,
|1\rangle\}$, it collapses to one of the basis states with probability:

\[
  \boxed{
    P(0) = \norm{\alpha_1}^2, \quad P(1) = \norm{\alpha_2}^2.
  }
\]

The post-measurement state is:

\[
  \boxed{
    |\psi_{\text{measurement}}\rangle = \frac{|b\rangle \langle b | \psi
    \rangle}{\sqrt{P(b)}}
  }
\]

\noindent
where $b \in \{0,1\}$. This formula captures the quantum measurement
postulate and ensures proper normalization of the post-measurement state.

\vspace{0.3cm}

In the computational basis, the probability of measuring $|b\rangle$ is:

\index{universal bases!computational!measurement}

\[
  \boxed{
    P(b) = \norm{\langle b | \psi \rangle}^2
  }
\]


\index{quantum measurement!properties@\textit{properties}}
\nt{Probability Properties of Measurement:
  \[
    \begin{aligned}
      P(0) &= 1 - P(1) \\
      P(+) &= 1 - P(-) \\
      P(+i) &= 1 - P(-i)
    \end{aligned}
  \]
}

\subsection*{Quantum Gates and Operations} \index{quantum gates}

\textbf{Quantum gates} are unitary matrices that transform qubits. A general
qubit transformation is given by:

\[
  |\psi_{\text{final}}\rangle = U |\psi_{\text{initial}}\rangle
\]

\noindent
where $U$ is a unitary matrix satisfying $U^\dagger U = I$. Key properties of
quantum gates include:

\index{quantum gates!properties@\textit{properties}}

\begin{itemize}
  \item \textbf{Reversibility:} All quantum operations are reversible due
    to unitarity

  \item \textbf{Preservation of Norm:} The normalization condition
    $|\alpha|^2 + |\beta|^2 = 1$ is preserved

  \item \textbf{Linearity:} Gates act linearly on superposition states

\end{itemize}

\index{quantum gates!rotation gates}
\dfn{Rotation Gates}{Rotation gates rotate a qubit state around the Bloch
  sphere:
  \begin{itemize}

    \index{quantum gates!rotation gates!X-axis}
    \item \textbf{Rotation about X-axis:}
      \[
        \boxed{
          R_X(\omega) =
          \begin{bmatrix}
            \cos\frac{\omega}{2} & -i\sin\frac{\omega}{2} \\
            -i\sin\frac{\omega}{2} & \cos\frac{\omega}{2}
          \end{bmatrix}
        }
      \]

      Effect: Rotates state by angle $\omega$ around X-axis

      \index{quantum gates!rotation gates!Y-axis}
    \item \textbf{Rotation about Y-axis:}
      \[
        \boxed{
          R_Y(\omega) =
          \begin{bmatrix}
            \cos\frac{\omega}{2} & -\sin\frac{\omega}{2} \\
            \sin\frac{\omega}{2} & \cos\frac{\omega}{2}
          \end{bmatrix}
        }
      \]

      Effect: Rotates state by angle $\omega$ around Y-axis

      \index{quantum gates!rotation gates!Z-axis}
    \item \textbf{Rotation about Z-axis:}
      \[
        \boxed{
          R_Z(\omega) =
          \begin{bmatrix}
            e^{-i\omega/2} & 0 \\
            0 & e^{i\omega/2}
          \end{bmatrix}
        }
      \]

      Effect: Adds a relative phase between $|0\rangle$ and $|1\rangle$
      components
  \end{itemize}
}

\index{matrix!Pauli matrices}
\dfn{Pauli Matrices and Gates}{The \textbf{Pauli matrices} define fundamental
  quantum operations:

  \index{quantum gates!NOT gate}
  \begin{itemize}
    \item \textbf{Pauli-X (NOT Gate, Bit-Flip):}
      \[
        \boxed{
          X = \begin{bmatrix} 0 & 1 \\ 1 & 0 \end{bmatrix}
        }
      \]

      Effect: $X|0\rangle = |1\rangle$, $X|1\rangle = |0\rangle$

      \index{quantum gates!Pauli-Y gate}
    \item \textbf{Pauli-Y (Combination of X and Z with phase):}
      \[
        \boxed{
          Y = \begin{bmatrix} 0 & -i \\ i & 0 \end{bmatrix}
        }
      \]

      Effect: $Y|0\rangle = i|1\rangle$, $Y|1\rangle = -i|0\rangle$

      \index{quantum gates!Phase-Flip gate}
    \item \textbf{Pauli-Z (Phase-Flip Gate):}
      \[
        \boxed{
          Z = \begin{bmatrix} 1 & 0 \\ 0 & -1 \end{bmatrix}
        }
      \]

      Effect: $Z|0\rangle = |0\rangle$, $Z|1\rangle = -|1\rangle$

  \end{itemize}

  \aside{Each of these matrices is both \textbf{Hermitian} ($A = A^\dagger$) and
  \textbf{unitary} ($A^\dagger A = I$).}

  Important relationships:
  \begin{itemize}
    \item $X^2 = Y^2 = Z^2 = I$
    \item $XY = iZ$, $YZ = iX$, $ZX = iY$
    \item $YX = -iZ$, $ZY = -iX$, $XZ = -iY$
\end{itemize}}

\subsection*{Circuit Notation} \index{Circuit notation}

Quantum circuits visually represent quantum operations. Each qubit is
represented as a horizontal line, and gates are applied sequentially from
left to right. Important circuit elements include: \footnote{For rendering
quantum circuits, consider using the \texttt{quantikz} package in \LaTeX:
\url{https://ctan.org/pkg/quantikz}}

\begin{itemize}

  \item \textbf{Single-qubit gates:} Represented as boxes with gate symbols

  \item \textbf{Measurements:} Depicted with a meter symbol

  \item \textbf{Time flow:} Left to right in circuits (\textit{\textbf{opposite
    of matrix multiplication order}}) \footnote {
      For example, the circuit $U_1U_2$ corresponds to the
      matrix product $U_2U_1$.
  }

\end{itemize}

\ex{Example: Complex Circuit Analysis}{Consider the circuit applying the
  sequence $HZH$ to $|0\rangle$:

  \[
    \begin{aligned}
      |\psi_1\rangle &= H|0\rangle = \frac{1}{\sqrt{2}}(|0\rangle +
      |1\rangle) \\
      %
      |\psi_2\rangle &= Z|\psi_1\rangle = \frac{1}{\sqrt{2}}(|0\rangle -
      |1\rangle) \\
      %
      |\psi_3\rangle &= H|\psi_2\rangle = |1\rangle
    \end{aligned}
  \]

  This sequence performs a NOT operation on $|0\rangle$ using only Hadamard and
  Phase-flip gates.

  \[
    \begin{quantikz}
      \lstick{$|0\rangle$} & \gate{H} & \gate{Z} & \gate{H} & \one \\
    \end{quantikz}
  \]
}

\ex{Another Circuit Example}{

  \[
    \begin{aligned}
      XY \zero & = X \begin{bmatrix} 0 & -i \\ i & 0 \end{bmatrix} \begin{bmatrix} 1 \\ 0 \end{bmatrix}\\
      %
      & = \begin{bmatrix} 0 & 1 \\ 1 & 0 \end{bmatrix} \begin{bmatrix} 0 \\ i \end{bmatrix} \\
      %
      & = \begin{bmatrix} i \\ 0 \end{bmatrix} = i \one
    \end{aligned}
  \]

  \[
    \begin{quantikz}
      \lstick{$\zero$} & \gate{Y} & \gate{X} & i \zero \\
    \end{quantikz}
  \]

}

%%%%%%%%%%%%%%%%%%%%%%%

\qs{Exercise 1}{Apply the sequence $SXH$ to $|0\rangle$ and calculate:

  \begin{itemize}
    \item The final state vector

    \item The probabilities of measuring $|0\rangle$ and $|1\rangle$

    \item The possible post-measurement states

\end{itemize}}

\sol{
  \begin{align*}
    H|0\rangle &= \frac{1}{\sqrt{2}}(|0\rangle + |1\rangle) \\
    %
    XH|0\rangle &= \frac{1}{\sqrt{2}}(|1\rangle + |0\rangle) =
    \frac{1}{\sqrt{2}}(|0\rangle + |1\rangle) \\
    %
    SXH|0\rangle &= \frac{1}{\sqrt{2}}(|0\rangle + i|1\rangle)
  \end{align*}

  Therefore:

  \begin{itemize}
    \item Final state: $|\psi\rangle = \frac{1}{\sqrt{2}}(|0\rangle +
      i|1\rangle)$

    \item Measurement probabilities: $P(0) = P(1) = \frac{1}{2}$

    \item Post-measurement states: Either $|0\rangle$ or $|1\rangle$ with
      equal probability
\end{itemize}}

\vspace{0.3cm}

\qs{Exercise 2}{Show that the Hadamard gate is its own inverse by calculating
$H^2$.}

\sol{
  \[
    H^2 = \begin{bmatrix}
      \frac{1}{\sqrt{2}} & \frac{1}{\sqrt{2}} \\
      \frac{1}{\sqrt{2}} & -\frac{1}{\sqrt{2}}
    \end{bmatrix}
    \begin{bmatrix}
      \frac{1}{\sqrt{2}} & \frac{1}{\sqrt{2}} \\
      \frac{1}{\sqrt{2}} & -\frac{1}{\sqrt{2}}
    \end{bmatrix}
    %
    = \begin{bmatrix}
      1 & 0 \\
      0 & 1
    \end{bmatrix} = I
  \]
}

\qs{Exercise 3}{Calculate the effect of applying $R_Z(\pi/2)$ to the state
$|+\rangle$.}

\sol{
  \[
    R_Z(\pi/2)|+\rangle = \begin{bmatrix}
      e^{-i\pi/4} & 0 \\
      0 & e^{i\pi/4}
    \end{bmatrix}
    \begin{bmatrix}
      \frac{1}{\sqrt{2}} \\
      \frac{1}{\sqrt{2}}
    \end{bmatrix}
    = \begin{bmatrix}
      \frac{1}{\sqrt{2}} \\
      \frac{i}{\sqrt{2}}
    \end{bmatrix}
    = |+i\rangle
  \]
}


\section{Lecture 5: Other Quantum Gates, Measurement, Multi-Qubit Systems}
\label{sec:lecture5}

\index{quantum gates!single-qubit gates}
\dfn{Single-Qubit Gates}{Quantum gates manipulate individual qubits.
  Single-qubit gates are represented by unitary matrices that operate on a
single qubit.}

The following are common single-qubit gates:

\begin{itemize}
  \index{quantum gates!single-qubit gates!Hadamard gate}
  \item \textbf{Hadamard Gate (H):}
    \[
      \boxed{
        H = \frac{1}{\sqrt{2}} \begin{pmatrix} 1 & 1 \\ 1 & -1 \end{pmatrix}
      }
    \]

    Creates superposition: \boxed{H\zero = \ket{+}, H\one = \ket{-}}

    \vspace{0.3cm}

    Properties:

    \begin{itemize}
      \item Self-inverse: $H^2 = I$
      \item Maps computational basis to $\ket{\pm}$ basis:
        \begin{align*}
          \ket{+} &= \frac{1}{\sqrt{2}}(\zero + \one) \\
          \ket{-} &= \frac{1}{\sqrt{2}}(\zero - \one)
        \end{align*}
    \end{itemize}

    \nt{
      \[
        H \zero = \ket{+}, \quad H \one = \ket{-}, \quad H \ket{+} = \zero,
        \quad H \ket{-} = \one
      \]
    }

    \index{quantum gates!single-qubit gates!Phase gate}
  \item \textbf{Phase Gate (S):}

    \[
      \boxed{
        S = \begin{pmatrix} 1 & 0 \\ 0 & i \end{pmatrix}
      }
    \]

    Adds a $\pi/2$ phase to $\one$, so it is also referred to as the "$\pi$/4
    gate" due to $\theta/2$ term in the Bloch sphere equation.

    \vspace{0.3cm}

    \textbf{Properties:}

    \begin{itemize}
      \item \textbf{Unitary but not Hermitian}
      \item $S^2 = Z$
      \item Effect on $\ket{+}$ : \boxed{S\ket{+} =
        \frac{1}{\sqrt{2}}(\zero + i\one)}
    \end{itemize}

    \index{quantum gates!single-qubit gates!T gate}
  \item \textbf{T Gate:}

    \[
      \boxed{
        T = \begin{pmatrix} 1 & 0 \\ 0 & e^{i\pi/4} \end{pmatrix}
      }
    \]

    Adds a $\pi/4$ phase to $\one$, so it is also known as the "$\pi$/8 gate".

    \vspace{0.3cm}

    \textbf{Properties:}

    \begin{itemize}
      \item $T^2 = S$
      \item $T^4 = Z$
      \item Often used in quantum error correction
    \end{itemize}

    \index{quantum gates!single-qubit gates!General Phase gate}
  \item \textbf{General Phase Gate } $P(\theta)$:

    \[
      P(\theta) = \begin{pmatrix} 1 & 0 \\ 0 & e^{i\theta} \end{pmatrix}
    \]

    Generalizes S and T gates: \boxed{S = P(\pi/2), T = P(\pi/4)}

\end{itemize}

\ex{Example of Applying the Hadamard Gate}{
  \begin{align*}
    H\zero &= \frac{1}{\sqrt{2}} \begin{bmatrix} 1 & 1 \\ 1 & -1
      \end{bmatrix} \begin{bmatrix} 1 \\ 0 \end{bmatrix} \\
      & \frac{1}{\sqrt{2}} \begin{bmatrix} 1 \\ 1 \end{bmatrix} = \ket{+}
    \end{align*}
  }

  \nt{
    The following properties arise from applying the Hadamard gate:
    \begin{align*}
      Z &= H X H \\
      X &= H Z H
    \end{align*}

    \begin{proof}
      \[
        HXH = \frac{1}{\sqrt{2}}\begin{pmatrix} 1 & 1 \\ 1 & -1 \end{pmatrix}
        \begin{pmatrix} 0 & 1 \\ 1 & 0 \end{pmatrix}
        \frac{1}{\sqrt{2}}\begin{pmatrix} 1 & 1 \\ 1 & -1 \end{pmatrix} =
        \begin{pmatrix} 1 & 0 \\ 0 & -1 \end{pmatrix} = Z
      \]
    \end{proof}


  }

  \index{quantum measurement}
  \subsection*{Back to Measurement}

  Measurement collapses quantum states to basis states with
  probabilities determined by amplitudes.

  \index{quantum measurement!measurement bases}
  \begin{itemize}
    \item \textbf{Z-basis:} Standard computational basis ($\zero$, $\one$)

      \begin{itemize}[label={*}]
        \item For state $\ket{\psi} = \alpha\zero + \beta\one$:
          \begin{align*}
            P(0) &= ||\alpha||^2 \\
            P(1) &= ||\beta||^2
          \end{align*}
      \end{itemize}

    \item \textbf{X-basis:} Hadamard basis ($\ket{+}$, $\ket{-}$)

      \begin{itemize}[label={*}]
        \item Measure in Z-basis after applying H gate

        \item $P(+) = |\bra{+}\psi\rangle|^2$

        \item $P(-) = |\bra{-}\psi\rangle|^2$
      \end{itemize}

    \item \textbf{Y-basis:} Eigenstates of Y

      \begin{itemize}[label={*}]
        \item $\ket{+i} = \frac{1}{\sqrt{2}}(\zero + i\one)$

        \item $\ket{-i} = \frac{1}{\sqrt{2}}(\zero - i\one)$
      \end{itemize}

  \end{itemize}


  \index{multi-qubit systems}
  \subsection*{Multi-Qubit Systems}

  States for multiple qubits are represented as
  tensor products:

  \[
    \boxed{
      \ket{\psi} = \sum_{k=0}^{k = 2^n-1} \alpha_k \ket{k}, \quad \sum
      ||\alpha_k||^2 = 1
    }
  \]

  \noindent
  \textbf{Properties of tensor products:}
  \index{vector!tensor product!properties@\textit{properties}}

  \begin{itemize}
    \item \textbf{Not commutative:} $(\ket{0} \otimes \ket{1} \neq \ket{1}
      \otimes \ket{0})$

    \item Associative: $((\ket{a} \otimes \ket{b}) \otimes \ket{c} = \ket{a}
      \otimes (\ket{b} \otimes \ket{c}))$

    \item Distributive: $((\alpha\ket{a} + \beta\ket{b}) \otimes \ket{c} =
      \alpha(\ket{a} \otimes \ket{c}) + \beta(\ket{b} \otimes \ket{c}))$
  \end{itemize}

  \vspace{0.3cm}

%%%%%%%%%%%%%%%%%%%%%%%%%%%

  \qs{Exercise 1}{Prove that the Hadamard gate is unitary and Hermitian.}

  \sol{
    To prove H is unitary and Hermitian:

    \begin{align*}
      H^\dagger &= \frac{1}{\sqrt{2}}\begin{pmatrix} 1 & 1 \\ 1 & -1
      \end{pmatrix} = H \\
      HH &= \frac{1}{2}\begin{pmatrix} 1 & 1 \\ 1 & -1
        \end{pmatrix}\begin{pmatrix} 1 & 1 \\ 1 & -1 \end{pmatrix} =
        \begin{pmatrix} 1 & 0 \\ 0 & 1 \end{pmatrix} = I
      \end{align*}

      Thus, H is both unitary ($HH^\dagger = I$) and Hermitian ($H = H^\dagger$).
    }

    \vspace{0.3cm}

    \qs{Exercise 2}{For $\ket{\psi} = \frac{1}{\sqrt{2}}(\ket{00} + \ket{11})$,
    find measurement probabilities for $\ket{00}$ and $\ket{11}$.}

    \sol{
      For $\ket{\psi} = \frac{1}{\sqrt{2}}(\ket{00} + \ket{11})$:

      \begin{align*}
        P(00) &= |\bra{00}\psi\rangle|^2 = \left|\frac{1}{\sqrt{2}}\right|^2 =
        \frac{1}{2} \\
        P(11) &= |\bra{11}\psi\rangle|^2 = \left|\frac{1}{\sqrt{2}}\right|^2 =
        \frac{1}{2}
      \end{align*}
    }

    \vspace{0.3cm}

    \qs{Exercise 3}{Determine if $U = \frac{1}{\sqrt{2}}\begin{pmatrix} 1 & i \\
    i & 1 \end{pmatrix}$ is unitary.}

    \sol{
      To verify unitarity, compute $UU^\dagger$:

      \begin{align*}
        U^\dagger &= \frac{1}{\sqrt{2}}\begin{pmatrix} 1 & -i \\ -i & 1
        \end{pmatrix} \\
        UU^\dagger &= \frac{1}{2}\begin{pmatrix} 1 & i \\ i & 1
          \end{pmatrix}\begin{pmatrix} 1 & -i \\ -i & 1 \end{pmatrix} =
          \begin{pmatrix} 1 & 0 \\ 0 & 1 \end{pmatrix}
        \end{align*}

        $\therefore$ U is unitary.
      }

      \vspace{0.3cm}

      \qs{Exercise 4}{If we apply H $\otimes$ H to $\ket{00}$, what state do we
      get?}
      \sol{

        \begin{align*}
          (H \otimes H)\ket{00} &= (H\ket{0}) \otimes (H\ket{0}) \\
          &= \frac{1}{\sqrt{2}}(\ket{0} + \ket{1}) \otimes
          \frac{1}{\sqrt{2}}(\ket{0} + \ket{1}) \\
          &= \frac{1}{2}(\ket{00} + \ket{01} + \ket{10} + \ket{11})
        \end{align*}

        This creates an equal superposition of all two-qubit basis states.
      }

%%%%%%%%%%%%%%%%%%%%%%%%%%%%%%%%%%%%%%%%%%%%%%
% End of Lecture 5
%%%%%%%%%%%%%%%%%%%%%%%%%%%%%%%%%%%%%%%%%%%%%%

\section{Lecture 6: Multi-Qubit Gates and Circuit Construction}
\label{sec:lecture6}

$n$-qubit gates are unitary transformations that operate on $n$ qubits. This
section explores key multi-qubit gates, their properties, and how to
construct quantum circuits using these gates.

\index{quantum gates!multi-qubit gates}
\subsection*{Controlled Gates}

\index{quantum gates!multi-qubit gates!controlled gates}
\dfn{Controlled X Gate}{(CNOT) is a two-qubit gate that flips the target qubit
  if the control qubit is in state $\ket{1}$, and does nothing if the control
  qubit is in state $\ket{0}$. The matrix representation of CNOT is given by:

  \[
    CNOT = CX = \begin{pmatrix}
      1 & 0 & 0 & 0 \\
      0 & 1 & 0 & 0 \\
      0 & 0 & 0 & 1 \\
      0 & 0 & 1 & 0
    \end{pmatrix}
  \]

  Representing the gate in a circuit diagram:

  \begin{align*}
    \begin{quantikz}
      \lstick{$\ket{c}$} & \ctrl{1} & \qw \\
      \lstick{$\ket{t}$} & \gate{X} & \qw \\
    \end{quantikz}
  \end{align*}

  The control qubit is denoted by $\ket{c}$ and the target qubit is denoted by
$\ket{t}$.}

\vspace{0.3cm}

Applying the CNOT gate to a two-qubit state $\ket{\psi} = \alpha\ket{00} +
\beta\ket{01} + \gamma\ket{10} + \delta\ket{11}$:

\begin{equation}
  \begin{aligned}
      % CX |00>
    CX \ket{00} & = \begin{bmatrix}
      1 & 0 & 0 & 0 \\
      0 & 1 & 0 & 0 \\
      0 & 0 & 0 & 1 \\
      0 & 0 & 1 & 0
      \end{bmatrix} \begin{bmatrix}
      1 \\ 0 \\ 0 \\ 0
      \end{bmatrix} & = \begin{bmatrix}
      1 \\ 0 \\ 0 \\ 0
    \end{bmatrix} = \ket{00} \\
      % CX |01>
    CX \ket{01} & = \begin{bmatrix}
      1 & 0 & 0 & 0 \\
      0 & 1 & 0 & 0 \\
      0 & 0 & 0 & 1 \\
      0 & 0 & 1 & 0
      \end{bmatrix} \begin{bmatrix}
      0 \\ 1 \\ 0 \\ 0
      \end{bmatrix} & = \begin{bmatrix}
      0 \\ 1 \\ 0 \\ 0
    \end{bmatrix} = \ket{01} \\
      % CX |10>
    CX \ket{10} & = \begin{bmatrix}
      1 & 0 & 0 & 0 \\
      0 & 1 & 0 & 0 \\
      0 & 0 & 0 & 1 \\
      0 & 0 & 1 & 0
      \end{bmatrix} \begin{bmatrix}
      0 \\ 0 \\ 1 \\ 0
      \end{bmatrix} & = \begin{bmatrix}
      0 \\ 0 \\ 0 \\ 1
    \end{bmatrix} = \ket{11} \\
      % CX |11>
    CX \ket{11} & = \begin{bmatrix}
      1 & 0 & 0 & 0 \\
      0 & 1 & 0 & 0 \\
      0 & 0 & 0 & 1 \\
      0 & 0 & 1 & 0
      \end{bmatrix} \begin{bmatrix}
      0 \\ 0 \\ 0 \\ 1
      \end{bmatrix} & = \begin{bmatrix}
      0 \\ 0 \\ 1 \\ 0
    \end{bmatrix} = \ket{10}
  \end{aligned}
\end{equation}

\subsection*{Swapping Control and Target}

The control and target qubits of a CNOT gate can be swapped using Hadamard
gates:

\[
  (H \otimes H) \cdot CNOT_{\text{control,target}} \cdot (H \otimes H) =
  CNOT_{\text{target,control}}
\]

This transformation can be visualized in a circuit diagram:

\begin{align*}
  \begin{quantikz}
    \lstick{$\ket{c}$} & \gate{H} & \ctrl{1} & \gate{H} & \qw \\
    \lstick{$\ket{t}$} & \gate{H} & \gate{X} & \gate{H} & \qw \\
  \end{quantikz}
\end{align*}


% TODO: move this to lecture 7
\index{quantum gates!multi-qubit gates!SWAP gate}
\subsection*{SWAP Gate}

\dfn{SWAP Gate}{exchanges the states of two qubits. Its matrix representation
  is:

  \[
    SWAP = \begin{pmatrix}
      1 & 0 & 0 & 0 \\
      0 & 0 & 1 & 0 \\
      0 & 1 & 0 & 0 \\
      0 & 0 & 0 & 1
    \end{pmatrix}
  \]

  The action of SWAP on basis states is given by:

  \[
    SWAP\ket{ab} = \ket{ba}, \quad a,b \in \{0,1\}
  \]
}

\index{quantum gates!multi-qubit gates!Controlled-Z gate}
\subsection*{Controlled-Z Gate}

\dfn{Controlled-Z Gate}{(CZ) applies a phase flip if both qubits are in state
  $\ket{1}$:

  \[
    CZ = \begin{pmatrix}
      1 & 0 & 0 & 0 \\
      0 & 1 & 0 & 0 \\
      0 & 0 & 1 & 0 \\
      0 & 0 & 0 & -1
    \end{pmatrix}
  \]
}

% TODO: move this to lecture 7
\index{quantum gates!multi-qubit gates!Toffoli gate}
\subsection*{Toffoli Gate}

\dfn{Toffoli Gate}{(CCX) is a three-qubit gate with two control qubits and
  one target qubit. Its matrix representation is:

  \[
    Toffoli = \begin{pmatrix}
      1 & 0 & 0 & 0 & 0 & 0 & 0 & 0 \\
      0 & 1 & 0 & 0 & 0 & 0 & 0 & 0 \\
      0 & 0 & 1 & 0 & 0 & 0 & 0 & 0 \\
      0 & 0 & 0 & 1 & 0 & 0 & 0 & 0 \\
      0 & 0 & 0 & 0 & 1 & 0 & 0 & 0 \\
      0 & 0 & 0 & 0 & 0 & 1 & 0 & 0 \\
      0 & 0 & 0 & 0 & 0 & 0 & 0 & 1 \\
      0 & 0 & 0 & 0 & 0 & 0 & 1 & 0
    \end{pmatrix}
  \]

  The Toffoli gate is Hermitian and only flips the target qubit if both control
  qubits are in state $\ket{1}$.
}

\index{Circuit notation!multi-qubit gates}
\subsection*{Circuit Representations}

The standard circuit representations for these multi-qubit gates are:

\begin{align*}
  \begin{quantikz}
    % CNOT
    \lstick{CNOT:} & \ctrl{1} & \qw \\
    & \targ{} & \qw \\[0.5cm]
    % SWAP
    \lstick{SWAP:} & \swap{1} & \qw \\
    & \targX{} & \qw \\[0.5cm]
    % CZ
    \lstick{CZ:} & \ctrl{1} & \qw \\
    & \gate{Z} & \qw \\[0.5cm]
    % Toffoli
    \lstick{Toffoli:} & \ctrl{1} & \qw \\
    & \ctrl{1} & \qw \\
    & \targ{} & \qw
  \end{quantikz}
\end{align*}

\subsection*{Tensor Product Ordering and Circuit Representations}

When converting quantum circuits to mathematical expressions, it's important
to understand that tensor products are not commutative: $A \otimes B \neq B
\otimes A$ in general. However, we can modify the ordering of tensor products
in our mathematical representation as long as we maintain the dependencies
established by the circuit diagram. This leads to multiple valid mathematical
representations of the same quantum circuit.

\dfn{Bit Ordering Convention}{Due to the non-commutativity of tensor
  products, we adopt the convention of representing qubits from most
  significant to least significant in our mathematical expressions. For an
  $n$-qubit system:

  \[
  \ket{q_{n-1}q_{n-2}...q_1q_0} = \ket{q_{n-1}} \otimes \ket{q_{n-2}} \otimes
  ... \otimes \ket{q_1} \otimes \ket{q_0}
  \]

  where $q_0$ is the least significant qubit.
}

\noindent

For example, consider a circuit with two Hadamard gates applied to different
qubits:

\begin{align*}
  \begin{quantikz}
    \lstick{$\ket{q_1}$} & \gate{H} & \qw \\
    \lstick{$\ket{q_0}$} & \gate{H} & \qw
  \end{quantikz}
\end{align*}

This circuit can be represented mathematically in equivalent ways:

\[
  (H \otimes H)\ket{q_1q_0} = (H\ket{q_1}) \otimes (H\ket{q_0}) = H\ket{q_1}
  \otimes H\ket{q_0}
\]

\noindent
When gates have dependencies (like controlled operations), the ordering must
respect these dependencies. For the CNOT gate:

\begin{align*}
  \begin{quantikz}
    \lstick{$\ket{q_1}$} & \ctrl{1} & \qw \\
    \lstick{$\ket{q_0}$} & \gate{X} & \qw
  \end{quantikz}
\end{align*}

\noindent
The mathematical representation must preserve the control-target
relationship, though intermediate calculations may use different but
equivalent orderings:

\[
  CNOT_{1,0}\ket{q_1q_0} = CNOT(\ket{q_1} \otimes \ket{q_0})
\]

This flexibility in representation, while maintaining functional equivalence,
is particularly useful when analyzing complex quantum circuits or optimizing
quantum computations.


%%%%

% ● Reversibility property and no-cloning theorem
\section{Lecture 7:  More Multi-Qubit Gates, Reversibility Property,
No-Cloning Theorem}\label{sec:Lecture 7}

\subsection*{Review Questions}

\qs{Number of Measurement Bases}{
  How many different bases can we measure a $n$-qubit system in?
}

\sol{\boxed{\textbf{Infinite}}

  \[
    P(b) = \norm{\langle b | \psi \rangle}^2
  \]

  In quantum mechanics, a measurement basis for an $n$-qubit system is a set of
  orthonormal basis states in a $2^n$-dimensional Hilbert space. The most
  common measurement basis is the \textbf{computational basis}, given by
  $\{|0\rangle, |1\rangle\}^{\otimes n}$. However, we can measure in
  \textbf{any} orthonormal basis.

  The space of all possible measurement bases corresponds to the space of all
  possible orthonormal bases, which is parameterized by the unitary group
  $U(2^n)$. The set of all $2^n$-dimensional orthonormal bases is described
  by the unitary group $U(2^n)$, modulo the global phase $U(1)$. Since this
  space is continuous and has infinitely many parameters, there exist an
  \textbf{infinite} number of measurement bases.
}

\qs{Output of Quantum Circuit}{

  What is the output quantum state of the following quantum circuit:

  \[
    \begin{quantikz}
      q_2 \quad \lstick{$\zero$} & \gate{X} & \gate{X} & \meter{} \\
      q_1 \quad \lstick{$\zero$} & \gate{S} & \gate{X} & \meter{} \\
      q_0 \quad \lstick{$\zero$} & \gate{X} & & \meter{} \\
    \end{quantikz}
  \]
}

\sol{
  \boxed{011}

  \begin{itemize}
    \item For $q_2$, the two $X$ gates essentially cancel each other out as
      the gate is Hermitian ($XX = I$).
    \item For $q_1$, the $S$ gate does not affect the starting state $\zero$,
      and then the $X$ gate flips the signal to $\one$.
    \item For $q_0$, the $X$ gate simply flips the signal from $\zero$ to
      $\one$.
  \end{itemize}

  Remembering that we read the diagram top-down as the most significant bit
  to the least significant bit, respectively, the output is 001.
}

\subsection*{More Multi-Qubit Gates}

\subsubsection*{More 2-Qubit Gates}

\index{quantum gates!multi-qubit gates!CNOT with swapped target and control}
\dfn{$CX(q_0 \rightarrow q_1)$ Gate}{

  The control and target qubits of a CNOT gate can be swapped using Hadamard
  gates:

  \[
    (H \otimes H) \cdot CNOT_{\text{control,target}} \cdot (H \otimes H) =
    CNOT_{\text{target,control}}
  \]

  This transformation can be visualized in a circuit diagram:

  \begin{align*}
    \begin{quantikz}
      \lstick{$\ket{c}$} & \gate{H} & \ctrl{1} & \gate{H} & \qw \\
      \lstick{$\ket{t}$} & \gate{H} & \gate{X} & \gate{H} & \qw \\
    \end{quantikz}
  \end{align*}

  The $CX(q_0 \rightarrow q_1)$ gate, also known as $CNOT_{target,control}$,
  is a variant of the CNOT gate where the control and target qubits are
  swapped. Its matrix representation is:

  \[
    CNOT_{target,control} = \begin{pmatrix}
      1 & 0 & 0 & 0 \\
      0 & 0 & 0 & 1 \\
      0 & 0 & 1 & 0 \\
      0 & 1 & 0 & 0
    \end{pmatrix}
  \]

  This gate flips the target qubit ($q_1$) if the control qubit ($q_0$) is in
  state $\ket{1}$.
}

\subsubsection*{Derivation of the $CX(q_0 \rightarrow q_1)$ Gate}

The control and target qubits of a CNOT gate can be swapped using Hadamard
($H$) gates applied to both qubits. Mathematically, this operation is
represented as:

\[
  (H \otimes H) \cdot \text{CNOT}_{\text{control,target}} \cdot (H \otimes H)
  = \text{CNOT}_{\text{target,control}}
\]


The transformed CNOT gate is:

\[
  (H \otimes H) \cdot \text{CNOT} \cdot (H \otimes H).
\]

\vspace{0.3cm}

\textbf{Step 1: Compute $\text{CNOT} \cdot (H \otimes H)$}

First, multiply the CNOT matrix with $H \otimes H$:

\[
  \text{CNOT} \cdot (H \otimes H) = \begin{pmatrix}
    1 & 0 & 0 & 0 \\
    0 & 1 & 0 & 0 \\
    0 & 0 & 0 & 1 \\
    0 & 0 & 1 & 0
    \end{pmatrix} \cdot \frac{1}{2} \begin{pmatrix}
    1 & 1 & 1 & 1 \\
    1 & -1 & 1 & -1 \\
    1 & 1 & -1 & -1 \\
    1 & -1 & -1 & 1
  \end{pmatrix}
\]

This results in:

\[
  \text{CNOT} \cdot (H \otimes H) = \frac{1}{2} \begin{pmatrix}
    1 & 1 & 1 & 1 \\
    1 & -1 & 1 & -1 \\
    1 & -1 & -1 & 1 \\
    1 & 1 & -1 & -1
  \end{pmatrix}
\]

Notice that the third and fourth rows of the matrix switch places due to the
CNOT gate's effect.

\vspace{0.3cm}

\textbf{Step 2: Multiply $(H \otimes H)$ to the Above Result}

Now, multiply $H \otimes H$ from the left:

\[
  (H \otimes H) \cdot \frac{1}{2} \begin{pmatrix}
    1 & 1 & 1 & 1 \\
    1 & -1 & 1 & -1 \\
    1 & -1 & -1 & 1 \\
    1 & 1 & -1 & -1
  \end{pmatrix}
\]

Expanding this:

\[
  \frac{1}{2} \begin{pmatrix}
    1 & 1 & 1 & 1 \\
    1 & -1 & 1 & -1 \\
    1 & 1 & -1 & -1 \\
    1 & -1 & -1 & 1
    \end{pmatrix} \cdot \frac{1}{2} \begin{pmatrix}
    1 & 1 & 1 & 1 \\
    1 & -1 & 1 & -1 \\
    1 & -1 & -1 & 1 \\
    1 & 1 & -1 & -1
  \end{pmatrix}
\]

After computation, the resulting matrix is:

\[
  \text{CNOT}_{\text{target,control}} = \begin{pmatrix}
    1 & 0 & 0 & 0 \\
    0 & 0 & 0 & 1 \\
    0 & 0 & 1 & 0 \\
    0 & 1 & 0 & 0
  \end{pmatrix}
\]

This matrix corresponds to the CNOT gate with the control and target qubits
swapped.


\index{quantum gates!multi-qubit gates!SWAP gate}
\subsection*{SWAP Gate}

\dfn{SWAP Gate}{
  The SWAP gate exchanges the states of two qubits. Its matrix representation
  is:

  \[
    SWAP = \begin{pmatrix}
      1 & 0 & 0 & 0 \\
      0 & 0 & 1 & 0 \\
      0 & 1 & 0 & 0 \\
      0 & 0 & 0 & 1
    \end{pmatrix}
  \]

  The action of SWAP on basis states is given by:

  \[
    SWAP\ket{ab} = \ket{ba}, \quad a,b \in \{0,1\}
  \]
}


\paragraph{Effects of SWAP Gate}\label{par:Effects of SWAP Gate}
The SWAP gate interchanges the states of two qubits. For any 2-qubit
computational basis state, its action is:

\nt{
  \[
    \begin{aligned}
      SWAP\,\ket{00} &= \ket{00}, \\
      SWAP\,\ket{01} &= \ket{10}, \\
      SWAP\,\ket{10} &= \ket{01}, \\
      SWAP\,\ket{11} &= \ket{11}.
    \end{aligned}
  \]
}

Thus, for any superposition of 2-qubit states, the SWAP gate exchanges the
amplitudes corresponding to each qubit's position.


\index{quantum gates!$n$-qubit gates}
\subsection*{$n$-Qubit Gates}

Before we just looked at the 2-qubit version of the Controlled X gate, but
it extends to $n$-qubits.

\index{quantum gates!multi-qubit gates!Toffoli gate}
\subsection*{Toffoli Gate}

\dfn{Toffoli Gate}{(CCX) is a three-qubit gate with two control qubits and
  one target qubit. Its matrix representation is:

  \[
    \textsc{Toffoli} = \begin{pmatrix}
      1 & 0 & 0 & 0 & 0 & 0 & 0 & 0 \\
      0 & 1 & 0 & 0 & 0 & 0 & 0 & 0 \\
      0 & 0 & 1 & 0 & 0 & 0 & 0 & 0 \\
      0 & 0 & 0 & 1 & 0 & 0 & 0 & 0 \\
      0 & 0 & 0 & 0 & 1 & 0 & 0 & 0 \\
      0 & 0 & 0 & 0 & 0 & 1 & 0 & 0 \\
      0 & 0 & 0 & 0 & 0 & 0 & 0 & 1 \\
      0 & 0 & 0 & 0 & 0 & 0 & 1 & 0
    \end{pmatrix}
  \]

  The Toffoli gate is Hermitian and only flips the target qubit if both control
  qubits are in state $\ket{1}$.
}

\paragraph{Effects of Toffoli Gate}\label{par:Effects of Toffoli Gate}

The Toffoli (CCX) gate acts on a 3-qubit system, where the first two qubits
serve as control qubits and the third is the target. Its effect on the
computational basis states is:

\nt{
  \[
    \begin{aligned}
      \text{Toffoli}\,\ket{000} &= \ket{000}, \\
      \text{Toffoli}\,\ket{001} &= \ket{001}, \\
      \text{Toffoli}\,\ket{010} &= \ket{010}, \\
      \text{Toffoli}\,\ket{011} &= \ket{011}, \\
      \text{Toffoli}\,\ket{100} &= \ket{100}, \\
      \text{Toffoli}\,\ket{101} &= \ket{101}, \\
      \text{Toffoli}\,\ket{110} &= \ket{111}, \\
      \text{Toffoli}\,\ket{111} &= \ket{110}.
    \end{aligned}
  \]
}

In essence, the target qubit is flipped only when both control qubits are in
the \(\ket{1}\) state; otherwise, the state remains unchanged.

As you would expect, multi-controlled $X$ gates are Hermitian.

\ex{Quantum Circuit Showing Hermitian Property}{
  A quantum circuit composed of Hermitian gates (e.g., $X$, $Z$, $H$) will
  cancel out when the circuit is reversed, resulting in the identity operation.

  \begin{center}
    \begin{quantikz}
      \lstick{\ket{\psi}} & \gate{H} & \gate{Z} & \gate{X} & \gate{X} &
      \gate{Z} & \gate{H} & \qw & \rstick{\ket{\psi}}
    \end{quantikz}
  \end{center}

  In this example, the quantum circuit consists of a sequence of Hermitian
  gates: the Hadamard ($H$), Pauli-Z ($Z$), and Pauli-X ($X$) gates. These
  gates satisfy the property:

  \[
    H = H^\dagger, \quad X = X^\dagger, \quad Z = Z^\dagger
  \]

  Since these gates are their own inverses (\textit{i.e.,} $H H = I$, $X X =
  I$, and $Z Z = I$), if we apply the same sequence of gates in reverse
  order, they cancel out, leaving the identity operation.

  The circuit above first applies $H$, then $Z$, then $X$ twice (which
  cancels itself out), then $Z$ again, and finally $H$ again. This results in:

  \[
    H Z X X Z H = I
  \]

  Hence, the overall operation on the qubit is the identity transformation,
  meaning the final state remains the same as the initial state $\ket{\psi}$.
}

\vspace{0.3cm}

\noindent\rule{\linewidth}{0.2pt}

\vspace{0.3cm}

\index{Reversibility Property of Quantum Computing}
\dfn{Reversibility Property of Quantum Computing}{
  Quantum operations are inherently reversible due to the \textbf{unitary}
  nature of quantum gates. This means that any quantum circuit can be
  reversed by applying the inverse of each gate in the reverse order.
  Mathematically, if a quantum circuit is represented by a unitary matrix
  $U$, its reverse is represented by $U^{\dagger}$, and since quantum gates
  are unitary, they satisfy the property:

  \[
    U^{\dagger} U = U U^{\dagger} = I
  \]

  This reversibility is a fundamental difference between quantum and classical
  computing and is directly tied to the \textbf{no information loss}
  principle in quantum mechanics.
}

\paragraph{Why Reversibility is Important:}
In classical computing, operations such as the AND gate lose information.
For example, given the output of an AND gate, we cannot uniquely determine
the original input:

\[
  (0,0) \mapsto 0, \quad (0,1) \mapsto 0, \quad (1,0) \mapsto 0, \quad
  (1,1) \mapsto 1
\]

Since multiple inputs can produce the same output, information is lost,
making the operation \textbf{irreversible}.

\paragraph{Classical AND Gate (Irreversible)}
\begin{center}
  \begin{tabular}{cc}
    % Circuit diagram
    \begin{circuitikz} \draw
      % Input nodes and AND gate
      (0,1) node[left] {$A$} -- ++(1,0)
      (0,0) node[left] {$B$} -- ++(1,0)
      (2,0.5) node[and port] (and) {}
      % Connect everything
      (1,1) -- (and.in 1)
      (1,0) -- (and.in 2)
      (and.out) -- ++(1,0) node[right] {$A \land B$};
    \end{circuitikz}
    &
    % Truth table
    \begin{tabular}{|c|c|c|}
      \hline
      $A$ & $B$ & $A \land B$ \\
      \hline
      0 & 0 & 0 \\
      0 & 1 & 0 \\
      1 & 0 & 0 \\
      1 & 1 & 1 \\
      \hline
    \end{tabular}
  \end{tabular}
\end{center}

\noindent This classical AND gate demonstrates information loss
(irreversibility) because multiple distinct input states map to the same
output state. As shown in the truth table, three different input
combinations (0,0), (0,1), and (1,0) all produce the same output 0. Given
only the output 0, it is impossible to determine which of these three input
states generated it—this loss of information about the system's initial
state makes the operation irreversible.

\paragraph{Quantum Reversibility:}
In contrast, \textbf{quantum gates are always unitary}, meaning they
preserve the total amount of information. Given the final state of a
quantum system, we can always determine its previous state by applying the
inverse transformation. This is why quantum circuits must be composed of
reversible operations.

\ex{Example of a Reversible Quantum Circuit:}{
  \begin{center}
    \begin{quantikz}
      \lstick{\ket{\psi}} & \gate{H} & \gate{X} & \gate{X} & \gate{H} & \qw &
      \rstick{\ket{\psi}}
    \end{quantikz}
  \end{center}

  In this example, the quantum circuit consists of a sequence of unitary
  gates: the Hadamard ($H$) and the Pauli-X ($X$) gates. These gates satisfy:

  \[
    H = H^\dagger, \quad X = X^\dagger
  \]

  Since these gates are their own inverses (\textit{i.e.,} $H H = I$ and $X X
  = I$), if we apply the same sequence of gates in reverse order, they cancel
  out, leaving the identity operation:

  \[
    H X X H = I
  \]
}

\paragraph{Key Properties of Reversible Quantum Gates:}
\begin{enumerate}
  \item Every quantum gate \( U \) has an inverse \( U^\dagger \), ensuring
    that no information is lost.
  \item The composition of unitary gates remains unitary, preserving
    reversibility.
  \item Classical Toffoli and Fredkin
    \footnote{More on Fredkin gates: \url{https://en.wikipedia.org/wiki/Fredkin_gate}}
    are reversible and can be used to construct reversible classical circuits,
    which is why they are also fundamental in quantum computing.
  \item Measurement is \textbf{not} reversible, as it collapses the quantum
    state and introduces information loss.
\end{enumerate}

The reversibility of quantum computing is crucial for error correction,
fault-tolerant quantum computation, and simulating physical systems where
information is conserved.


\ex{Example of Reversing a Quantum Circuit}{
  Consider a circuit composed of gates $A$, $B$, and $C$:

  \[
    |\psi_{\text{final}}\rangle = CBA|\psi_{\text{initial}}\rangle.
  \]

  The reverse circuit applies $C^{\dagger}$, $B^{\dagger}$, and $A^{\dagger}$
  in reverse order:

  \[
    |\psi_{\text{reversed}}\rangle =
    A^{\dagger}B^{\dagger}C^{\dagger}|\psi_{\text{final}}\rangle =
    A^{\dagger}B^{\dagger}C^{\dagger}CBA|\psi_{\text{initial}}\rangle =
    |\psi_{\text{initial}}\rangle.
  \]
}


\index{Quantum No-Cloning Theorem}
\dfn{Quantum No-Cloning Theorem}{
  The no-cloning theorem states that it is impossible to create an identical
  copy of an arbitrary unknown quantum state. This result follows directly
  from the linearity of quantum mechanics.
}

\ex{Simple 2-Qubit Example}{

  Consider two qubits in states $\ket{\psi}$ and $\ket{0}$. The no-cloning
  theorem implies that there is no unitary operation $U$ such that:

  \[
    U(\ket{\psi} \otimes \ket{0}) = \ket{\psi} \otimes \ket{\psi}.
  \]
}

\index{Quantum No-Cloning Theorem!proof@\textit{proof}}
\begin{proof}
  \textbf{Proof of the Quantum No-Cloning Theorem}

  Assume a unitary operator $U$ exists that can clone an arbitrary quantum
  state $\ket{\psi}$. Then, for two different states $\ket{\psi}$ and
  $\ket{\phi}$, we would have:

  \[
    U(\ket{\psi} \otimes \ket{0}) = \ket{\psi} \otimes \ket{\psi},
  \]

  \[
    U(\ket{\phi} \otimes \ket{0}) = \ket{\phi} \otimes \ket{\phi}.
  \]

  Now, consider the superposition state $\ket{\xi} = a\ket{\psi} +
  b\ket{\phi}$. Applying $U$ to $\ket{\xi} \otimes \ket{0}$ should produce:

  \[
    U(\ket{\xi} \otimes \ket{0}) = aU(\ket{\psi} \otimes \ket{0}) +
    bU(\ket{\phi} \otimes \ket{0}) = a(\ket{\psi} \otimes \ket{\psi}) +
    b(\ket{\phi} \otimes \ket{\phi}).
  \]

  However, if $U$ could clone $\ket{\xi}$, the result should be:

  \[
    U(\ket{\xi} \otimes \ket{0}) = \ket{\xi} \otimes \ket{\xi} = (a\ket{\psi}
    + b\ket{\phi}) \otimes (a\ket{\psi} + b\ket{\phi}).
  \]

  Expanding this, we get:

  \[
    \ket{\xi} \otimes \ket{\xi} = a^2(\ket{\psi} \otimes \ket{\psi}) +
    ab(\ket{\psi} \otimes \ket{\phi}) + ab(\ket{\phi} \otimes \ket{\psi}) +
    b^2(\ket{\phi} \otimes \ket{\phi}).
  \]

  Comparing the two expressions, we see that the terms $\ket{\psi} \otimes
  \ket{\phi}$ and $\ket{\phi} \otimes \ket{\psi}$ appear in the expanded
  $\ket{\xi} \otimes \ket{\xi}$, but they do not appear in $a(\ket{\psi}
  \otimes \ket{\psi}) + b(\ket{\phi} \otimes \ket{\phi})$. This inconsistency
  demonstrates that a unitary operator $U$ cannot clone an arbitrary quantum
  state, proving the no-cloning theorem.
\end{proof}




% ==============================
% PHASE II: FUNDAMENTALS
% ==============================
\chapter{Phase II: Fundamentals of Quantum Algorithms}

% ● Entanglement
% ● Bell-state and GHZ state generation circuits
% ● Basic quantum algorithms
% ● Quantum computing using Cirq
\section{Lecture 8: Entanglement, Bell State, and Intro to \texttt{Cirq}}
\label{sec:lecture8}

\subsection*{Review from Previous Lecture}

\qs{Quantum Circuit Final State}{
  Given this circuit diagram and assuming an initial state $\ket{\psi_i}
  =\ket{00}$, what will the final state $\ket{\psi_f}$ measure?

  \[
    \begin{quantikz}
      q_1 \quad \lstick{$\zero$} & \ctrl{0} & \gate{X} & \ctrl{1} & \swap{1}
                                 & \gate{H} & \meter{} \\
      q_0 \quad \lstick{$\zero$} & \ctrl{0} & \qw & \gate{X} & \swap{-1} &
      \gate{H} & \meter{} \\
    \end{quantikz}
  \]
}

\sol{\boxed{\ket{\psi_f} = \ket{--}}}

%%%%%%%%%%%%%%%

\index{entanglement}
\subsection*{Entanglement}

In this section, we introduce quantum entanglement—a phenomenon where the
states of two or more qubits become inseparably linked, leading to
correlations between measurement outcomes that cannot be explained classically.

\dfn{Entanglement}{
  \textit{"\textbf{Quantum entanglement} is the phenomenon of a group of
    particles being generated, interacting, or sharing spatial proximity in a
    manner such that the quantum state of each particle of the group cannot be
    described independently of the state of the others, including when the
  particles are separated by a large distance."}\footnote{\url{https://en.wikipedia.org/wiki/Quantum_entanglement}}
}

\subsubsection*{What is Entanglement?}

To understand quantum entanglement, let's compare classical and quantum
communication systems:

\vspace{0.3cm}

\noindent
Consider two \textbf{classical} email clients, Alice and
Bob:

\begin{itemize}
  \item When Alice sends an email, she knows exactly what she sent
  \item Bob knows exactly what he received
  \item The message state is definite and independent
  \item Multiple copies can exist
\end{itemize}

\vspace{0.3cm}

\noindent
Now consider two \textbf{quantum} "email clients" with entangled qubits:

\begin{itemize}
  \item Neither Alice nor Bob knows their qubit's state before measurement
  \item When Alice measures her qubit, Bob's qubit instantly becomes correlated
  \item The qubits share a single quantum state
  \item No copies can be made (no-cloning theorem)
  \item The correlation exists regardless of distance
\end{itemize}

\vspace{0.3cm}

This "spooky action at a distance"\footnote{\href{https://en.wikipedia.org/wiki/Action\_at\_a\_distance\#\%22Spooky\_action\_at\_a\_distance\%22}{"Spooky action at a distance": \texttt{https://en.wikipedia.org/wiki/Action\_at\_a\_distance}}},
as Einstein called it, has no classical analog and is a fundamental feature
of quantum mechanics.

\index{entanglement!Bell state}
\subsubsection*{Bell-Pair Circuit}

The Bell-Pair circuit is the most analogous thing to quantum computing's
\texttt{Hello, world!} program. It prepares a maximally entangled two-qubit
state (a Bell state).


\[
  \begin{quantikz}
    q_1 \quad \lstick{$\zero$} & \gate{H} & \ctrl{1} & \meter{} \\
    q_0 \quad \lstick{$\zero$} & \qw & \gate{X} & \qw \\
  \end{quantikz}
\]


\index{entanglement!proof@\textit{proof}}
\begin{proof}{Showing Full Entanglement Mathematically}
  Starting with the initial state:

  \[
    \ket{\psi_0} = \ket{0}_1 \otimes \ket{0}_0 = \ket{00},
  \]

  we first apply a Hadamard gate on qubit \(q_1\):

  \[
    H\ket{0} = \frac{1}{\sqrt{2}}(\ket{0} + \ket{1}).
  \]

  Thus, the state after the Hadamard becomes:

  \[
    (H \otimes I)\ket{00} = \frac{1}{\sqrt{2}}(\ket{0} + \ket{1}) \otimes
    \ket{0} = \frac{1}{\sqrt{2}}(\ket{00} + \ket{10}).
  \]

  Next, we apply the CNOT gate with qubit \(q_1\) as the control and qubit
  \(q_0\) as the target. The CNOT gate acts as follows:

  \[
    \text{CNOT}\,\ket{00} = \ket{00}, \quad \text{CNOT}\,\ket{10} = \ket{11}.
  \]

  Thus, the final state after the CNOT is:

  \[
    \boxed{
      \ket{\psi_{\text{Bell}}} = \frac{1}{\sqrt{2}}(\ket{00} + \ket{11}).
    }
  \]

  This state is fully entangled since it cannot be factored as a tensor
  product of individual qubit states.

\end{proof}

\vspace{0.3cm}

\noindent
\textbf{
  As a consequence of being fully entangled:
}

\begin{enumerate}
  \item Entangled circuits cannot be expressed as a product state. For
    example, the Bell state

    \[
      \ket{\psi_{\text{Bell}}} = \frac{1}{\sqrt{2}}(\ket{00} + \ket{11})
    \]

    cannot be decomposed into \(\ket{\phi}_1 \otimes \ket{\chi}_0\).

  \item Measurement outcomes are correlated. For instance, if qubit \(q_1\)
    is measured as \(\ket{0}\), then qubit \(q_0\) will also be found in the
    state \(\ket{0}\); similarly, if \(q_1\) is measured as \(\ket{1}\), then
    \(q_0\) will be \(\ket{1}\).

\end{enumerate}


\index{entanglement!n-qubit Bell state}
\subsubsection*{Extending the Bell-Pair Circuit to a $n$-Qubit System}

The Bell state can be generalized to an \(n\)-qubit system, often taking the
form of a Greenberger–Horne–Zeilinger (GHZ) state:

\index{entanglement!GHZ state}

\[
  \begin{quantikz}
    q_{n - 1}\quad \lstick{$\zero$} & \gate{H} & \ctrl{1} & \qw & \qw & \qw \\
    q_{n -2} \quad \lstick{$\zero$} & \qw & \gate{X} & \ctrl{0} & \qw & \qw \\
    \vdots \\
    q_{1} \quad \lstick{$\zero$} & \qw & \qw & \qw & \gate{X} & \ctrl{1} \\
    q_{0} \quad \lstick{$\zero$} & \qw & \qw & \qw & \qw & \gate{X} \\
  \end{quantikz}
\]

\vspace{0.3cm}

\noindent
\textbf{
  The $n$-qubit Bell circuit follows a clear pattern:
}

\begin{enumerate}
  \item Initial Hadamard gate on the topmost qubit creates superposition
  \item Cascade of CNOT gates propagates the entanglement down
  \item Each CNOT uses the previous qubit as control and current qubit as
    target
  \item Final state has all qubits entangled together
\end{enumerate}

The resulting state:

\[
  \boxed{
    \ket{\psi_f} = \frac{1}{\sqrt{2}}\lt(\ket{00\cdots 00} +
    \ket{11\cdots11}\rt)
  }
\]

represents a superposition where all qubits are either all 0 or all 1.
Measuring any qubit collapses the entire state, forcing all other qubits to
match the measured value.

\vspace{0.3cm}

\nt{
  It is important to note that there are equivalent diagrams that can be
  drawn for the $n$-qubit Bell state circuit, and that you should not be
  fooled by diagrams that do not match the general diagram shown above.
}

\ex{Alternative $n$-Qubit Bell Circuit}{

  \[
    \begin{quantikz}
      q_{n-1} \quad \lstick{$\zero$} & \gate{H} & \ctrl{1} & \ctrl{2} &
      \ctrl{3} & \qw \\
      q_{n-2} \quad \lstick{$\zero$} & \qw & \gate{X} & \qw & \qw & \qw \\
      q_{1} \quad \lstick{$\zero$} & \qw & \qw & \gate{X} & \qw & \qw \\
      q_{0} \quad \lstick{$\zero$} & \qw & \qw & \qw & \gate{X} & \qw
    \end{quantikz}
  \]

  This alternative representation achieves the same final state by having the
  first qubit control all CNOT operations directly. While functionally
  equivalent, this requires longer-range interactions which may be harder to
  implement on real quantum hardware.
}


\aside{\textbf{Shorthand Notation Used for Final Quantum States}\\

  Common notational shortcuts can be confusing for beginners. Here are some
  important distinctions:

  \begin{itemize}
    \item $\ket{+1}$ typically means $\frac{1}{\sqrt{2}}(\ket{01} +
      \ket{11})$, not $\ket{+} \otimes \ket{1}$

    \item $\ket{-1}$ represents $\frac{1}{\sqrt{2}}(\ket{01} - \ket{11})$,
      not to be confused with $-\ket{1}$ or $Z\ket{1}$

    \item In general, $\ket{\pm x}$ represents $\frac{1}{\sqrt{2}}(\ket{0x}
      \pm \ket{1x})$ where $x$ is a bit string

    \item The notation $\ket{\psi^+}$ and $\ket{\psi^-}$ is often used for
      Bell states $\frac{1}{\sqrt{2}}(\ket{01} \pm \ket{10})$
  \end{itemize}

  Always check the context and definitions when encountering shorthand
  notation, as conventions may vary between sources.
}

\index{entanglement!partial entanglement}
\paragraph{Partial Entanglement}

There also exists the middle ground of partially-entangled systems where only
some of the qubits are entangled while others have independent states.

\[
  \begin{quantikz}
    q_2 \quad \lstick{$\zero$} & \gate{H} & \qw & \qw \\
    q_1 \quad \lstick{$\zero$} & \gate{H} & \ctrl{1} & \qw \\
    q_0 \quad \lstick{$\zero$} & \qw & \gate{X} & \qw
  \end{quantikz}
\]

In this case, \(q_1\) and \(q_0\) become entangled through the CNOT gate,
while \(q_2\) remains in a superposition state but independent of the other
qubits. The final state is:

\[
  \ket{\psi_f} = \frac{1}{\sqrt{2}}\ket{+}_2 \otimes
  \frac{1}{\sqrt{2}}(\ket{00}_{10} + \ket{11}_{10})
\]

%%%%%%%%%%%%%%%

\subsection*{Getting Started with \texttt{Cirq}}

\index{Cirq}
\texttt{Cirq}\footnote{\texttt{Cirq} documentation:
\url{https://quantumai.google/cirq}} is an open-source framework for
programming quantum computers, developed by Google Quantum AI. It provides
tools for creating, manipulating, and optimizing quantum circuits, as well as
simulating quantum computations.

\vspace{0.3cm}

\noindent
\textbf{
  Key features of \texttt{Cirq} include:
}

\begin{itemize}
  \item Native support for quantum gates and circuits
  \item Built-in quantum simulator
  \item Tools for noise modeling and error mitigation
  \item Integration with TensorFlow Quantum
\end{itemize}

\index{Cirq!introduction}
\subsubsection*{Overview and Installation}
Cirq allows you to work with qubits, gates, and operations to build circuits.
To begin, install \texttt{Cirq} in your Python environment (a Google Colab
workbook is my personal recommendation):

\begin{minted}{python}
  pip install --quiet cirq
\end{minted}

Then import the library:

\begin{minted}{python}
  import cirq
\end{minted}

\subsubsection*{Defining Qubits}
Cirq supports different types of qubits:
\begin{itemize}

  \item \textbf{NamedQubit}: Qubits with custom names.

  \item \textbf{LineQubit}: Qubits arranged in a linear array.

  \item \textbf{GridQubit}: Qubits arranged in a 2D grid.
\end{itemize}

For example, to create line qubits:

\begin{minted}{python}
  q0, q1, q2 = cirq.LineQubit.range(3)
\end{minted}

\subsubsection*{Gates and Operations}

In Cirq, a \emph{Gate} represents an abstract quantum operation, while an
\emph{Operation} is a gate applied to specific qubits.

\begin{itemize}
  \item Single-qubit gates: \texttt{cirq.X}, \texttt{cirq.H},
    \texttt{cirq.S}, etc.

  \item Two-qubit gates: \texttt{cirq.CNOT}, \texttt{cirq.CZ},
    \texttt{cirq.SWAP}, etc.

\end{itemize}

Example: Applying a Hadamard gate to qubit \(q0\) and a CNOT with \(q0\) as
control and \(q1\) as target:

\begin{minted}{python}
  circuit = cirq.Circuit()
  circuit.append(cirq.H(q0))
  circuit.append(cirq.CNOT(q0, q1))
\end{minted}

\subsubsection*{Circuits and Moments}
A \emph{Circuit} in Cirq is composed of \emph{Moments}—collections of
operations that occur simultaneously (on disjoint sets of qubits). When you
append operations, Cirq automatically schedules them into moments.

\subsubsection*{Simulation and Measurement}

Cirq provides a simulator to compute the resulting state vector or sample
measurement outcomes.

\begin{itemize}
  \item \texttt{simulate()}: Returns the final state vector.

  \item \texttt{run()}: Samples measurement outcomes.
\end{itemize}

\textbf{Example: Creating and Simulating a Bell State}
\begin{minted}{python}
  import cirq

  # Create two line qubits.
  q0, q1 = cirq.LineQubit.range(2)

  # Construct the Bell state circuit.
  bell_circuit = cirq.Circuit(
  cirq.H(q0),         # Create superposition on q0.
  cirq.CNOT(q0, q1)   # Entangle q0 with q1.
  )

  print("Bell Circuit:")
  print(bell_circuit)

  # Initialize the simulator.
  simulator = cirq.Simulator()

  # Simulate to obtain the final state vector.
  result = simulator.simulate(bell_circuit)
  print("Final state vector:")
  print(result.final_state_vector)

  # Add measurement and run the circuit.
  bell_circuit.append(cirq.measure(q0, q1, key='result'))
  samples = simulator.run(bell_circuit, repetitions=1024)
  print("Measurement histogram:")
  print(samples.histogram(key='result'))
\end{minted}

\subsubsection*{Advanced Topics in Cirq}

Cirq also supports:

\begin{itemize}

  \item \textbf{Parameter Sweeps}: Optimize variational algorithms using free
    parameters.

  \item \textbf{Unitary Access and Decomposition}: Retrieve matrix
    representations and decompose complex gates.

  \item \textbf{Transformers}: Automatically merge or re-schedule operations
    for hardware compatibility.

\end{itemize}

\textbf{Example: Sweeping a Parameter in an \(X\) Gate}
\begin{minted}{python}
  import sympy
  import matplotlib.pyplot as plt

  # Define a qubit.
  q = cirq.GridQubit(1, 1)

  # Create a circuit with an X gate raised to a symbolic exponent.
  circuit = cirq.Circuit(cirq.X(q) ** sympy.Symbol('t'),
  cirq.measure(q, key='m'))

  # Sweep the parameter 't' from 0 to 2.
  param_sweep = cirq.Linspace('t', start=0, stop=2, length=200)

  # Simulate the sweep.
  simulator = cirq.Simulator()
  trials = simulator.run_sweep(circuit, param_sweep, repetitions=1000)

  # Plot the frequency of measuring 1.
  x_data = [trial.params['t'] for trial in trials]
  y_data = [trial.histogram(key='m')[1] / 1000.0 for trial in trials]
  plt.scatter(x_data, y_data)
  plt.xlabel("Parameter t")
  plt.ylabel("Frequency of 1")
  plt.show()
\end{minted}

%%%%%%%%%%%%%%%%%%%%%%%%%%%%%%%%%%%%%%%%%%%%%%
% End of Lecture 8
%%%%%%%%%%%%%%%%%%%%%%%%%%%%%%%%%%%%%%%%%%%%%%


% ● Grover's search algorithm and coding
% LECTURES TBD

% ● Introduction to variational quantum algorithms
% ● Training and optimization of variational algorithms
% LECTURES TBD

% ==============================
% PHASE III: ADVANCED
% ==============================
% \chapter{Phase III: Advanced Quantum Algorithms}
% LECTURES TBD


% ==============================
% PHASE IV: SPECIAL TOPICS
% ==============================
% \chapter{Phase IV: Special Topics in Quantum Computing}
% LECTURES TBD


% ==============================
% PHASE V: CONCLUDING LECTURES
% ==============================
% \chapter{Phase V: Concluding Lectures}
% LECTURES TBD


% ==============================
% Appendix
% ==============================

\begin{appendices}\label{sec:Appendix}
  \printindex
\end{appendices}

% ==============================

\end{document}
